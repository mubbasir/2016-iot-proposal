\section{Data Management Plan}

\subsection{Types of Data} 
The proposed research will generate the following material and data that might be of use to the general scientific community: 
\begin{itemize}
\item Open-source software, either in the form of stand-along packages or modules for existing packages, such as the following: 
	SteerSuite, developed by Co-PI Mubbasir Kapadia, which is an open framework for developing, evaluating and sharing steering algorithms,
\item Robot Operating System (ROS), a popular middleware for communicating with robotic devices and sensors, 
\item Open Motion Planning Library (OMPL), a software package developed by Rice University for sharing motion planning algorithms 
\end{itemize}

Data collected from experiments evaluating different approaches for sensing and modeling heterogeneous cyber-human IoTs, as well as for the control of IoTs. The experiments will correspond to simulated challenges, using models of real systems and crowds, as well as physical experiments in office and classroom environments given the platforms developed as part of this project. The data will correspond to the available information to the overall system (e.g., the crowd’s configuration), and the choices made by different algorithms. 

\subsection{Standards and Formats} 
The input files and software developed will be compatible with popular protocols in the research community, such as those followed by the ROS software package. For instance, the control and planning methods will be decomposed into multiple processes that will communicate using the messaging infrastructure of ROS. This will also allow the processes to be compatible with existing and future packages developed by different research teams that follow a similar protocol and use the same middleware. 
The format for the data will be documented and provided in parallel with the data. An effort will be made to follow standards in the motion planning community (e.g., current format used by OMPL) and it is expected that the data will be stored in YAML files that will be compatible with ROS’ parameter server and easily readable by people. 

\subsection{Provisions for Archiving and Preservation}
The electronic data will be stored on multiple workstations and notebooks used by the PIs and the graduate students involved in the project. They will be backed up through the use of versioning control systems (Mercurial, Subversion) on the servers of the Computer Science Department at Rutgers University. These servers are backed up regularly off-site. The software modules will be also stored externally at online repositories, such as BitBucket. 

No additional funding or special institutional commitment or special preparation for long-term preservation will be required for the preservation of the data as described above. 

\subsection{Access and Re-Use Policies and Provisions}
The research groups websites’ will provide access to data as they will become available, as well as the software modules that will be developed, documented and prepared for sharing. Sharing of data and software modules will take place no longer than a year after their generation. If requested, access to data or software will be provided earlier via contact with the PIs. 

The software will be using the BSD license, an open-source software license. BSD is a permissive free soft- ware license, which introduces only minimal requirements about how the software can be redistributed. 

There is no anticipation of significant ethical or intellectual property or confidentiality or copyright issues involved with the acquisition of the data. In the event that discoveries or inventions are made in direct connection with this data, access to the data will be granted upon request once appropriate invention disclosures and/or provisional patent filings are made. 

\subsection{Plans for Transition or Termination of the Data Collection}
The data and software modules will be preserved for at least three years beyond the award period, as required by NSF guidelines. The Principal Investigator, assisted by the Information Technology staff of the Computer Science Department at Rutgers University, will be responsible for managing the data beyond the expiration of the research project. No special preparation is needed for migrating, deleting or transitioning the data to another media into the future. 

The data acquired and preserved in the context of this proposal will be further governed by Rutgers University’s policies pertaining to intellectual property, record retention, and data management. 