We now proceed with identifying the main challenges that the proposed project will address, with a brief overview of the proposed research tasks (research challenges), followed by an introduction of the team and their expertise.


Research Challenge 1 (RC-1): Data collection and processing
…



Research Challenge 2 (RC-2): Knowledge discovery
…


Research Challenge 3 (RC-3): Security, privacy (anonymity) and trust

The Internet of Things is foreseen to bring a multitude of services with a vision of creating a smart self-configuring and interconnected world for the benefit of end users. With the extensive research and development of computer, communication and control technologies, it is possible to connect all things to the Internet such that the so-called Internet of Things (IoT) can be formed. These things may be equipped with devices such as sensors, actuators, and tags, in order to allow people and things to be connected anytime and anywhere, with anything and anyone. IoT will enable collaborations and communications among people and things, and among things themselves, which expand the current Internet and will radically change our personal, corporate, and community environments. However, a plethora of security, privacy, and trust challenges need to be addressed in order to fully realize this vision [20, 21, 22, 23]. When more and more things connect to the Internet, security and privacy issues become more serious, especially in the case that these things are equipped with actuators and can support control.
For better protection of secure communication and user privacy, including location, identity and behavior habits, it is necessary to develop anonymous communication theories, methods and key technologies of anonymous communication systems in all varieties of application environments. Anonymous communication is used to hide communication participants or communication relations to achieve effective protection for network nodes and user identities. Anonymous communication can address potential network security issues, and becomes one of the hot topics in the field of network and information security. With reference to security, data anonymity, confidentiality, and integrity need to be guaranteed, along with providing authentication and authorization mechanisms in order to prevent unauthorized users (i.e., humans and devices) to access any system. Concerning privacy requirement, both data and user personal information have to be confidentially manipulated since devices may manage sensitive information (e.g., user habits, locations, etc). Finally, trust is also a fundamental issue since the IoT environment is characterized by multiple devices that have to process and handle the data in compliance with user needs and rights.
To realize a trusted, secure, and privacy preserving social network of things, we plan to address in this proposal the following security, privacy, and trust management related tasks:

T-3.1: Define a new scheme of anonymizing things and representing them as avatars in the social network of things.

T-3.2: Authentication of data collected in fogs in the presence of malicious attackers, despite the attack surface being very broad, ranging from PHY layer to application layer.

T-3.3: Achieving end-to-end secure and privacy-preserving information flow monitoring between users and things, covering all sort of things and access networks using physiological and physical layer properties.

T-4.3: Design of secure, fast and friendly authentication schemes to allow users of the social network of things to access the data of interest through diverse mobile devices such as smartphones and tablets.