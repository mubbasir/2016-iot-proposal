From Rutgers:

The proposed highly interdisciplinary project combines computer vision, human modelling and simulation, optimization, robotics and autonomy, and control theory, and has the potential to impact several research areas. The proposed sensing, simulation, and optimization framework will have wide applicability in various scientific areas involving decision-making in heterogeneous and dynamic networks comprising human-operated, and autonomous sensors, with concepts that could easily generalize to event understanding in complex social, economical or cyber-biological systems, thus impacting multiple societal applications including urban infrastructure, emergency response, safety, and quality of life.

\noindent \textbf{Datasets.} We will release the reconstructed time-stamped behaviors that are collected as part of our real-world experiments in the classroom and the dining hall. This real-world dataset will include both the original, unprocessed, noisy, incomplete trajectories, as well as the processed trajectories for researchers to use and compare, when developing their own algorithms. We will also release synthetic datasets from our simulation experiments using the data-driven crowd simulator that will be developed as part of this research. All crowd datasets (both real and synthetic) will be captured for both the original, un-optimized environments, as well as the optimized layouts. We will also release extended designs actuated of environmental elements such as mobile queue separators and mobile furniture.

\noindent \textbf{Open-Source Software.} The PIs have an established record of releasing software packages related to crowd modeling and simulation ~\cite{steersuite,10.1109/TVCG.2014.251}. We will build on top of these foundations and extend our existing open-source platforms to include software solutions for: (1) crowd trajectory reconstruction, (2) data-driven crowd simulation, (3) static and dynamic analysis of environments, (4) crowd-aware environment optimization, and (5) environment reconfiguration planning.

\noindent \textbf{Kaggle Competitions}. We plan to organize 2 public data analytics competitions using the collected data, hosted on kaggle.com. Kaggle is a platform for predictive modelling and analytics competitions, which is often used in academia to incentivize students to tackle challenging problems through a combination of research, system building, and peer-competitions. The goals will include trajectory estimation and reconstruction, data-driven crowd modelling, and computer-assisted designs of environments.

\noindent \textbf{Curriculum Development and Outreach.} The PIs will generate educational material on the autonomy, hardware design, sensing and simulation aspects of \precise. PI Kapadia will design a graduate seminar on \emph{Crowd-Aware Smart Environments} that will introduce students to concepts in crowd simulation, and environment optimization. Advanced undergraduate students will also be permitted to enroll. PI Pavlovic will introduce an assignment in the machine learning course where students will learn generative models of crowd movement. The PI's will collectively organize a workshop on crowd-aware cyber-physical environments during CPS week (tentatively April 2017), which will bring together researchers from various disciplines including computer vision, machine learning, simulation, robotics, and potential adopters of the technology. Our research environment will provide interdisciplinary training ranging from distributed sensing and control, simulation and optimization, to robot motion planning.

\noindent \textbf{Under-Represented Groups and K-12 Level}. Our project can help attract a diverse group of students and broaden the diversity of students recruited into these and other STEM disciplines. Rutgers office of Enrollment Management conducts the Rutgers Future Scholars Program (RFS), which provides mentoring activities to grade 9-12 minority students from disadvantaged backgrounds and full scholarships to undergraduate programs. We will closely work with this office to participate in outreach, enrichment, and mentoring activities. We will also leverage other programs to recruit and mentor students from under-represented groups: the summer undergraduate Project RiSE (Research in Science and Engineering) for undergraduates from underrepresented populations for 8/1-week intensive summer research internships; and the Aresty Center for Rutgers Undergraduate Research- places Rutgers undergraduates in research laboratories on campus. PI Pavlovic has a multi-year track record of working with over 14 minority and female students through Aresty and RiSE, as well as multiple REUs. We will leverage the extensive diversity programs at Rutgers to recruit and support women and underrepresented minorities.
