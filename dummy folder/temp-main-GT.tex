\documentclass[10pt]{article}
%%%\documentclass{article}
%%%\usepackage[2-16]{pagesel}
%\usepackage[english,spanish]{babel}

\usepackage{times}
\usepackage{cite,verbatim,rotating}
\usepackage{subfigure}

\usepackage[paper=letterpaper,margin=1in,ignorefoot,ignorehead]{geometry}

%\usepackage{fullpage,setspace}
\usepackage{url}
%\usepackage{algorithmic}
%\usepackage{algorithm2e}
\usepackage{amsfonts}
\usepackage{amssymb}
%\setcounter{tocdepth}{3}
%\usepackage{simplemargins}
\usepackage{graphicx}
\usepackage{wrapfig}
%\usepackage{amssymb}
\usepackage{amsmath}
%\usepackage[dvips]{graphicx}
%%% remove comment delimiter ('%') and specify parameters if required
%\usepackage[dvips]{graphics}
%\setallmargins{1.0in}

\usepackage{xcolor}
\usepackage{xspace}


\newcommand{\todo}[1]{\textcolor{red}{#1}\xspace}
\newcommand{\avatar}{\textcolor{red}{avatar}\xspace}


\makeatletter
\renewcommand\section{\@startsection{section}{1}{\z@}%
                                    {-0.5ex \@plus -1ex \@minus -.2ex}%
                                    {1.0ex \@plus.2ex}%
                                    {\normalfont\Large\bfseries}}
\renewcommand\subsection{\@startsection{subsection}{2}{\z@}%
                                     {-0.5ex\@plus -1ex \@minus -.2ex}%
                                     {0.5ex \@plus .2ex}%
                                     {\normalfont\large\bfseries}}
\renewcommand\subsubsection{\@startsection{subsubsection}{3}{\z@}%
                                     {-0.5ex\@plus -1ex \@minus -.2ex}%
                                     {0.5ex \@plus .2ex}%
                                     {\normalfont\normalsize\bfseries}}
\renewcommand\paragraph{\@startsection{paragraph}{4}{\z@}%
                                    {-0.5ex \@plus -1ex \@minus.2ex}%
                                    {-2em}%
                                    {\normalfont\normalsize\bfseries}}
\renewcommand\subparagraph{\@startsection{subparagraph}{5}{\parindent}%
                                       {0.5ex \@plus1ex \@minus .2ex}%
                                       {-1em}%
                                      {\normalfont\normalsize\bfseries}}
\makeatother

\setlength{\abovedisplayskip}{2pt plus 2pt minus 2pt} \setlength{\belowdisplayskip}{2pt plus 2pt minus 2pt}
\setlength{\abovedisplayshortskip}{0pt} \setlength{\belowdisplayshortskip}{\belowdisplayskip}
\setlength{\itemsep}{0pt}

\setlength{\textfloatsep}{2 mm} \setlength{\dbltextfloatsep}{0 mm}
\setlength{\parindent}{0mm}
\setlength{\parskip}{0.4mm}

\newcommand{\mypage}[1]{\renewcommand{\thepage}{#1-- \arabic{page}}
\setcounter{page}{1}}

\renewcommand{\figurename}{Fig.}

%\renewcommand{\thesubfigure}{\figurename\ \thefigure.\alph{subfigure}}
%\makeatletter
%\renewcommand{\@thesubfigure}{\thesubfigure:\space}
%\renewcommand{\p@subfigure}{}
%\makeatother

\renewcommand{\baselinestretch}{0.940}
%%%%%%%%%%%%%%%%%From RANDY
\newcommand{\sens}{z}
\newcommand{\comnbrs}{{\mathscr{S}}}
%\newcommand{\physnbrs}{{\cal X}^{\text{phys}}}
\newcommand{\sensnbrs}{{\mathscr{X}}}
\newcommand{\metastate}{{\mathscr{Y}}}
\newcommand{\R}{{\ensuremath{{\mathbb{R}}}}}
\providecommand{\norm}[1]{\lVert#1\rVert}
\newcommand{\environ}{{\mathscr{E}}}

%\usepackage[pdftex]{graphicx}
%\vfuzz2pt % Don't report over-full v-boxes if over-edge is small
%\hfuzz2pt % Don't report over-full h-boxes if over-edge is small
%
\DeclareGraphicsExtensions{.jpg,.pdf,.mps,.png,.eps} %For pdftex

\newtheorem{definition}{Definition}[section]
\newtheorem{property}{Property}[section]
\newtheorem{corollary}{Corollary}[section]
\newtheorem{propose}{Proposition}[section]
\newtheorem{theorem}{Theorem}[section]
\newtheorem{lemma}{Lemma}[section]
\newtheorem{example}{Example}[section]

\DeclareMathOperator{\Cov}{Cov}
\providecommand{\abs}[1]{\lvert#1\rvert}
\providecommand{\norm}[1]{\lVert#1\rVert}
\newcommand{\E}{\ensuremath{\mathcal E}}
\newcommand{\F}{\ensuremath{\mathcal F}}
%%%%%%%%%%END FROM RANDY%%%%%%%%%%%%%

\usepackage{fancyhdr}
\setlength{\headheight}{12pt} \pagestyle{fancy}

%%%\fancyhf{} \lhead{\small Project Description} \rhead{\footnotesize
%%%CPS:Large:Collaborative: {\bf CASAP} -- Context-Aware Systems for
%%%Agricultural Practices} \cfoot{\thepage}

%%%\small
\fancyhf{}
%\lhead{Context-Aware Management of Heterogeneous Sensor Networks}
%%%\lhead{\small Project Description}

%\rhead{\footnotesize CPS:Frontiers:Collaborative Research:Dynamic Entanglement of Control and Data Management in Precision Agriculture} \cfoot{\thepage}

\rhead{\footnotesize CHS: Medium: Collaborative: Dynamic Data-Driven Networks of Cyber-Human Internet of Things for Smart and Connected Communities} \cfoot{\thepage}

\normalsize

\newcounter{taskcount}

\setcounter{taskcount}{0}

\newenvironment{task}
{% This is the begin code
\refstepcounter{taskcount} \textbf{\underbar{Research Tasks Group
(RTG)-\arabic{taskcount}:}} \itshape }
{% This is the end code
}

\renewcommand{\thetaskcount}{\textbf{RTG-\arabic{taskcount}}}

\newenvironment{plan}
{% This is the begin code
\textbf{\underbar{Proposed Approaches:}}} {}



%% references 
\newcommand{\refsec}[1]{Section~\ref{#1}}
\newcommand{\refeq}[1]{Eq.~\ref{#1}}
\newcommand{\reffig}[1]{Fig.~\ref{#1}}
\newcommand{\reftab}[1]{Table~\ref{#1}}
\newcommand{\refalg}[1]{Alg.~\ref{#1}}
\newcommand{\refline}[1]{Line~\ref{#1}}
\newcommand{\reflines}[2]{Lines~\ref{#1}-\ref{#2}}
\newcommand{\refprop}[1]{Proposition~\ref{#1}}
\newcommand{\refsupp}[1]{~\cref{S#1}}
\newcommand{\refsupps}[2]{~\cref{S#1,S#2}}

\newcommand*\BitAnd{\mathrel{\&}}
\newcommand*\BitOr{\mathrel{|}}
\newcommand*\ShiftLeft{\ll}
\newcommand*\ShiftRight{\gg}
\newcommand*\BitNeg{\ensuremath{\mathord{\sim}}}

\newcommand*\powerSetSymbol{\ensuremath{\mathcal{P}}\xspace}
\newcommand*\powerSet[1]{\ensuremath{\mathcal{P}\left(#1\right)}\xspace}
\newcommand*\numParameters{\ensuremath{n}\xspace}

\newcommand{\suchthat}{\, \mid \,} % nice "such that"


\newcommand{\camerareadycomment}[1]{}

\newcommand{\mysymbola}{\textcolor{mycolora}{$\mathbf{\ominus}$}\xspace}
\newcommand{\mysymbolb}{\textcolor{mycolorb}{$\mathbf{\oplus}$}\xspace}

\newcommand{\theoretical}[1]{\SetCommentSty{mycommfonta} \textcolor{mycolora}{#1} \tcp*{\mysymbola}}
\newcommand{\practical}[1]{\SetCommentSty{mycommfontb} \textcolor{mycolorb}{#1} \tcp*{\mysymbolb}}

\newcommand{\eventsig}[1]{\textit{#1}}

\newcommand{\CANVAS}{\texttt{CANVAS}\xspace}

\newcommand{\world}{\ensuremath{\mathcal{W}}\xspace}
\newcommand{\smartobj}[1]{\ensuremath{w_{#1}}\xspace}

%% TODO remove 
\newcommand{\smartobjs}{\ensuremath{\mathbf{S}}}

%% TODO consoldiate smartobject group and group state 
\newcommand{\smartObjectGroup}[1]{\ensuremath{\mathbf{w}_{#1}}\xspace}

\newcommand{\affordances}{\ensuremath{\mathbf{F}}}
\newcommand{\affordance}{\ensuremath{f}}
\newcommand{\affordanceowner}{\ensuremath{w_o}}
\newcommand{\affordanceuser}{\ensuremath{w_u}}

\newcommand{\state}[1]{\ensuremath{s_{#1}}}
\newcommand{\worldstate}{\ensuremath{\mathbf{\state{}}}}
\newcommand{\groupState}[1]{\ensuremath{\mathbf{\state{}}_{#1}}}

\newcommand{\nextState}[1]{\ensuremath{s_{#1}^{'}}}
\newcommand{\nextGroupState}[1]{\ensuremath{\mathbf{\state{}}_{#1}{'}}}
\newcommand{\currentGroupState}[2]{\ensuremath{\mathbf{\state{}}_{#1}^{#2}}}

%% TODO 
\newcommand{\states}{\ensuremath{S}} %% ?? 
\newcommand{\worldstates}{\ensuremath{\boldsymbol{\Psi}}}

\newcommand{\attributeSymbol}{\ensuremath{\theta}\xspace}
\newcommand{\attributemapping}[2]{\ensuremath{\attributeSymbol(#1,#2)}}
\newcommand{\numAttributes}{\ensuremath{N}}
%\newcommand{\attributemappings}{\ensuremath{\Theta}}
\newcommand{\attributeLabel}[1]{\texttt{#1}\xspace}

\newcommand{\Incapacitated}[1]{\attributeLabel{Incapacitated}(#1)\xspace}
\newcommand{\HasWeapon}[1]{\attributeLabel{HasWeapon}(#1)\xspace}
\newcommand{\HasKeys}[1]{\attributeLabel{HasKeys}(#1)\xspace}
\newcommand{\HasBriefcase}[1]{\attributeLabel{HasMoney}(#1)\xspace}
\newcommand{\Unlocked}[1]{\attributeLabel{Unlocked}(#1)\xspace}

\newcommand{\MButtonPressed}[1]{\attributeLabel{BMPressed}(#1)\xspace}
\newcommand{\TButtonPressed}[1]{\attributeLabel{BTPressed}(#1)\xspace}
\newcommand{\ButtonPressed}[2]{\attributeLabel{BTPressed}(#1,#2)\xspace}

\newcommand{\IsOpen}[1]{\attributeLabel{Open}(#1)\xspace}
\newcommand{\LeftHandEmpty}[1]{\attributeLabel{LHandEmpty}(#1)\xspace}
\newcommand{\RightHandEmpty}[1]{\attributeLabel{RHandEmpty}(#1)\xspace}
\newcommand{\Occupied}[1]{\attributeLabel{Occupied}(#1)\xspace}
\newcommand{\IsGuardable}[1]{\ruleLabel{IsGuardable}(#1)\xspace}
\newcommand{\IsUnguarded}[1]{\ruleLabel{IsUnguarded}(#1)\xspace}


\newcommand{\relationshipSymbol}{\ensuremath{R}}
\newcommand{\relationship}[3]{\ensuremath{\relationshipSymbol_{#3}(#1,#2)}}
\newcommand{\relationshipLabel}[1]{\texttt{#1}\xspace}

\newcommand{\IsAlliedWith}[2]{\relationshipLabel{AlliedWith}(#1,#2)\xspace}
\newcommand{\IsGuarding}[2]{\relationshipLabel{IsGuarding}(#1,#2)\xspace}
\newcommand{\IsInZone}[2]{\relationshipLabel{InZone}(#1,#2)\xspace}
\newcommand{\FrontZone}[2]{\relationshipLabel{FrontZone}(#1,#2)\xspace}


\newcommand{\ruleLabel}[1]{\texttt{#1}\xspace}
\newcommand{\CanAccessObject}[2]{\ruleLabel{CanAccess}(#1,#2)\xspace}
\newcommand{\CanAccessZone}[2]{\ruleLabel{CanAccessZone}(#1,#2)\xspace}
\newcommand{\CanManipulateObject}[2]{\ruleLabel{CanManipulate}(#1,#2)\xspace}
\newcommand{\PathExistsForObject}[3]{\ruleLabel{PathExistsForObject}(#1,#2,#3)\xspace}


\newcommand{\true}{\textnormal{\texttt{TRUE}}\xspace}
\newcommand{\false}{\textnormal{\texttt{FALSE}}\xspace}
\newcommand{\unknown}{\textnormal{\texttt{UNKNOWN}}\xspace}

\newcommand{\event}{\ensuremath{e}\xspace}

\newcommand{\roleSet}[1]{\ensuremath{\mathbf{r}_{#1}}}
\newcommand{\role}[1]{\ensuremath{r_{#1}}}
\newcommand{\roleFunction}[2]{\ensuremath{r_{#1}(#2)}}
\newcommand{\validBin}[1]{\ensuremath{l_{{\role{}}^{#1}}}}

%\newcommand{\preconditionSet}[1]{\ensuremath{\mathbf{\Phi}_{#1}}}

\newcommand{\precondition}[1]{\ensuremath{{\Phi}_{#1}}}
\newcommand{\preconditionFunction}[2]{\ensuremath{{\Phi}_{#1}(#2)}}

\newcommand{\clause}[2]{\ensuremath{\phi_{#1}^{#2}}}


\newcommand{\postcondition}[1]{\ensuremath{{\Omega}_{#1}}}
\newcommand{\postconditionFunction}[2]{\ensuremath{{\Omega}_{#1}(#2)}}

%% this is just basically using smart object group -- TODO: remove 
\newcommand{\participants}[1]{\ensuremath{P}}


%\newcommand{\roles}{\ensuremath{\mathbf{R}}}
%\newcommand{\rolespec}{\ensuremath{\mathbf{C}}}

%\newcommand{\smartactor}{\ensuremath{a}}

\newcommand{\tree}{\ensuremath{t}\xspace}
\newcommand{\routine}{\ensuremath{\rho}}


\newcommand{\lexicon}{\ensuremath{\mathcal{E}}\xspace}
\newcommand{\crowdLexicon}{\ensuremath{\lexicon_{g}}}

%%% TODO THIS SHOULD BE REMOVED 
%\newcommand{\events}{\ensuremath{\mathcal{L}}\xspace}


%\newcommand{\role}{\ensuremath{r}}


\newcommand{\crowd}{\ensuremath{\smartobj{g}}}
\newcommand{\members}{\ensuremath{\mathbf{M}}}



\newcommand{\decision}{\ensuremath{d}}
\newcommand{\priority}{\ensuremath{p}}
\newcommand{\instance}{\ensuremath{I}\xspace}

\newcommand{\newInstance}{\ensuremath{I_{\scriptscriptstyle \mathrm{new}}}\xspace}
\newcommand{\startInstance}{\ensuremath{I_{\scriptscriptstyle \mathrm{0}}}\xspace}



\newcommand{\instances}{\ensuremath{\mathbb{I}}}
\newcommand{\beat}{\ensuremath{\mathbf{B}}\xspace}

\newcommand{\arc}{\ensuremath{\alpha}\xspace}


\newcommand{\storysequence}{\ensuremath{Q}}
\newcommand{\vertices}{\ensuremath{V}}
\newcommand{\smartobjectvertices}{\ensuremath{V_S}}
\newcommand{\eventvertices}{\ensuremath{V_I}}

\newcommand{\participationedge}{\ensuremath{\pi}}
\newcommand{\participationedges}{\ensuremath{E_\pi}}
\newcommand{\sequenceedge}{\ensuremath{\sigma}}
\newcommand{\sequenceedges}{\ensuremath{E_\sigma}}
\newcommand{\terminationedges}{\ensuremath{E_\varphi}}

\newcommand{\edges}{\ensuremath{E}}
\newcommand{\smartobjectedges}{\ensuremath{E_S}}
\newcommand{\eventedges}{\ensuremath{E_I}}
\newcommand{\edge}{\ensuremath{e}}

%%%%%%%%%%%%%%%%%%%%%%%%%%%%%%%%%% POP SYMBOLS JANUARY 2015 %%%%%%%%%%%%%%%%%%%%%

\newcommand{\partialArc}{\ensuremath{\alpha_{p}}\xspace}
\newcommand{\completeArc}{\ensuremath{\alpha_{c}}\xspace}

\newcommand{\pop}{\ensuremath{\pi}\xspace}
\newcommand{\partialPop}{\ensuremath{\pi_p}\xspace}
\newcommand{\completePop}{\ensuremath{\pi_c}\xspace}

\newcommand{\lessThan}{\ensuremath{\prec}\xspace}
\newcommand{\st}{\ensuremath{\; \textnormal{s.t.} \;}\xspace}

\newcommand{\eventInstanceSet}{\ensuremath{\mathcal{I}}\xspace}
%\newcommand{\endInstanceSet}{\ensuremath{\mathcal{I}}\xspace}

\newcommand{\instanceSetEnd}{\ensuremath{\mathcal{I}_{\scriptscriptstyle \textnormal{end}}}\xspace}


\newcommand{\orderingConstraint}{\ensuremath{\mathbf{o}}\xspace}
\newcommand{\orderingConstraintSet}{\ensuremath{\mathcal{O}}\xspace}
\newcommand{\bindingConstraintSet}{\ensuremath{\mathcal{B}}\xspace}

\newcommand{\causalLink}{\ensuremath{\mathbf{l}}\xspace}
\newcommand{\causalLinkSet}{\ensuremath{\mathcal{L}}\xspace}

\newcommand{\agenda}{\ensuremath{\mathcal{A}}\xspace}
\newcommand{\agendaItem}{\ensuremath{a}\xspace}
%\newcommand{\agenda}{\ensuremath{\mathcal{A}}\xspace}

%%% algorithm 

\newcommand{\resolveStoryArcFunction}{\textbf{Resolve}\xspace}
\newcommand{\plannerFunction}{\textbf{Plan}\xspace}
\newcommand{\keyFunction}{\textbf{key}\xspace}
\newcommand{\linearizeFunction}{\textbf{Linearize}\xspace}
\newcommand{\consistentFunction}{\textnormal{\textbf{Consistent}}\xspace}


%Planning
\newcommand{\mytilde}{\raise.17ex\hbox{$\scriptstyle\mathtt{\sim}$}}

%% TODO this should be removed
%% Why?
\newcommand{\attribute}{\ensuremath{a}}


%% TODO remove 
\newcommand{\id}{\ensuremath{i}}

\newcommand{\constraint}[1]{\ensuremath{C_{#1}}\xspace}

\newcommand{\sequence}{\ensuremath{\boldsymbol{\Pi}}}
\newcommand{\transitions}{\ensuremath{\boldsymbol{\Gamma}}}
\newcommand{\immutables}{\ensuremath{I}}
\newcommand{\cost}{\ensuremath{c}}
\newcommand{\Rule}{\ensuremath{\mathcal{R}}}
\newcommand{\Rules}{\ensuremath{\mathbf{R}}}

%Functions
\newcommand{\functionStyle}[1]{\textbf{{#1}}}

\newcommand{\relaxedFillIn}[2]{\ensuremath{\functionStyle{RelaxedFillIn} (#1, #2)}\xspace}
\newcommand{\computeTransitionList}[2]{\functionStyle{ComputeTransitionList ({#1}, {#2})}\xspace}
\newcommand{\detectInconsistencies}[2]{\functionStyle{DetectInconsistencies ({#1}, {#2})}\xspace}
\newcommand{\buildActionSpace}[2]{\functionStyle{BuildActionSpace ({#1}, {#2})}\xspace}
\newcommand{\findConsistentArc}[3]{\functionStyle{FindConsistentArc ({#1}, {#2}, {#3})}\xspace}
\newcommand{\simulate}[2]{\ensuremath{\functionStyle{Simulate}(#1,#2)}\xspace}

\newcommand{\planFunction}[1]{\ensuremath{\functionStyle{Plan}(#1)}\xspace}

\newcommand{\generateParametersFunction}[1]{\ensuremath{\functionStyle{GenerateBindings}(#1)}\xspace}

\newcommand{\curTry}{\ensuremath{j}\xspace}
\newcommand{\curTryF}{\ensuremath{k}\xspace}
\newcommand{\cutoff}{\ensuremath{\varepsilon}\xspace}
\newcommand{\nrPopulations}{\ensuremath{\lambda}\xspace}


\newcommand{\nrOnes}[1]{\ensuremath{||{#1}||_b}}

\newcommand{\heuristicFunction}[2]{\ensuremath{h(#1,#2)}\xspace}


\newcommand{\problemdomain}{\ensuremath{\Sigma}}
\newcommand{\probleminstance}{\ensuremath{\mathbf{P}}}
\newcommand{\bitmask}[1]{\ensuremath{\beta(#1)}}
\newcommand{\paths}{\ensuremath{\mathcal{P}}}

\newcommand{\planSymbol}{\ensuremath{\boldsymbol{\Pi}}\xspace}
%\newcommand{\plan}[2]{\ensuremath{\boldsymbol{\Pi}(#1,#2)}\xspace}


%\newcommand{\argmin}{\operatornamewithlimits{argmin}}
%\newcommand{\argmax}{\operatornamewithlimits{argmax}}


\newcommand{\spaciousItemize}{\setlist{topsep=0pt, parsep=5pt, partopsep=0pt, leftmargin=10pt}}
\newcommand{\spaciousEnumerate}{\setlist{topsep=0pt, parsep=5pt, partopsep=0pt, leftmargin=15pt}}
\newcommand{\tightItemize}{\setlist{topsep=0pt, parsep=2pt, partopsep=0pt, leftmargin=10pt}}
\newcommand{\tightEnumerate}{\setlist{topsep=0pt, parsep=2pt, partopsep=0pt, leftmargin=15pt}}

\newcommand{\symStateMember}[1]{\texttt{#1}}
\newcommand{\symNot}[1]{\ensuremath{\neg}#1\xspace}

\newcommand{\matchRules}[0]{\ensuremath{B}\xspace}
\newcommand{\matchRule}[0]{\ensuremath{\beta}\xspace}
\newcommand{\rejectRules}[0]{\ensuremath{\Theta}\xspace}
\newcommand{\rejectRule}[0]{\ensuremath{\theta}\xspace}
\newcommand{\populations}[0]{\ensuremath{\mathbf{PO}}\xspace}
\newcommand{\population}[0]{\ensuremath{\alpha}\xspace}
\newcommand{\match}[0]{\ensuremath{\mathbf{MA}}\xspace}

%Events
\newcommand{\eventStyle}[1]{#1}
\newcommand{\auto}[1]{\textbf{#1}}
\newcommand{\autoP}[1]{\ensuremath{\mathbf{#1}}}


\newcommand{\CoerceIntoUnlockDoor}[3]{\eventStyle{CoerceIntoUnlockDoor}({#1}, {#2}, {#3})\xspace}

\newcommand{\Incapacitate}[2]{\eventStyle{Incapacitate}(#1, #2)\xspace}
\newcommand{\IncapacitateStealthly}[2]{\eventStyle{IncapacitateStealthily}(#1, #2)\xspace}

\newcommand{\OpenVault}[1]{\eventStyle{OpenVault}({#1})\xspace}
\newcommand{\OpenDoor}[2]{\eventStyle{OpenDoor}({#1}, {#2})\xspace}
\newcommand{\PressButton}[2]{\eventStyle{PressButton}({#1}, {#2})\xspace}

\newcommand{\TakeWeaponFromIncapacitated}[2]{\eventStyle{TakeWeaponFromIncapacitated}({#1}, {#2})\xspace}

\newcommand{\TakeFromIncapacitated}[3]{\eventStyle{TakeFromIncapacitated}({#1}, {#2}, {#3})\xspace}

\newcommand{\EnterRoom}[2]{\eventStyle{EnterRoom}({#1}, {#2})\xspace}

\newcommand{\PickUp}[2]{\eventStyle{PickUp}({#1}, {#2})\xspace}



\newcommand{\PickUpBriefcase}[2]{\eventStyle{PickUpBriefcase}({#1}, {#2})\xspace}
\newcommand{\PickUpWeapon}[2]{\eventStyle{PickUpWeapon}({#1}, {#2})\xspace}

\newcommand{\PressTellerButton}[1]{\eventStyle{PressTellerButton}({#1})\xspace}
\newcommand{\PressManagerButton}[1]{\eventStyle{PressManagerButton}({#1})\xspace}
\newcommand{\DistractAndIncapacitate}[3]{\eventStyle{DistractAndIncapacitate}({#1}, {#2}, {#3})\xspace}
\newcommand{\TakeKeyFromIncapacitated}[2]{\eventStyle{TakeKeyFromIncapacitated}({#1}, {#2})\xspace}

\newcommand{\GiveWeapon}[2]{\eventStyle{GiveWeapon}({#1}, {#2})\xspace}
\newcommand{\GiveKey}[2]{\eventStyle{GiveKey}({#1}, {#2})\xspace}

\newcommand{\WarningShot}[2]{\eventStyle{WarningShot}({#1}, {#2})\xspace}
\newcommand{\UnlockDoor}[2]{\eventStyle{UnlockDoor}({#1}, {#2})\xspace}
\newcommand{\LockDoor}[2]{\eventStyle{LockDoor}({#1}, {#2})\xspace}
\newcommand{\Flee}[1]{\eventStyle{Flee}({#1})\xspace}
\newcommand{\BreakAlliance}[2]{\eventStyle{BreakAlliance}({#1}, {#2})\xspace}
\newcommand{\Escape}[1]{\eventStyle{Escape}({#1})\xspace}
\newcommand{\CoerceTellerAtCounterIntoPressingButton}[2]{\eventStyle{CoerceTellerAtCounterIntoPressingButton}({#1}, {#2})\xspace}

\newcommand{\CoerceIntoPressingButton}[3]{\eventStyle{CoerceIntoPressButton}({#1}, {#2}, {#3})\xspace}

\newcommand{\CoerceIntoPressingTellerButton}[2]{\eventStyle{CoerceIntoPressingTellerButton}({#1}, {#2})\xspace}

\newcommand{\TakeMoney}[1]{\eventStyle{TakeMoney}({#1})\xspace}


\newcommand{\Converse}[2]{\eventStyle{Converse}({#1}, {#2})\xspace}
\newcommand{\Argue}[2]{\eventStyle{Argue}({#1}, {#2})\xspace}

\newcommand{\CoerceIntoSurrender}[2]{\eventStyle{CoerceIntoSurrender}({#1}, {#2})\xspace}
\newcommand{\CoerceIntoDropWeapon}[2]{\eventStyle{CoerceIntoDropWeapon}({#1}, {#2})\xspace}
\newcommand{\CoerceIntoGiveKey}[2]{\eventStyle{CoerceIntoGiveKey}({#1}, {#2})\xspace}

\newcommand{\CoerceIntoMove}[3]{\eventStyle{CoerceIntoMove}({#1}, {#2}, {#3})\xspace}



%Results
\newcommand{\robberA}{\ensuremath{r_{\textnormal{\tiny 1}}}\xspace}
\newcommand{\robberB}{\ensuremath{r_{\textnormal{\tiny 2}}}\xspace}
\newcommand{\robberC}{\ensuremath{r_{\textnormal{\tiny 3}}}\xspace}

\newcommand{\guardA}{\ensuremath{g_{\textnormal{\tiny 1}}}\xspace}
\newcommand{\guardB}{\ensuremath{g_{\textnormal{\tiny 2}}}\xspace}
\newcommand{\guardC}{\ensuremath{g_{\textnormal{\tiny 3}}}\xspace}
\newcommand{\guardD}{\ensuremath{g_{\textnormal{\tiny 4}}}\xspace}

\newcommand{\tellerA}{\ensuremath{t_{\textnormal{\tiny 1}}}\xspace}
\newcommand{\tellerB}{\ensuremath{t_{\textnormal{\tiny 2}}}\xspace}
\newcommand{\manager}{\ensuremath{m}\xspace}

\newcommand{\managerDoor}{\ensuremath{d_{\textnormal{\tiny m}}}\xspace}
\newcommand{\tellerDoor}{\ensuremath{d_{\textnormal{\tiny t}}}\xspace}
\newcommand{\vaultDoor}{\ensuremath{d_{\textnormal{\tiny v}}}\xspace}

\newcommand{\managerButton}{\ensuremath{b_{\textnormal{\tiny m}}}\xspace}
\newcommand{\tellerButton}{\ensuremath{b_{\textnormal{\tiny t}}}\xspace}

\newcommand{\managerRoom}{\ensuremath{f_{\textnormal{\tiny m}}}\xspace}
\newcommand{\tellerRoom}{\ensuremath{f_{\textnormal{\tiny t}}}\xspace}
\newcommand{\lobbyRoom}{\ensuremath{f_{\textnormal{\tiny l}}}\xspace}

\newcommand{\weapon}{\ensuremath{w}\xspace}

\newcommand{\customerCrowd}{\ensuremath{cr}\xspace}
\newcommand{\customer}{\ensuremath{c}\xspace}
\newcommand{\customerA}{\ensuremath{c_{\textnormal{\tiny 1}}}\xspace}
\newcommand{\customerB}{\ensuremath{c_{\textnormal{\tiny 2}}}\xspace}
\newcommand{\customerC}{\ensuremath{c_{\textnormal{\tiny 3}}}\xspace}
\newcommand{\customerD}{\ensuremath{c_{\textnormal{\tiny 4}}}\xspace}

%%%%%%%%%%%%%%%%%%%%%% USER STUDY %%%%%%%%%%%%%%%%%%%%%%%%%%%%%%%


\newcommand{\narrative}[1]{\ensuremath{\mathrm{N_{\textnormal{\tiny #1}}}}\xspace}
\newcommand{\category}[1]{\ensuremath{\mathrm{C_{\textnormal{\tiny #1}}}}\xspace}
\newcommand{\authoringTool}[1]{\ensuremath{\mathrm{T_{\textnormal{\tiny #1}}}}\xspace}
\newcommand{\authoringBlock}[1]{\ensuremath{\mathrm{B_{\textnormal{\tiny #1}}}}\xspace}

\newcommand{\narrativeA}{\narrative{A}}
\newcommand{\narrativeB}{\narrative{B}}

\newcommand{\structuredCategory}{\category{S}}
\newcommand{\unstructuredCategory}{\category{U}}

\newcommand{\authoringToolA}{\authoringTool{A}}
\newcommand{\authoringToolB}{\authoringTool{B}}
\newcommand{\authoringToolC}{\authoringTool{C}}


%\newcommand{\narrative}[2]{\ensuremath{\mathrm{N_{\textnormal{\tiny #1}}^{\textnormal{\tiny #2}}}}\xspace}

%\newcommand{\structuredNarrative}{\narrative{S}{}}
%\newcommand{\unstructuredNarrative}{\narrative{U}{}}

%\newcommand{\structuredNarrativeA}{\narrative{S}{1}}
%\newcommand{\structuredNarrativeB}{\narrative{S}{2}}
%\newcommand{\unstructuredNarrativeA}{\narrative{U}{1}}
%\newcommand{\unstructuredNarrativeB}{\narrative{U}{2}}


\newcommand{\authoringTask}[1]{\ensuremath{\mathrm{XX_{#1}}}\xspace}

\newcommand{\authoringTime}{\ensuremath{t}\xspace}
\newcommand{\authoringEffort}{\ensuremath{e}\xspace}
\newcommand{\narrativeRatio}{\ensuremath{r}\xspace}
\newcommand{\storySimilarity}{\ensuremath{d}\xspace}
\newcommand{\meanStorySimilarity}{\ensuremath{\storySimilarity_{\scriptscriptstyle \mu}}\xspace}
\newcommand{\storySimilarityFunction}[2]{\ensuremath{\storySimilarity(#1,#2)}\xspace}
\newcommand{\meanStorySimilarityFunction}[1]{\ensuremath{\storySimilarity(#1)}\xspace}


\newcommand{\storyLength}{\ensuremath{l}\xspace}

\newcommand{\transitiveClosure}[1]{\ensuremath{\mathbf{T}(#1)}\xspace}



\sloppy

\begin{document}

%\flushbottom

%%%\pagestyle{empty}
\mypage{Project Description }

\section{Introduction}\label{intro}%%

According to IDC (www.idc.com) -- a global provider of market intelligence, advisory services, and events for the information technology, telecommunications and consumer technology markets -- the worldwide shipment of smart connected devices will surpass 2.5 billion units in 2017. The decreasing trends of cost, size, and power requirements of the various devices, were accompanied by an increase in embedded computing capabilities. In addition -- every device can be connected with other devices, each of which: (1) is capable of sensing/computing, communicating and actuating; (2) depends on values and state-descriptions of other devices, not necessarily within geo-spatial proximity nor with homogeneous physical characteristics, to steer the course of its own activities.  Such heterogeneous device range from computers, tablets and smartphones, through embedded systems in cars and traffic lights, precision agriculture sensors and actuators, to various home appliances, bodily sensors, etc.~\cite{Al-FuqahaGMAA15,AtzoriIM10,MelodiaPGA07,SalarianCN12}. According to recent a study by Goldman \& Sachs, over 12 billion devices are already connected to the Internet of Thing (IoT) and by 2020 that number should become 20 billion~\footnote{http://www.goldmansachs.com/our-thinking/pages/iot-meets-everything.html}.

%%People and data are already online. Soon, all sort of things with sensors, actuators and places will be also online. All sectors of activity will %%benefit from applications that leverage the interconnection of embedded systems with IT systems. Intelligent transport solutions can speed up %%traffic flows, reduce fuel consumption, and safe lives. Smart farming solutions will improve the production and delivery of safe and healthy food. %%Remote health monitoring will provide convenient access to healthcare, raises it quality, and save money. Utilities management, education, %%government services, retail sectors etc.… all will also benefit from such couplings and the things will leave across all different systems. %%\textbf{The next step is to interconnect these individual systems into a system of systems, and here we are confronted to such a scale and %%heterogeneity that we are reaching a limit in networking and software engineering}.\\

Recent works have already attempted to capitalize on the results from social networks research~\cite{AhmedEH15,Hai-Jew16,MorrisTP10,VenturiniJL15} -- namely~\cite{AtzoriCI14,IeraMA15} have specifically pointed how the experiences from social networks of users equipped with smart devices could be generalized to {\it social networks of devices} in the IoT realm. In a similar spirit, attempts have been made to demonstrate the benefits of {\it virtual objects} as a good abstraction for tackling IoT-related problems~\cite{NittiPCA16}. Studies regarding integration of heterogeneous devices in various contexts abound -- e.g., from smart homes in a manner that would balance the Quality of Experience (QoE) and Energy consumption~\cite{FlorisMPA15}; to better use of smart grid~\cite{LuoAM14} and enabling smart cities~\cite{ZanellaBCVZ14}.

One of the core motivations for the proposed project is based on the observation that current practices related to the design of IoT systems are still rather compartmentalized, and the dynamics-aware fusion of data collection, analytics processing and actuation across a variety of heterogeneous IoTs has limited support. As a simple example, consider a scenario where one has a (collection of) smart bed(s), such as, e.g., RestBed (https://www.restperformance.com/) with a surface equipped with multiple sensors to improve sleeping experience. A smart home may, in addition to a RestBed, also have smart refrigerators such as, e.g., Family Hub from Samsung (http://www.samsung.com/us/explore/family-hub-refrigerator/). While the QoE estimate can be obtained based on customers survey (or, even from bodily sensors), and it can even be balanced with the energy-efficiency (cf.~\cite{FlorisMPA15}) ``in-concert'' with such devices, there are some important functionalities that are lacking, and could yet significantly improve the overall experience and efficient use of the available devices/technology:

\begin{itemize}
\item The quality/satisfaction of sleeping is not correlated with the nutritional habits (i.e., the content of the food in the smart refrigerator and consumption patterns). This, in turn, prevents the respective manufacturers from improving their products/devices in a collaborative and context-aware manner.
    
\item There is no formal methodology of {\it where} the data assembling (as well as data aggregation and decision/recommendation as well as actuation) should take place. For instance, should all the nutrition-relevant data from an ensemble of smart refrigerators be sent at the local grocery store to optimize deliveries, or should it be aggregated at some geo-social level of hierarchy. Moreover, this raises the question of how the aspects of privacy and security should be handled.
\end{itemize}



%In this project, we aim to investigate issues related to the design and development of a secure physical and computational networked environment, %consisting of a large collection of connected smart objects (things). The proposed environment would allow interoperability and would also %facilitate efficient collection and processing of data from these {\it things} and development of analytics based on the collected data. Based on %these analytics, the {\it things} would be able to relate to each other to form dynamic networks (akin to social networks, in this case, of {\it %things}). These smart objects may come from diverse environments such as smart appliances at homes, smart mobile surveillance system such as fire %trucks and police cars, smart transportation entities (e.g., cabs or buses), etc. 



%Nowadays, existing start of the art related to the design of IoT systems is highly compartmentalized, the data collection and analytics processing %across a variety of heterogeneous IoTs has limited support, and the interoperability of objects across domains has not been addressed at a level %that would provide the kinds of optimizations that this project aims to achieve. In this project, we introduce the concept of collaborative Social %Network of Things that would enable data and analytics sharing across objects that exhibit similar operating characteristics and/or have similar %profiles in terms of usage and user/manufacturer parameters. In addition, we propose to develop a hierarchical middleware implemented as fogs to allow interoperability with context-aware security properties associated with different nodes. These middleware fogs, mainly implemented in smart boxes/gateways within local networks, will manage the communication between the associated smart objects and the network in an optimized way that takes into account both the characteristics of the smart objects and the preferences of the users. The massive generated data by all these connected smart objects and stored in the system will enable the cognitive network, through deep learning algorithms, to raise dynamic procedures that will enable a multi-criteria optimizations of the  behavior  of these smart objects.\\

 Extending the above scenario, one may consider an evolution along spatio-temporal dimension. For instance, during a lunch-break, coffee break at the Jack decides to lower by 5F the set temperature of his home heating system – to activate when he is less than 1 mile from his home and in any case no later than 6PM. However, if the outside temperature drops below 55F, the home temperature must go up by 5 degrees. A colleague gives Jack a list of instructions corresponding to a new method for washing wool sweaters; he  sends the file directly to his home through the social network where a service will adapt the instructions to his washing machine’s particular technical features and his personal preferences (e.g., extra soft cloths). % Back at home, Jack launches “my angry day” profile through  smartphone, which automatically dims the light and has the Hi-Fi play his favorite relaxing music. Meanwhile, 
  At home a tip from the network tells Jack to turn off all TVs to avoid getting further depressed. During the night the washing machine starts the cycle at the time the electric company informs it to be the most favorable.

The above describes a single user interacting seamlessly with multiple devices. We observe that one could attempt to generalize the existing approaches for Social Network of Things~\cite{IeraMA15,NittiPCA16} that can be generated/organized and dissolved on-the-fly, and enable scalability.
 %where different criteria can be used to orchestrate the behavior of (groups of) instances subject to multiple constraints; and certain security and privacy concerns are ensured. 
However, even in the context of ``Jack'' scenario, there are certain aspects that span accross heterogeneous domains/types and at different levels of certain (e.g., hierarchical) relationships. For instance, collaboratively regulating the temperature with other units in Jaack’s building has one kind of spatial, temporal, and devices extent, whereas collaboratively deciding about the cycle of the washing machine may go beyond the building -- i.e., optimizng grid-demand within a city-block.
%adjusting the Hi-Fi themes based on the fact that Jack decided to join his co-workers for a happy-hour; etc… 

By establishing a dynamic instantiations of various components/devices comprising Internet of Things (IoT) structures, we aim to  provide an automatic resource optimization and other intelligence into these evolving collections of smart connected objects, going beyond the social networks of things. Rather than endowing them with autonomously cognitive chips, we aim to enable manufacturers, retailers and users/owners, to elaborate finely tuned dispatches to be distributed toward the smart objects through {\it middleware fogs} located various units (e.g., homes).
As part of the proposed research we plan to provide novel data analytics methodologies that will: (a) incorporate the fact that some users may not be cooperative in terms of suggested regime of use for certain devices/task; (b) provide a specific behavioral insights to be used by various manufacturers when planning the design of certain devices so that they can be better adapted to changing conditions of use.
 A high-level description of the architecture we target to develop for supporting such  scenarios is depicted in Figure~\ref{fig1}, providing methodologies and tools for two basic modes: \\

\begin{wrapfigure}{R}{0.60\textwidth} \vspace{-3mm}
	\centerline{\includegraphics[width=0.60\textwidth]{./fig1.png}}
	\vspace{-3mm} \caption{\small Depiction of the proposed system}
	\label{fig1}
	\vspace{-3mm}
\end{wrapfigure}

(I)  The first setup consists of a set of middleware fogs, located in each smart box/gateway of a smart home or public area, grouped together in an evolving (social-like) network along with profiles of manufacturers, retailers and users/owners. The lower layer is formed by all the real devices/objects (things), such as a fridge, a washing machine, a heater, a TV, a Hi-Fi, etc., where each one is abstracted by what we refer to as WiThing (WiT) for Wireless Thing. The WiT is the first level of the device abstraction and its role is to (1) uniquely identify an object/device (2) represent the object in terms of its properties/functionalities through an MIB (Management Information Base), and (3) constitute a bidirectional interface for all communications between the object and the middleware fog located in the smart box/gateway. The intermediate layer contains the smart box/gateway where the middleware fog is implemented. The latter is composed of an ensemble of interconnected modules that will govern all the requests issued by the user through a proper front-end application. To this end, this smart box must interface with any object/device it meets in the surroundings, i.e., any virtual device communicated by the real objects/devices. The upper layer illustrates the complexity of the different couplings to form ensembles of (networks of) manufacturers, retailers and users/owners with the generic representations of their objects/devices. 
%%%It consists of a large database with an enquiry system based on sophisticated algorithms.



\begin{wrapfigure}{R}{0.40\textwidth} \vspace{-3mm}
	\centerline{\includegraphics[width=0.38\textwidth]{./fig2-3.jpg}}
	\vspace{-3mm} \caption{\small Representation of Smart Objects as Avatars and Meta-avatars in the Cloud.}
	\label{fig2}
	\vspace{-3mm}
\end{wrapfigure}

(II) The second setup complements the previous one in terms of inter-contexts couplings and dynamic formations (and dissolution) of corresponding instances of ensembles with heterogeneous participants. In terms of notation in Figure 1. this can be explained by the following consideration:
(a) If we have an instance of a ``network'' of particular devices (e.g., a washers) from a collection of residential units, depending on the geographical context we may want to compose instances of collections corresponding a group of same devices but operating in a group of buildings in a near-downtown area.
(b) Complementary to this, we may need to couple the ``entity'' corresponding to the ensemble of washers in different units in an apartment complex, with the one of entertainment devices in the same and/or neighboring buildings.


One could readily generalize this in multiple contexts -- e.g., the instances representing the traffic-density in different geographical regions may be tied with the crime-rate in those and/or near-by regions and with the ones corresponding to activities of electronic devices in the police cars. %%However, after a certain time of the day/night, such (meta)Avatars may cease to exist.



To enable such situational awareness that is flexible enough to incorporate the dynamic formation of instances that can span not only from networks of users to network of heterogeneous devices, but also through the existence of ensembles comprised of different contextual impacts (space, time, objective to be optimized, etc.) we  propose the concept of Heterogeneity And Context Aware Dynamic Avatars (HACADA) illustrated in Figure~\ref{fig2}. We note that, in addition to the scenario-oriented discussions above, the HACADA's may be created, updated/modified and ceased based on different criteria: (a) particular cyber-properties: e.g., status of network infrastructure, privacy constraints, etc.; (b) Logical or semantical properties which may entail dynamically forming (and dissolving) hierarchies of HACADAs; (c) ``expiration'' of certain criteria (e.g., pat 10PM) enabling the leftover descriptors in two different HACADA-instances to be merged into one.


\noindent {\bf Intellectual Merits:}

Throughout this complex environment, and in order to automate the deployment and control, as well as optimize the use of such ensembles of HACADAs comprising multiple instances of operational Internet of Things,
%%gathering of important data, identification of correlations in this chain of complex events, and triggering the required responses,
a plethora of issues in data collection and processing (storage/retrieval, querying and information fusion), knowledge discovery, and security and privacy need to be investigated in unified framework. We assert that  such integrated large scale effort, involving researchers from diverse backgrounds, is necessary to  enable exploiting the full potential of the Internet of Things technology  and make it ubiquitous of use. The proposed project has the following list of collaborative research activities as its main Intellectual Merits:

{\it Design and development of novel techniques and abstractions for acquisition and representation of data from IoT devices and users} {\bf NOTE: mention HACADA + Mubbasir's discussion}

{\it Development of privacy-preserving analytics to discover hidden correlations and usage patterns and behaviors} {\bf NOTE: mention a summary from the section written by Vladimir and Ashfaq}.

{\it Design and development of network protocols to mitigate the DoS and greedy behavior in IoT devices} {\bf NOTE: Alek and Farid section-summary} 


{\it Development of simulation platform as well as real-world testbed of heterogeneous IoTs (devices and users)}. {\bf NOTE: mention a realization of application scencario(s)}.


\noindent {\bf Broader Impacts}: The salient broader impacts of the proposed project are:

{\bf NOTE: need to expand}

This research effort aims to contribute to the rapid development of national and local capacity to respond to critical events and to develop technology to advance the ability and scope of Internet of Things to detect and monitor events in applications of high priority, such as surveillance and habitat monitoring. The proposed research activity is a multidisciplinary effort involving researchers with expertise in distributed databases, security, and data management and analytics. This work will also contribute to, and complement efforts under way for, the design and management of reliable smart facilities. Leveraging on the existing cooperation between Illinois Institute of Technology, Northwestern University and Rutgers University. The PIs have a long record of working on collaborative research projects and have extensively involved undergraduate students, minority students and women in their research.



The proposed collaborative project will be executed by a team of six highly accomplished researchers and their students from three different institutions, who have expertise in all the complimentary areas relevant to the research activities proposed in this project, including: Human Computer Interface, Machine Learning, Information Systems and Data Sciences, Sensor and Sensor Networking, Security, and Privacy. The PI and Co-PIs have a long and well established record of collaborating on different federal government and industry funded  research projects. In the following we provide a brief introduction of the PIs and their relevant expertise.
\begin{itemize}
\vspace{-3mm} \item Ashfaq Khokhar, Illinois Institute of Technology: Wireless Sensor Networks, Data Mining, Cloud Computing
\vspace{-3mm} \item Farid Nait-Abdesselam, Illinois Institute of Technology: Network Security, Wireless Sensor Networks
\vspace{-3mm} \item Mubbasir Kapadia, Rutgers University: Human Computer Interface, Virtual Human Systems
\vspace{-3mm} \item Vladimir Pavlovic, Rutgers University: Machine Learning, Dynamic System Modeling, Computer Vision
\vspace{-3mm} \item Aleksander Kuzmanovic, Northwestern University: Networks Security, Denial-of-service Resiliency, Content Distribution
\vspace{-3mm} \item Goce Trajcevski, Northwestern University: Moving Objects Databases,Data Management in Sensor Networks
\end{itemize}




\section{State of the art}\label{soa}
We now present an overview of the state of the art, both from the perspective of the current properties of devices and networking technology features, as well as from the perspective of related works in several fields that, in one way or another, may be used to achieve our objectives however, the fall short of several important aspects.\\

\subsection{Whatever Proper title of this sub-section}
The vast majority of devices are endowed with electronic interfaces, made of buttons and the like, inviting users to operate selections from among a limited set of options pre-featured by the manufacturers. They may concern options, such as, ventilation on/off button in an oven, a selection of delicate option in a washing machine, air-condition on/off button in a car, and so on. While these older devices are considered somehow efficient and simple, the trend is towards more technological and ubiquitous devices.
The operational innovation of our approach is to make a move of the interface a step ahead and a step farther so as to realize a deep, remote and smart control. In this case, the manufacturer makes all/most of the operational parameters, such as spin speed and duration, totally manageable and configurable by the user. In turn, the user drives these parameters via software, hence makes them obeying to automatic procedures/instructions that have been elaborated and optimized in advance, with the help of the manufacturer, other contributors, and/or the networked intelligence.
Thus, we need to have this new hardware/software interface managing signals/data back and forth between actuators/sensors and a logic unit with the result of detaching the user from any physical contact with the device (thing). To this end, in the project we will use open source electronic boards, such as Arduino boards (https://www.arduino.cc/), Parallella boards (https://www.parallella.org/) and/or Raspberry-Pi boards (https://www.raspberrypi.org/). They are powerful prototyping boards based on flexible, open and easy-to-use hardware and software. These devices are wirily and/or wirelessly connected to the Internet from which they can be easily visible and accessible. We foresee that their exploitation will lead microelectronic companies to develop specific chips for a massive diffusion of this new technology in a near future.
In order to realize the vision of an Ambient Intelligence in a future network and service environment, heterogeneous wireless sensor and actuator networks have to be integrated into a common framework of global scale and made available to services and applications via universal service interfaces \cite{SSR05}. The goal is to reach a distributed open architecture with interoperability of heterogeneous systems, neutral and easy access, clear layering and resilience. It should provide the necessary network and information management services to enable reliable, secure and accurate interactions with the physical environment.
The idea is to provide an integrated platform that offers unified data access, processing and services on top of existing ubiquitous services of the Internet of Things to integrate heterogeneous sensors/actuators in a uniform way. From an application perspective, a set of basic services encapsulates sensor/actuator network infrastructures hiding the underlying layers with the network communication details and heterogeneous sensor hardware and lower level protocols.\\

A heterogeneous networking environment indeed calls for means to hide the complexity from the end-user, as well as applications, by providing intelligent and adaptable connectivity services, thus providing an efficient application development framework. To face the coexistence of many heterogeneous set of things, a common trend in Internet of Things applications is the adoption of an abstraction layer capable of harmonizing the access to different devices with a common language and procedures \cite{ERA09}. Standard interfaces and data models ensure a high degree of interoperability among multiple systems. However, typical drawbacks of misconfigurations and traffic congestions are normally exasperated by the node heterogeneity. These drawbacks will be overcome in the project through the adoption of islands architecture. On the one hand, smart gateways in their locations will hide all the complexities of the underlying standards. Hence, wireless connections of the devices/machines to the web/Internet, using communication technologies like Zigbee \cite{Zigbee12}, Z-wave \cite{Zwave12}, Wi-Fi \cite{Wifi}, plus the mapping of every device to a unique ID, will provide full, intelligent and secure control over it. With these smart boxes/gateways, technological limitations will completely disappear and the devices will become identifiable only through their functionalities with clear and consistent APIs. On the other hand, the middleware fog constitutes an abstraction level where all the devices, viewed as entities sharing their functionalities (avatars), are transparently managed and used. The goal is to manage the collaboration between heterogeneous devices through a simple API level in conjunction with the mentioned communication protocol able to reach the peer within the location singularly.\\

Most existing solutions for middleware adopt in general a service-oriented design tailored mainly to support a network topology of sensors that is both unknown and dynamic. But while some projects focus on abstracting the sensors in the network as services (such as in HYDRA \cite{ZH09, ERA10, ZH08}, SENSEI \cite{PBEV09}, SOCRADES \cite{GTKSS10}, and COBIS \cite{cobis07}), other projects devote more attention to data/information abstractions and their integrations with services (among which are SOFIA \cite{HLBT10}, SATware \cite{MDMV09}, and Global Sensor Networks GSN \cite{AHS07}). A common thread throughout all of these solutions, however, is that they handle the challenge of an unknown topology through the use of discovery methods that are largely based on the well-known traditional service/resource discovery approaches of the existing Internet, ubiquitous environments and wireless sensor and actuator networks \cite{AM08, ZMN05, MRPM08}. For instance, SOCRADES provides discovery on two levels, the sensor level and the service level, which can employ either standard web-service discovery or a RESTful discovery mechanism (for RESTful services). COBIS, on the other hand, uses its own service description language COBIL 2 (Collaborative Business Item Language), where service functions and keywords are annotated with a verbal description.
Another point of agreement in the state-of-the-art of middleware solutions is in the widespread use of semantics and metadata to overcome heterogeneity challenges. Indeed, it is standard practice to use ontologies to model sensors, their domains, and sensor data repositories  \cite{ERA10, ELS07, LZ05}. Some projects even go a step further and also include context information \cite{PBEV09}, or service descriptions \cite{ZH09, ERA10, ZH08}. And as a type of service composition, many projects support the concept of virtual/semantic sensors (for instance, in HYDRA, GSN and SATware), i.e. entities that abstract several aggregated physical devices under a single service. A different implementation of a similar idea, though, is provided in the SATware project, where virtual sensors actually correspond to transformations applied to a set of raw sensor streams to produce another semantically meaningful stream.\\

Regarding scalability, most projects address this challenge by pursuing modifications in the underlying sensor/actuator network topology. Sometimes, this is done by adopting fully-distributed infrastructures (such as in COBIS and SOFIA), and sometimes through an architecture of peer-to-peer clusters (e.g., GSN). In our view, however, while these approaches work well for the existing Internet, where traffic is made up of a relatively small amount of service interactions, they will not fit for the complex weave of interactions that will be commonplace in the Internet of Things. In the Internet of Things, a large number of requests will involve intricate coordination among millions of things and services, whereas on today's Internet most requests are largely point-to-point. Therefore, the number of packets transmitted in the network will grow strongly and nonlinearly as the number of available services increases. In such an environment, performing even a simple service discovery may exceed acceptable time, processing, and memory constraints. //NOTE: the last paragraph needs to be tied (cf. (2)/Fig.2 in the intro with the problems related to ''Possible Worlds Semantics'' and the efficient pruning the needs to be performed

\subsection{Data and Process Integration}
Heterogeneous data integration is a topic which, in addition to the database community broadly \cite{Cohen98,HalevyRO06}, has also been investigated by more specialized research foci: -- bioinformatics \cite{KirstenR06,MostafaviM10}; information management and text retrieval \cite{CouletGDAMS11,HammerGIPUW95,Torlone08}; sensing/actuation and tracking \cite{AvciTS16,ChawatheKRS04,SalarianCN12} (to name but a few). While the role and the impact of the semantics have been recognized and addressed \cite{BergamaschiCVB01,CastanoFMR04}, the existing results still have limited cross-contextual support. For instance, the works investigating the benefits of semantics for resource discovery (cf. \cite{CastanoFMR04}) do not take into consideration the role of different devices that may enter and/or leave a particular ''geo-social'' network dynamically, as well as their impact on real-time adjustments (as well as the impact of their cessation).
To-Do -- Goce:
Overview of:

a.	Data integration + distributed query processing; workflows~\cite{LiCLWPZB12,PandeyB12,PoolaRB16}; Wireless Actuator Networks

b.	Analytics and platforms in-context (Cloud/Hadoop; NoSQL; Warehousing)~\cite{ToosiCB14}


\section{Proposed research overview}\label{research-overview}

We now proceed with identifying the main challenges that the proposed project will address, with a brief overview of the proposed research tasks (research challenges), followed by an introduction of the team and their expertise.\\

Research Challenge 1 (RC-1): Data collection and processing\\
…\\
\\

Research Challenge 2 (RC-2): Knowledge discovery\\
…\\
\\

Research Challenge 3 (RC-3): Security, privacy (anonymity) and trust\\

The Internet of Things is foreseen to bring a multitude of services with a vision of creating a smart self-configuring and interconnected world for the benefit of end users. With the extensive research and development of computer, communication and control technologies, it is possible to connect all things to the Internet such that the so-called Internet of Things (IoT) can be formed. These things may be equipped with devices such as sensors, actuators, and tags, in order to allow people and things to be connected anytime and anywhere, with anything and anyone. IoT will enable collaborations and communications among people and things, and among things themselves, which expand the current Internet and will radically change our personal, corporate, and community environments. However, a plethora of security, privacy, and trust challenges need to be addressed in order to fully realize this vision [20, 21, 22, 23]. When more and more things connect to the Internet, security and privacy issues become more serious, especially in the case that these things are equipped with actuators and can support control.
For better protection of secure communication and user privacy, including location, identity and behavior habits, it is necessary to develop anonymous communication theories, methods and key technologies of anonymous communication systems in all varieties of application environments. Anonymous communication is used to hide communication participants or communication relations to achieve effective protection for network nodes and user identities. Anonymous communication can address potential network security issues, and becomes one of the hot topics in the field of network and information security. With reference to security, data anonymity, confidentiality, and integrity need to be guaranteed, along with providing authentication and authorization mechanisms in order to prevent unauthorized users (i.e., humans and devices) to access any system. Concerning privacy requirement, both data and user personal information have to be confidentially manipulated since devices may manage sensitive information (e.g., user habits, locations, etc). Finally, trust is also a fundamental issue since the IoT environment is characterized by multiple devices that have to process and handle the data in compliance with user needs and rights.\\

To realize a trusted, secure, and privacy preserving social network of things, we plan to address in this proposal the following security, privacy, and trust management related tasks:\\

T-3.1: Define a new scheme of anonymizing things and representing them as avatars in the social network of things.\\

T-3.2: Authentication of data collected in fogs in the presence of malicious attackers, despite the attack surface being very broad, ranging from PHY layer to application layer.\\

T-3.3: Achieving end-to-end secure and privacy-preserving information flow monitoring between users and things, covering all sort of things and access networks using physiological and physical layer propertie.\\

T-4.3: Design of secure, fast and friendly authentication schemes to allow users of the social network of things to access the data of interest through diverse mobile devices such as smartphones and tablets.\\

\section{Research tasks}\label{research-tasks}
We now proceed with a detailed presentation of the main research of the proposed project. 


\subsection{Data collection and processing} 
(Goce + Mubbasir)\\
Data cleaning, data collection, sharing, processing, analytics\\
How can we secure the collected data and how to make it available?\\
APIs and libraries for building applications 


Advances in global positioning technologies
(GPS)~\cite{MannuciA04} enabled a fusion of spatial~\cite{pelekis-r2,HjaltSamet99,SpatialShashi03}
and temporal~\cite{TemporalBook} databases, extending them
to the field of spatio-temporal and mobile data
management~\cite{MobiEyes2,IndexJensen06,OurGeoInf07,MokbelA08}. The main relevance for the 
proposed project is that the popular query categories 
are \textit{continuous} (i.e., their answers may have to be
re-evaluated based on the changes in the motion of the entities);
and/or \textit{persistent} (i.e., their answers may need to be
re-evaluated based both on the changes of the motion as well as
the history of such
changes)~\cite{OurGeoInf07}. From a complementary perspective, research works in energy-efficient 
tracking~\cite{AvciTS16,LiuS11,RenLC11} and query processing~\cite{MaddenTinyDB06} in Wireless Sensor Networks (WSN)
brought about the concepts of distributed/localized processing (i.e., minimizing the communication) along with the selection of
notes in-charge of a particular data-gathering and processing -- e.g., {\it tracking principals}~\cite{GhicaTZ10}, {\it cluster-heads}~\cite{Kulik99,Tavli06}, etc.

In the settings focal to the proposed project and management of evolving HACADAs, both bodies of existing works -- for which the co-PIs have research contributions and experience~\cite{AvciTS16,OurGeoInf07,GhicaTZ10,MyTODS04,ZhouT+12} -- can be used as leverage, however, there are quite a few additional unique challenges that we need to address.

\paragraph{Operands, Operations and Placements}

We assume that, at minimum, each WiT object will have the attributes described in Table~\ref{tab:table1}:

\begin{table}[h!]
  \centering
  \caption{WiT structure}
  \label{tab:table1}
  \begin{tabular}{l|l|l|l}
    IP\_Id & [(Value$_i$, Description$_i$)] & Location & Other Description\\
    \hline
    Unique IP & Types of values (sensed, transmitted, descriptive) & geo-location & Semantics; Operational Mode; etc.\\
  \end{tabular}
\end{table}

At a first approximation, a particular HACADA can be (logically) considered as a triplet $H_i =${\it ([WiT]$_i$, R(WiT$_i$,Wit$_j$), [$C_g$])}, where {\it [WiT]$_i$} denotes the list of its WiT constituents/objects; {\it R(WiT$_{ij}$,Wit$_{ik}$)} encodes a (possible) relationship  between the $j$-th and $k$-th WiT; and {\it [$C_g$]} denotes a (possible) list of {\it global constraints} such as {\it C1} introduced in Sec.~\ref{dat-cp} or other kinds, limiting the number of possible transitions among states.

One of the primary tasks of the proposed work will be to {\it define a set of operations} over the universe of HACADAs, along with efficient algorithms for their processing. But few examples:

\noindent $\bullet$ {\it merge($H_i$,$H_j$)} -- this operand will merge the representations $H_i$ and $H_j$ into a single HACADA $H_j$. The need to execute this operation may be due to optimization of various analytics tasks (e.g., detecting association rules).

\noindent $\bullet$ {\it split(H,$H_i$,$H_j$,A)} -- this operand will split an existing instance of a HACADA ($H$) into two new instances, based on a criterion pertaining to attribute $A$.

Even the two simple examples above have a lot of inherent complexities. Firstly, one may observe that different overloadings are possible: for instance, {\it merge($H_i$,$H_j$)} can also have a signature {\it merge($H_i$,$H_j$, $H_k$)}, indicating that a brand new HACADA instance is to be created -- and some of the traditional techniques and approaches (cf.~\cite{GoguenM92,JouannaudKKM92}) may need to be revisited for incorporating them in the IoT context~\cite{Adaikkalavan05,snoop11}. Secondly, context-based policies will need to be considered which, in turn, may affect the implementation: for example, should any of the input-operands continue to exist as an independent instance. Thirdly, as much as one can attempt to rely on existing techniques for integrating heterogeneous data~\cite{BergamaschiCVB01,Cohen98,CouletGDAMS11,HalevyRO06,SalarianCN12} -- how can different networking and/or security requirements be brought ``in concert'' with the specification of the declarative and/or procedural semantics of the operators?

An important question in its own right for this part of the proposed research is how can one selectively {\it place the operands and operations' execution} for the purpose of efficient execution of particular operations {\it and} how are the outputs of the operations to be placed, having in mind not only representation-related issues, but also the possibility of analytics and/or security based requirements. There are plethora of works from distributed query processing through processing aggregates in sensor networks, to resources re-allocation~\cite{DewanSHH94,GuoPZA14,JinC06,LiuOBC12,Nisheeth04,Synopsis} to leverage upon -- however, one cannot expect that a straightforward adaptation and/or extension of the existing techniques can yield good performance, especially in dynamic scenarios of multiple WiTs from completely heterogeneous sources entering and exiting the working-context. We will attempt to couple the existing works with our recent results on on-demand resource guidance in mobile sensor networks and detection of motion trends~\cite{AvciTTSZ16,MohamedKT15} in order to ensure that the processing of the operands of interests, as well as evaluation of queries/predicates of interest is done in an optimized and balanced manner~\cite{omCom16-1}.




\paragraph{Uncertainty and Data Compression in Evolving Contexts}

%%of
%%spatio-temporal queries: \textit{range, (k) Nearest-Neighbor
%%((k-)NN)}~\cite{MyTODS04,XiongMA05,YuPK05} are typically: (1)
%%\textit{continuous} (i.e., their answers may have to be
%%re-evaluated based on the changes in the motion of the entities);
%%and/or (2) \textit{persistent} (i.e., their answers may need to be
%%re-evaluated based both on the changes of the motion as well as
%%the history of such
%%changes)~\cite{MobiEyes2,MokbelA08,OurGeoInf07}. 

One of the challenges to be addressed in the proposed project is how to properly incorporate the uncertainty in all the aspects of the query/operations processing. Its sources can be plentiful -- from errors in the values sensed, to errors due to attempting to represent a continuous phenomenon with a discrete samples and use of interpolation in-between~\cite{DevendranL14,GoodchildZK09,HunterG96}. A specific source of uncertainty is the quest for a compactness in the representation -- i.e., data compression, which is sometimes essential (like in the settings of streaming data)~\cite{CormodeMYZ12,CormodeGHJ12}. 
However, unless it is properly captured in the very syntax of the predicates and operators, as well as in the processing algorithms -- its impact can be unpredictable~\cite{MyTODS04}. Throughout the proposed work, we will capitalize on the works coupling aggregation and compression in WSNs~\cite{LinGKL05,KadayifK04,PattemKG08} and our recent results on fusing uncertain data from heterogeneous sources~\cite{ZhangTL16} to formalize the representation of the uncertainty when managing the instances of HACADAs and the execution of the novel operands.

An important component of our research will be how to pro-actively steer the collaborative orchestration of the processes of data generation, compression, (re)placement -- along with queries processing and actuation. We will leverage on our works on proactive management of resources in WSNs~\cite{McClurgTY12} and attempt to apply the concept of evolving triggers~\cite{old-r35} to balance the impact of the (bounded) uncertainty on the quality of the service/experience. This part of the research will be coupled with the challenges addressed in the context of detecting the placement of the predicates whenever multiple levels of granularity may need to be maintained about a data of evolving nature and with semantic annotations~\cite{VaismanZ09,TrajcevskiDVAZT15}.



\paragraph{Knowledge Representation and Reasoning}

The processed data from above will be represented as a set of discrete time series, or \emph{fluents} which encode the states and properties of different \avatar which change over time. \todo{give example}. To facilitate efficient reasoning and knowledge discovery in a dynamic smart and inter-connected system of avatars, we will consider a level of abstraction, termed as \emph{events }. Events constitute meaningful interactions between two or more \avatar and serve as the basis for identifying relationships between them towards creating social networks of avatars. \todo{give example}. We define the specific terms and concepts we will use for event-centric knowledge representation and reasoning below.

\noindent \textbf{Smart Avatars} The notion of smart objects~\cite{Kallmann:1999:DIS:323663.323683} has been popularly used in the graphics and animation community to embed intelligence and semantics in virtual objects. We extend this formalism to represent both IoT objects as well as IoT users. This unified formalism allows us to seamless consider sensors, controllers, actuators, and human users within the same social community.  We define a smart avatar $\smartobj{} \in \world$ as $\smartobj{} = \langle \affordances, \state{} \rangle$ with a set of advertised affordances $\affordances$ and a state \state{}. An affordance $\affordance(\affordanceowner, \affordanceuser) \in \affordances$ is an advertised capability offered by a smart avatar that manipulates the states of the owner of an affordance $\affordanceowner$ and a smart avatar user $\affordanceuser$.

\noindent \textbf{State.} The state $\state{} = \langle \attributeSymbol, \relationshipSymbol \rangle$ of a smart object $\smartobj{}$ comprises a set of attribute mappings \attributeSymbol (fluents), and a collection of pairwise relationships $\relationshipSymbol$ with other avatars. With this representation, we can make logical inferences between objects using a declarative PROLOG-like knowledge reasoning engine.
%An attribute \attributemapping{i}{j} is a bit that denotes the value of the $j^{th}$ attribute for $\smartobj{i}$.



%A specific relationship $\relationship{\cdot}{\cdot}{a}$ is a sparse matrix of $|\world| \times |\world|$, where \relationship{i}{j}{a} is a bit that denotes the current value of the $a^{th}$ relationship between \smartobj{i} and \smartobj{j}.


%\noindent \textbf{Rules.}  A rule $\Rule_x(i,j) \in \Rules$ between two smart objects \smartobj{i}, \smartobj{j} is true or false, depending on the states and relationships of both objects. Rules allow for logical inference between objects and are used for reasoning such as evaluating whether a character can access a particular room, or manipulate another smart object based on the current world state. They are defined and solved using a declarative PROLOG-like interface

\noindent \textbf{Events.} Events are used to encode context-specific interactions between two or more smart avatars, and provide an appropriate level of abstraction for knowledge discovery  An event is formally defined as $\event{} = \langle \tree,  \precondition{}, \postcondition{} \rangle$. A precondition $\precondition{}: \groupState{\smartObjectGroup{}} \leftarrow \{\true, \false \} $ is a logical expression on the compound state \groupState{\smartObjectGroup{}} of a particular set of smart avatars $\smartObjectGroup{}: \{ \smartobj{1}, \smartobj{2}, \ldots \smartobj{|\roleSet{}|} \}$ that checks the validity of the states of each smart object. \precondition{} is represented as a conjunction of clauses $\clause{}{} \in \precondition{}$ where each clause \clause{}{} is a literal that specifies the desired attributes of smart objects, and relationships between pairs of participants. A precondition is fulfilled by $\smartObjectGroup{} \subseteq \world $ if $\preconditionFunction{\event}{\smartObjectGroup{}} = \true$. The event postcondition $\postcondition{} : \groupState{} \rightarrow \nextGroupState{}$ transforms the current state of all event participants \groupState{} to \nextGroupState{} by executing the effects of the event. We can extend this definition to model non-deterministic, fuzzy events with a probabilistic notion of success or failure. Events may optionally have a controller which defines the series of affordance activations within the smart avatars to produce its desired outcome. We represent this control logic using an extended version of Behavior Trees that facilitate parameterization. Parameterized Behavior Tree (PBT)~\cite{Shoulson:2011:PBT:2177817.2177835} are an effective model for representing coordinated control logic  between multiple smart avatars.

PI Kapadia has extensive prior experience in developing event-centric knowledge bases for inference and reasoning in virtual worlds~\cite{Shoulson:2013:EPA:2522628.2522629,2015-fdg-bstl,Kapadia:2015:CAI:2699276.2699279,Kapadia:2016:CCN:2982818.2982846} which will be extended to represent the IoT domain.


%ADAPT~\cite{Shoulson:2013:EPA:2522628.2522629}
%BSTL~\cite{2015-fdg-bstl,Kapadia:2015:CAI:2699276.2699279}
%behavior authoring~\cite{behaviorCGA,Kapadia:2011:BAC:1944745.1944779}
%story world~\cite{AIIDE1511583}
%CANVAS~\cite{Kapadia:2016:CCN:2982818.2982846}





%$\roleSet{} = \{ \role{i} \}$  define the desired roles for each participant. \role{i} is a logical formula specifying the desired value of the immutable attributes \attributemapping{\cdot}{j} for \smartobj{j} to be considered as a valid candidate for that particular role in the event.


%An event instance $\instance = \langle \event, \smartObjectGroup{} \rangle$ is an event \event populated with an ordered list of smart object participants \smartObjectGroup{}.  $\preconditionFunction{\event}{\smartObjectGroup{}} = \true$. The event postcondition $\postcondition{} : \groupState{} \rightarrow \nextGroupState{}$ transforms the current state of all event participants \groupState{} to \nextGroupState{} by executing the effects of the event. When an event fails, $\nextGroupState{} = \groupState{}$. An event instance $\instance = \langle \event, \smartObjectGroup{} \rangle$ is an event \event populated with an ordered list of smart object participants \smartObjectGroup{}.

%? which represent the specific values of avatar fluents at specific points. Events occur on a set of avatars which are the participants, and produce a change in state of these avatars which are defined as event postconditions. Events may additionally be pre-conditioned on the participant states to indicate what conditions must be satisfied in order for an event to successfully execute.

This symbolic representation of avatars and their dynamic states in terms of key events will facilitate the development of an event calculus ? allowing us make inferences about the relationships and properties of these avatars using first and second-order logic. This reasoning will be used as the basis for knowledge discovery described in Section XX. Knowledge discovery will entail the identification of relationships between the properties of different avatars, clustering of avatar to create compound entities, and the discovery of salient events.


 

\subsection{Knowledge discovery: social network of avatars (virtual world)}
(Ashfaq + Vladamir)\\
….\\
…\\

\subsection{Security, privacy and anonymity}
(Farid + Alex)\\
…\\
Data security\\
System level security\\
Privacy, anonymity\\
\\
Greedy behavior (?)\\
Key sharing problem\\ 
Privacy, trust and anonymity \\

\subsection{Implementation and validation }
(All)\\
\\
Letters of Collaboration)\\
Campus wide implementation \\
Letters of Commitment\\
Shall we rely on KAA?\\
APIs: flexibility of formats/schemas\\


\section{Broader Impact}\label{broader-impact}
From Rutgers:

The proposed highly interdisciplinary project combines computer vision, human modelling and simulation, optimization, robotics and autonomy, and control theory, and has the potential to impact several research areas. The proposed sensing, simulation, and optimization framework will have wide applicability in various scientific areas involving decision-making in heterogeneous and dynamic networks comprising human-operated, and autonomous sensors, with concepts that could easily generalize to event understanding in complex social, economical or cyber-biological systems, thus impacting multiple societal applications including urban infrastructure, emergency response, safety, and quality of life.

\noindent \textbf{Datasets.} We will release the reconstructed time-stamped behaviors that are collected as part of our real-world experiments in the classroom and the dining hall. This real-world dataset will include both the original, unprocessed, noisy, incomplete trajectories, as well as the processed trajectories for researchers to use and compare, when developing their own algorithms. We will also release synthetic datasets from our simulation experiments using the data-driven crowd simulator that will be developed as part of this research. All crowd datasets (both real and synthetic) will be captured for both the original, un-optimized environments, as well as the optimized layouts. We will also release extended designs actuated of environmental elements such as mobile queue separators and mobile furniture.

\noindent \textbf{Open-Source Software.} The PIs have an established record of releasing software packages related to crowd modeling and simulation ~\cite{steersuite,10.1109/TVCG.2014.251}. We will build on top of these foundations and extend our existing open-source platforms to include software solutions for: (1) crowd trajectory reconstruction, (2) data-driven crowd simulation, (3) static and dynamic analysis of environments, (4) crowd-aware environment optimization, and (5) environment reconfiguration planning.

\noindent \textbf{Kaggle Competitions}. We plan to organize 2 public data analytics competitions using the collected data, hosted on kaggle.com. Kaggle is a platform for predictive modelling and analytics competitions, which is often used in academia to incentivize students to tackle challenging problems through a combination of research, system building, and peer-competitions. The goals will include trajectory estimation and reconstruction, data-driven crowd modelling, and computer-assisted designs of environments.

\noindent \textbf{Curriculum Development and Outreach.} The PIs will generate educational material on the autonomy, hardware design, sensing and simulation aspects. PI Kapadia will design a graduate seminar on \emph{Crowd-Aware Smart Environments} that will introduce students to concepts in crowd simulation, and environment optimization. Advanced undergraduate students will also be permitted to enroll. PI Pavlovic will introduce an assignment in the machine learning course where students will learn generative models of crowd movement. The PI's will collectively organize a workshop on crowd-aware cyber-physical environments during CPS week (tentatively April 2017), which will bring together researchers from various disciplines including computer vision, machine learning, simulation, robotics, and potential adopters of the technology. Our research environment will provide interdisciplinary training ranging from distributed sensing and control, simulation and optimization, to robot motion planning.

\noindent \textbf{Under-Represented Groups and K-12 Level}. Our project can help attract a diverse group of students and broaden the diversity of students recruited into these and other STEM disciplines. Rutgers office of Enrollment Management conducts the Rutgers Future Scholars Program (RFS), which provides mentoring activities to grade 9-12 minority students from disadvantaged backgrounds and full scholarships to undergraduate programs. We will closely work with this office to participate in outreach, enrichment, and mentoring activities. We will also leverage other programs to recruit and mentor students from under-represented groups: the summer undergraduate Project RiSE (Research in Science and Engineering) for undergraduates from underrepresented populations for 8/1-week intensive summer research internships; and the Aresty Center for Rutgers Undergraduate Research- places Rutgers undergraduates in research laboratories on campus. PI Pavlovic has a multi-year track record of working with over 14 minority and female students through Aresty and RiSE, as well as multiple REUs. We will leverage the extensive diversity programs at Rutgers to recruit and support women and underrepresented minorities.


\section{Results from current and recent prior NSF support}\label{prior-projects}


{\bf Aleksandar Kuzmanovic} has done extensive work in 
congestion control \cite{ecn05sigcomm,ecn05rfc,web06par,tcp-lp,tcp-lpToN,hstcp-lp,extr08ccr},
DoS resiliency \cite{receiver07jcn,friendly07infocom,dos05p2p,cache06icnp,tcpoison07icnp,shrew03sigcomm,shrewToN,aj08imc,adver10infocom},
measurements \cite{akamai06sigcomm,pong07sigmetrics,pong08icnp,KuzKni01,kuzkni-tpds,akamai08icdcs,aj08imc,google08sigcomm,serendipity09imc,p2p10infocom,geo11nsdi}, and Web \cite{geo11nsdi,migration11wi,spam13pam,selective12wi,adver11icdcs,myths10wi,ad10www,ad10infocom,myths14tweb,synthoid14wi,geoecho14wi,mosaic13sigcomm,glance13wow,fusion11,fusion13,move12www,vision12}.
He is currently the PI of CNS-1319086, \$473,445, 8/30/13 through 8/29/17, \emph{NetS: Small: Endpoint User Profile Control}. The \textbf{Intellectual Merit} of this proposal lies in developing auditing mechanisms on the web, and the key
\textbf{Broader Impact} of this proposal is the deployment of a set of first such tools for the Web; the list of publications resulted from this project is \cite{spam13pam,migration11wi,selective12wi,adver11icdcs,myths10wi,ad10www,ad10infocom,myths14tweb}, and the list of research products from this project is available at \url{http://networks.cs.northwestern.edu/audit-content/}.

...\\

...\\


\newpage
\section{Data management plan}\label{data-management-plan}
\section{Data Management Plan}

\subsection{Types of Data} 
The proposed research will generate the following material and data that might be of use to the general scientific community: 
\begin{itemize}
\item Open-source software, either in the form of stand-along packages or modules for existing packages, such as the following: 
	SteerSuite, developed by Co-PI Mubbasir Kapadia, which is an open framework for developing, evaluating and sharing steering algorithms,
\item Robot Operating System (ROS), a popular middleware for communicating with robotic devices and sensors, 
\item Open Motion Planning Library (OMPL), a software package developed by Rice University for sharing motion planning algorithms 
\end{itemize}

Data collected from experiments evaluating different approaches for sensing and modeling heterogeneous cyber-human IoTs, as well as for the control of IoTs. The experiments will correspond to simulated challenges, using models of real systems and crowds, as well as physical experiments in office and classroom environments given the platforms developed as part of this project. The data will correspond to the available information to the overall system (e.g., the crowd’s configuration), and the choices made by different algorithms. 

\subsection{Standards and Formats} 
The input files and software developed will be compatible with popular protocols in the research community, such as those followed by the ROS software package. For instance, the control and planning methods will be decomposed into multiple processes that will communicate using the messaging infrastructure of ROS. This will also allow the processes to be compatible with existing and future packages developed by different research teams that follow a similar protocol and use the same middleware. 
The format for the data will be documented and provided in parallel with the data. An effort will be made to follow standards in the motion planning community (e.g., current format used by OMPL) and it is expected that the data will be stored in YAML files that will be compatible with ROS’ parameter server and easily readable by people. 

\subsection{Provisions for Archiving and Preservation}
The electronic data will be stored on multiple workstations and notebooks used by the PIs and the graduate students involved in the project. They will be backed up through the use of versioning control systems (Mercurial, Subversion) on the servers of the Computer Science Department at Rutgers University. These servers are backed up regularly off-site. The software modules will be also stored externally at online repositories, such as BitBucket. 

No additional funding or special institutional commitment or special preparation for long-term preservation will be required for the preservation of the data as described above. 

\subsection{Access and Re-Use Policies and Provisions}
The research groups websites’ will provide access to data as they will become available, as well as the software modules that will be developed, documented and prepared for sharing. Sharing of data and software modules will take place no longer than a year after their generation. If requested, access to data or software will be provided earlier via contact with the PIs. 

The software will be using the BSD license, an open-source software license. BSD is a permissive free soft- ware license, which introduces only minimal requirements about how the software can be redistributed. 

There is no anticipation of significant ethical or intellectual property or confidentiality or copyright issues involved with the acquisition of the data. In the event that discoveries or inventions are made in direct connection with this data, access to the data will be granted upon request once appropriate invention disclosures and/or provisional patent filings are made. 

\subsection{Plans for Transition or Termination of the Data Collection}
The data and software modules will be preserved for at least three years beyond the award period, as required by NSF guidelines. The Principal Investigator, assisted by the Information Technology staff of the Computer Science Department at Rutgers University, will be responsible for managing the data beyond the expiration of the research project. No special preparation is needed for migrating, deleting or transitioning the data to another media into the future. 

The data acquired and preserved in the context of this proposal will be further governed by Rutgers University’s policies pertaining to intellectual property, record retention, and data management. 

\newpage
\section{Project Collaboration and Management Plan}\label{project-management-plan}

The collaboration efforts and the management structure are organized around the objectives of the project, the outcome assessment, and research team expertise. The collaborative effort will extend beyond the scientific research to teaming on testbed development, student supervision and training, and educational activity.

\subsection{Research team members and their expertise}
The research project pools diverse expertise from Illinois Institute of Technology (IIT), Northwestern University, and Rutgers University needed for the successful completion of the proposed interdisciplinary research activity. The project team has a prior history of successful collaborations. The team members and their complementary research expertise are as follows:\\

\textbf{Ashfaq Khokhar (AAK), IIT}, (Routing and MAC Layer in WSN, Data Security, Privacy Preserving Data Mining, Power Efficient Scalable Computing),  

\textbf{Farid Nait-Abdesselam (FN), IIT}, (Wireless Networks, Protocol Security, Data Analytics), 

\textbf{Mubbasir Kapadia (MK), Rutgers University}, (Machine Learning, Computer Vision), 

\textbf{Vladimir Pavlovic (VP), Rutgers University}, (xxx), 

\textbf{Aleksander Kuzamanovic (AK), Northwestern University} (xxx),

\textbf{Goce Trajcevski (GT), Northwestern University}, (Moving Objects Databases, Uncertainty in Mobile Data, Active Databases and Triggers). \\

The PIs involved in this project have a history of research collaboration on jointly funded projects. Dr. Khokhar, Dr. Trajcevski, and Dr. Nait-Abdesselam (then an international partner from France) have collaborated on a Large NeTS-NSF award, have co-advised PhD students, and have co-authored several conference and journal publications. 
\colorbox{red}{ADD OTHERS HERE…}\\
The collaborative effort will extend beyond the scientific research to teaming on testbed development, student supervision and training, course development, workshop organization, and performance evaluation of the technology developed.

\subsection{PI research collaboration} 
If the team awarded the funding, a kick-off meeting will be held within a week of the announcement for all investigators, participating senior research personnel, and graduates students. A broad overview of the research activities will be presented and yearly research targets and deliverables will be set. We will hold biweekly meetings with the attendance of the PIs and Ph.D students from all the campuses. Towards this end we plan to use social media tools such as Google Hangouts and Zoom.

The major research tasks, group interactions, research outcomes, deliverables, and possible applications likely to emerge out of the proposed research are depicted in Figure~\ref{CM-1}. The activities in the four research threads and their linkages are captured in the four rectangular boxes on the left. We will have the following collaborative roles for the PIs in the research activities of the project effort, with the team lead identified in bold:\\

\textbf{Trajcevski}, Kapadia, Pavlovic: Data Processing and Information Modeling 

\textbf{Pavlovic}, Khokhar, Kuzmanovic, Nait-Abdesselam: Data Analytics

\textbf{Kuzmanovic}, Khokhar, Nait Abdesselam: Security and Proivacy

\textbf{Nait-Abdesselam}, Kapadia, Trajcevski: Testbed Implementation and Validation \\

These roles are also identified (with the abbreviations of their first and last names) as white ovals in Figure~\ref{CM-1}. Also, we envision realization of the objectives defined for the research activities will produce three outcome components (i) Efficient IoT Representations, (ii) Privacy Preserving Data Analytics, and (iii) Private Secure Connectivity among heterogeneous devices and users. These in turn will be assembled into the overall deliverable of comprehensive software system and testbed for realizing Dynamic Data Driven Cyber Human Networks consisting of IoT avatars. Research in all key areas will proceed in parallel and overlapping phases, and the planned progress over a three-year timeline is shown in quarters at the bottom in Figure~\ref{CM-1}. 

\begin{wrapfigure}{R}{0.70\textwidth} \vspace{-3mm}
	\centerline{\includegraphics[width=0.61\textwidth]{./Timeline-v1.jpg}}
	\vspace{-3mm} \caption{Relationship of major research tasks, group interactions, research outcomes, deliverables, and possible applications likely to emerge out of the proposed research}
	\label{CM-1}
	\vspace{-3mm}
\end{wrapfigure}

The management structure is organized around the main research themes identified in Section 4.  Prof. Khokhar will assume overall responsibility for coordination of research activity. He will be in regular contact with the other investigators and personnel. He has extensive experience in managing large domestic and international teams and has organized NSF-sponsored workshops. The progress of then project will be monitored by a Project Steering Committee, consisting of one Co-PI from each institution. Three working groups will oversee the research on the main research threads described in the proposal, and will be responsible for assessing the research outcomes. Due to the need for collaborative work on the research activities, each investigator will serve on at least two of the three working groups.


\subsection{PI research collaboration and student supervision} 
Upon the award of the project, a kick-off meeting will be held within a week of the announcement for all investigators, participating senior research personnel, and graduates students. A broad overview of the research activities will be presented and yearly research targets and deliverables will be set. We will hold monthly cross-institutional and bi-weekly intra-institutional meetings with the attendance of the PIs and Ph.D students. 

In the first eight quarters, the proposed research activity will include participation of \textit{one postdoc and three graduate students} supported by the project and several undergraduate students carrying out Honors College activity. During the last eight quarters, the project will support \textit{seven graduate students and a Post-doctoral} researcher. The students will receive research training in diverse emerging topics such as Information abstraction, data analytics, and robust privacy and security of networks and its components. Since the work requires collaboration beyond the immediate field of interest, the students will be co-supervised by faculty from participating institutions and will gain extensive experience in inter-disciplinary research activity.

\subsection{Dissemination plan} 
A project website will be established to disseminate the results and the code developed during the project. We will organize at least one workshops in a leading conference on IoTs promote our research among peers working on related research issues. In addition we will work to organize special issues of journals on these topics. The travel cost budgeted in the proposal is related to these dissemination activities

\subsection{Management of testbed development and technology transfer} 
The protocols and techniques developed in this project will be evaluated using an experimental testbed consisting of a network of IoT devices. This effort will be managed jointly by Dr. Khokhar and Dr. Trajcevski. Furthermore multiple commercial organizations working the domain of smart sensor systems and applications has indicated strong interest in the proposed project. For additional details please we refer to the letters of collaboration uploaded as single copy documents.


%%%\newpage

%%%\mypage{M}

%%%\section{Management Plan and Validation}%%
%%%\input{ManagementPlan-v5.tex}
%%%\label{management}

%%%%%%%%%%%Take a look at Phase Transition Phenomena in Wireless Ad-Hoc Networks

\newpage
\mypage{References }

\bibliographystyle{plain}
\bibliography{goce,farid,kapadia,vladimir,neo,akuzma,akuzma1,bias,bib,bib2,chartad,emubrowse,google-ranking,mislove,proposal,softid,sysnet,trends,web-load,wireless}
%\bibliography{comp11mc-GT-Comprehensive-2011,cps_raf22,pdinda,CPS2012-nikos-12,WSN22,Mokbel,shekhar-refcs,terveen,mining,vipin-references,minnesota-refcs,Prior-bib}

\end{document}
=======
\documentclass[10pt]{article}
%%%\documentclass{article}
%%%\usepackage[2-16]{pagesel}
%\usepackage[english,spanish]{babel}

\usepackage{times}
\usepackage{cite,verbatim,rotating}
\usepackage{subfigure}

\usepackage[paper=letterpaper,margin=1in,ignorefoot,ignorehead]{geometry}

%\usepackage{fullpage,setspace}
\usepackage{url}
%\usepackage{algorithmic}
%\usepackage{algorithm2e}
\usepackage{amsfonts}
\usepackage{amssymb}
%\setcounter{tocdepth}{3}
%\usepackage{simplemargins}
\usepackage{graphicx}
\usepackage{wrapfig}
%\usepackage{amssymb}
\usepackage{amsmath}
%\usepackage[dvips]{graphicx}
%%% remove comment delimiter ('%') and specify parameters if required
%\usepackage[dvips]{graphics}
%\setallmargins{1.0in}

\makeatletter
\renewcommand\section{\@startsection{section}{1}{\z@}%
                                    {-0.5ex \@plus -1ex \@minus -.2ex}%
                                    {1.0ex \@plus.2ex}%
                                    {\normalfont\Large\bfseries}}
\renewcommand\subsection{\@startsection{subsection}{2}{\z@}%
                                     {-0.5ex\@plus -1ex \@minus -.2ex}%
                                     {0.5ex \@plus .2ex}%
                                     {\normalfont\large\bfseries}}
\renewcommand\subsubsection{\@startsection{subsubsection}{3}{\z@}%
                                     {-0.5ex\@plus -1ex \@minus -.2ex}%
                                     {0.5ex \@plus .2ex}%
                                     {\normalfont\normalsize\bfseries}}
\renewcommand\paragraph{\@startsection{paragraph}{4}{\z@}%
                                    {-0.5ex \@plus -1ex \@minus.2ex}%
                                    {-2em}%
                                    {\normalfont\normalsize\bfseries}}
\renewcommand\subparagraph{\@startsection{subparagraph}{5}{\parindent}%
                                       {0.5ex \@plus1ex \@minus .2ex}%
                                       {-1em}%
                                      {\normalfont\normalsize\bfseries}}
\makeatother

\setlength{\abovedisplayskip}{2pt plus 2pt minus 2pt} \setlength{\belowdisplayskip}{2pt plus 2pt minus 2pt}
\setlength{\abovedisplayshortskip}{0pt} \setlength{\belowdisplayshortskip}{\belowdisplayskip}
\setlength{\itemsep}{0pt}

\setlength{\textfloatsep}{2 mm} \setlength{\dbltextfloatsep}{0 mm}
\setlength{\parindent}{0mm}
\setlength{\parskip}{0.4mm}

\newcommand{\mypage}[1]{\renewcommand{\thepage}{#1-- \arabic{page}}
\setcounter{page}{1}}

\renewcommand{\figurename}{Fig.}

%\renewcommand{\thesubfigure}{\figurename\ \thefigure.\alph{subfigure}}
%\makeatletter
%\renewcommand{\@thesubfigure}{\thesubfigure:\space}
%\renewcommand{\p@subfigure}{}
%\makeatother

\renewcommand{\baselinestretch}{0.940}
%%%%%%%%%%%%%%%%%From RANDY
\newcommand{\sens}{z}
\newcommand{\comnbrs}{{\mathscr{S}}}
%\newcommand{\physnbrs}{{\cal X}^{\text{phys}}}
\newcommand{\sensnbrs}{{\mathscr{X}}}
\newcommand{\metastate}{{\mathscr{Y}}}
\newcommand{\R}{{\ensuremath{{\mathbb{R}}}}}
\providecommand{\norm}[1]{\lVert#1\rVert}
\newcommand{\environ}{{\mathscr{E}}}

%\usepackage[pdftex]{graphicx}
%\vfuzz2pt % Don't report over-full v-boxes if over-edge is small
%\hfuzz2pt % Don't report over-full h-boxes if over-edge is small
%
\DeclareGraphicsExtensions{.jpg,.pdf,.mps,.png,.eps} %For pdftex

\newtheorem{definition}{Definition}[section]
\newtheorem{property}{Property}[section]
\newtheorem{corollary}{Corollary}[section]
\newtheorem{propose}{Proposition}[section]
\newtheorem{theorem}{Theorem}[section]
\newtheorem{lemma}{Lemma}[section]
\newtheorem{example}{Example}[section]

\DeclareMathOperator{\Cov}{Cov}
\providecommand{\abs}[1]{\lvert#1\rvert}
\providecommand{\norm}[1]{\lVert#1\rVert}
\newcommand{\E}{\ensuremath{\mathcal E}}
\newcommand{\F}{\ensuremath{\mathcal F}}
%%%%%%%%%%END FROM RANDY%%%%%%%%%%%%%

\usepackage{fancyhdr}
\setlength{\headheight}{12pt} \pagestyle{fancy}

%%%\fancyhf{} \lhead{\small Project Description} \rhead{\footnotesize
%%%CPS:Large:Collaborative: {\bf CASAP} -- Context-Aware Systems for
%%%Agricultural Practices} \cfoot{\thepage}

%%%\small
\fancyhf{}
%\lhead{Context-Aware Management of Heterogeneous Sensor Networks}
%%%\lhead{\small Project Description}

%\rhead{\footnotesize CPS:Frontiers:Collaborative Research:Dynamic Entanglement of Control and Data Management in Precision Agriculture} \cfoot{\thepage}

\rhead{\footnotesize Dynamic Data-Driven Networks of Cyber-Human Internet of Things for Smart and Connected Communities} \cfoot{\thepage}

\normalsize

\newcounter{taskcount}

\setcounter{taskcount}{0}

\newenvironment{task}
{% This is the begin code
\refstepcounter{taskcount} \textbf{\underbar{Research Tasks Group
(RTG)-\arabic{taskcount}:}} \itshape }
{% This is the end code
}

\renewcommand{\thetaskcount}{\textbf{RTG-\arabic{taskcount}}}

\newenvironment{plan}
{% This is the begin code
\textbf{\underbar{Proposed Approaches:}}} {}

\begin{document}

%\flushbottom

%%%\pagestyle{empty}
\mypage{Project Description }

\section{Introduction}\label{intro}%%

According to IDC (www.idc.com) -- a global provider of market intelligence, advisory services, and events for the information technology, telecommunications and consumer technology markets -- the worldwide shipment of smart connected devices will surpass 2.5 billion units in 2017. The decreasing trends of cost, size, and power requirements of the various devices, were accompanied by an increase in embedded computing capabilities. In addition -- every device can be connected with other devices, each of which: (1) is capable of sensing/computing, communicating and actuating; (2) depends on values and state-descriptions of other devices, not necessarily within geo-spatial proximity nor with homogeneous physical characteristics, to steer the course of its own activities.  Such heterogeneous device range from computers, tablets and smartphones, through embedded systems in cars and traffic lights, precision agriculture sensors and actuators, to various home appliances, bodily sensors, etc.~\cite{Al-FuqahaGMAA15,AtzoriIM10,MelodiaPGA07,SalarianCN12}. According to recent a study by Goldman \& Sachs, over 12 billion devices are already connected to the Internet of Thing (IoT) and by 2020 that number should become 20 billion~\footnote{http://www.goldmansachs.com/our-thinking/pages/iot-meets-everything.html}.

%%People and data are already online. Soon, all sort of things with sensors, actuators and places will be also online. All sectors of activity will %%benefit from applications that leverage the interconnection of embedded systems with IT systems. Intelligent transport solutions can speed up %%traffic flows, reduce fuel consumption, and safe lives. Smart farming solutions will improve the production and delivery of safe and healthy food. %%Remote health monitoring will provide convenient access to healthcare, raises it quality, and save money. Utilities management, education, %%government services, retail sectors etc.… all will also benefit from such couplings and the things will leave across all different systems. %%\textbf{The next step is to interconnect these individual systems into a system of systems, and here we are confronted to such a scale and %%heterogeneity that we are reaching a limit in networking and software engineering}.\\

Recent works have already attempted to capitalize on the results from social networks research~\cite{AhmedEH15,Hai-Jew16,MorrisTP10,VenturiniJL15} -- namely~\cite{AtzoriCI14,IeraMA15} have specifically pointed how the experiences from social networks of users equipped with smart devices could be generalized to {\it social networks of devices} in the IoT realm. In a similar spirit, attempts have been made to demonstrate the benefits of {\it virtual objects} as a good abstraction for tackling IoT-related problems~\cite{NittiPCA16}. Studies regarding integration of heterogeneous devices in various contexts abound -- e.g., from smart homes in a manner that would balance the Quality of Experience (QoE) and Energy consumption~\cite{FlorisMPA15}; to better use of smart grid~\cite{LuoAM14} and enabling smart cities~\cite{ZanellaBCVZ14}.

One of the core motivations for the proposed project is based on the observation that current practices related to the design of IoT systems are still rather compartmentalized, and the dynamics-aware fusion of data collection, analytics processing and actuation across a variety of heterogeneous IoTs has limited support. As a simple example, consider a scenario where one has a (collection of) smart bed(s), such as, e.g., RestBed (https://www.restperformance.com/) with a surface equipped with multiple sensors to improve sleeping experience. A smart home may, in addition to a RestBed, also have smart refrigerators such as, e.g., Family Hub from Samsung (http://www.samsung.com/us/explore/family-hub-refrigerator/). While the QoE estimate can be obtained based on customers survey (or, even from bodily sensors), and it can even be balanced with the energy-efficiency (cf.~\cite{FlorisMPA15}) ``in-concert'' with such devices, there are some important functionalities that are lacking, and could yet significantly improve the overall experience and efficient use of the available devices/technology:

\begin{itemize}
\item The quality/satisfaction of sleeping is not correlated with the nutritional habits (i.e., the content of the food in the smart refrigerator and consumption patterns). This, in turn, prevents the respective manufacturers from improving their products/devices in a collaborative and context-aware manner.
    
\item There is no formal methodology of {\it where} the data assembling (as well as data aggregation and decision/recommendation as well as actuation) should take place. For instance, should all the nutrition-relevant data from an ensemble of smart refrigerators be sent at the local grocery store to optimize deliveries, or should it be aggregated at some geo-social level of hierarchy. Moreover, this raises the question of how the aspects of privacy and security should be handled.
\end{itemize}



%In this project, we aim to investigate issues related to the design and development of a secure physical and computational networked environment, %consisting of a large collection of connected smart objects (things). The proposed environment would allow interoperability and would also %facilitate efficient collection and processing of data from these {\it things} and development of analytics based on the collected data. Based on %these analytics, the {\it things} would be able to relate to each other to form dynamic networks (akin to social networks, in this case, of {\it %things}). These smart objects may come from diverse environments such as smart appliances at homes, smart mobile surveillance system such as fire %trucks and police cars, smart transportation entities (e.g., cabs or buses), etc. 



%Nowadays, existing start of the art related to the design of IoT systems is highly compartmentalized, the data collection and analytics processing %across a variety of heterogeneous IoTs has limited support, and the interoperability of objects across domains has not been addressed at a level %that would provide the kinds of optimizations that this project aims to achieve. In this project, we introduce the concept of collaborative Social %Network of Things that would enable data and analytics sharing across objects that exhibit similar operating characteristics and/or have similar %profiles in terms of usage and user/manufacturer parameters. In addition, we propose to develop a hierarchical middleware implemented as fogs to allow interoperability with context-aware security properties associated with different nodes. These middleware fogs, mainly implemented in smart boxes/gateways within local networks, will manage the communication between the associated smart objects and the network in an optimized way that takes into account both the characteristics of the smart objects and the preferences of the users. The massive generated data by all these connected smart objects and stored in the system will enable the cognitive network, through deep learning algorithms, to raise dynamic procedures that will enable a multi-criteria optimizations of the  behavior  of these smart objects.\\

 Extending the above scenario, one may consider an evolution along spatio-temporal dimension. For instance, during a lunch-break, coffee break at the Jack decides to lower by 5F the set temperature of his home heating system – to activate when he is less than 1 mile from his home and in any case no later than 6PM. However, if the outside temperature drops below 55F, the home temperature must go up by 5 degrees. A colleague gives Jack a list of instructions corresponding to a new method for washing wool sweaters; he  sends the file directly to his home through the social network where a service will adapt the instructions to his washing machine’s particular technical features and his personal preferences (e.g., extra soft cloths). % Back at home, Jack launches “my angry day” profile through  smartphone, which automatically dims the light and has the Hi-Fi play his favorite relaxing music. Meanwhile, 
  At home a tip from the network tells Jack to turn off all TVs to avoid getting further depressed. During the night the washing machine starts the cycle at the time the electric company informs it to be the most favorable.

The above describes a single user interacting seamlessly with multiple devices. We observe that one could attempt to generalize the existing approaches for Social Network of Things~\cite{IeraMA15,NittiPCA16} that can be generated/organized and dissolved on-the-fly, and enable scalability.
 %where different criteria can be used to orchestrate the behavior of (groups of) instances subject to multiple constraints; and certain security and privacy concerns are ensured. 
However, even in the context of ``Jack'' scenario, there are certain aspects that span accross heterogeneous domains/types and at different levels of certain (e.g., hierarchical) relationships. For instance, collaboratively regulating the temperature with other units in Jaack’s building has one kind of spatial, temporal, and devices extent, whereas collaboratively deciding about the cycle of the washing machine may go beyond the building -- i.e., optimizng grid-demand within a city-block.
%adjusting the Hi-Fi themes based on the fact that Jack decided to join his co-workers for a happy-hour; etc… 

By establishing a dynamic instantiations of various components/devices comprising Internet of Things (IoT) structures, we aim to  provide an automatic resource optimization and other intelligence into these evolving collections of smart connected objects, going beyond the social networks of things. Rather than endowing them with autonomously cognitive chips, we aim to enable manufacturers, retailers and users/owners, to elaborate finely tuned dispatches to be distributed toward the smart objects through {\it middleware fogs} located various units (e.g., homes).
As part of the proposed research we plan to provide novel data analytics methodologies that will: (a) incorporate the fact that some users may not be cooperative in terms of suggested regime of use for certain devices/task; (b) provide a specific behavioral insights to be used by various manufacturers when planning the design of certain devices so that they can be better adapted to changing conditions of use.
 A high-level description of the architecture we target to develop for supporting such  scenarios is depicted in Figure~\ref{fig1}, providing methodologies and tools for two basic modes: \\

\begin{wrapfigure}{R}{0.60\textwidth} \vspace{-3mm}
	\centerline{\includegraphics[width=0.60\textwidth]{./fig1.png}}
	\vspace{-3mm} \caption{\small Depiction of the proposed system}
	\label{fig1}
	\vspace{-3mm}
\end{wrapfigure}

(I)  The first setup consists of a set of middleware fogs, located in each smart box/gateway of a smart home or public area, grouped together in an evolving (social-like) network along with profiles of manufacturers, retailers and users/owners. The lower layer is formed by all the real devices/objects (things), such as a fridge, a washing machine, a heater, a TV, a Hi-Fi, etc., where each one is abstracted by what we refer to as WiThing (WiT) for Wireless Thing. The WiT is the first level of the device abstraction and its role is to (1) uniquely identify an object/device (2) represent the object in terms of its properties/functionalities through an MIB (Management Information Base), and (3) constitute a bidirectional interface for all communications between the object and the middleware fog located in the smart box/gateway. The intermediate layer contains the smart box/gateway where the middleware fog is implemented. The latter is composed of an ensemble of interconnected modules that will govern all the requests issued by the user through a proper front-end application. To this end, this smart box must interface with any object/device it meets in the surroundings, i.e., any virtual device communicated by the real objects/devices. The upper layer illustrates the complexity of the different couplings to form ensembles of (networks of) manufacturers, retailers and users/owners with the generic representations of their objects/devices. 
%%%It consists of a large database with an enquiry system based on sophisticated algorithms.



\begin{wrapfigure}{R}{0.40\textwidth} \vspace{-3mm}
	\centerline{\includegraphics[width=0.38\textwidth]{./fig2-3.jpg}}
	\vspace{-3mm} \caption{\small Representation of Smart Objects as Avatars and Meta-avatars in the Cloud.}
	\label{fig2}
	\vspace{-3mm}
\end{wrapfigure}

(II) The second setup complements the previous one in terms of inter-contexts couplings and dynamic formations (and dissolution) of corresponding instances of ensembles with heterogeneous participants. In terms of notation in Figure 1. this can be explained by the following consideration:
(a) If we have an instance of a ``network'' of particular devices (e.g., a washers) from a collection of residential units, depending on the geographical context we may want to compose instances of collections corresponding a group of same devices but operating in a group of buildings in a near-downtown area.
(b) Complementary to this, we may need to couple the ``entity'' corresponding to the ensemble of washers in different units in an apartment complex, with the one of entertainment devices in the same and/or neighboring buildings.


One could readily generalize this in multiple contexts -- e.g., the instances representing the traffic-density in different geographical regions may be tied with the crime-rate in those and/or near-by regions and with the ones corresponding to activities of electronic devices in the police cars. %%However, after a certain time of the day/night, such (meta)Avatars may cease to exist.



To enable such situational awareness that is flexible enough to incorporate the dynamic formation of instances that can span not only from networks of users to network of heterogeneous devices, but also through the existence of ensembles comprised of different contextual impacts (space, time, objective to be optimized, etc.) we  propose the concept of Heterogeneity And Context Aware Dynamic Avatars (HACADA) illustrated in Figure~\ref{fig2}. We note that, in addition to the scenario-oriented discussions above, the HACADA's may be created, updated/modified and ceased based on different criteria: (a) particular cyber-properties: e.g., status of network infrastructure, privacy constraints, etc.; (b) Logical or semantical properties which may entail dynamically forming (and dissolving) hierarchies of HACADAs; (c) ``expiration'' of certain criteria (e.g., pat 10PM) enabling the leftover descriptors in two different HACADA-instances to be merged into one.


\noindent {\bf Intellectual Merits:}

Throughout this complex environment, and in order to automate the deployment and control, as well as optimize the use of such ensembles of HACADAs comprising multiple instances of operational Internet of Things,
%%gathering of important data, identification of correlations in this chain of complex events, and triggering the required responses,
a plethora of issues in data collection and processing (storage/retrieval, querying and information fusion), knowledge discovery, and security and privacy need to be investigated in unified framework. We assert that  such integrated large scale effort, involving researchers from diverse backgrounds, is necessary to  enable exploiting the full potential of the Internet of Things technology  and make it ubiquitous of use. The proposed project has the following list of collaborative research activities as its main Intellectual Merits:

{\it Design and development of novel techniques and abstractions for acquisition and representation of data from IoT devices and users} {\bf NOTE: mention HACADA + Mubbasir's discussion}

{\it Development of privacy-preserving analytics to discover hidden correlations and usage patterns and behaviors} {\bf NOTE: mention a summary from the section written by Vladimir and Ashfaq}.

{\it Design and development of network protocols to mitigate the DoS and greedy behavior in IoT devices} {\bf NOTE: Alek and Farid section-summary} 


{\it Development of simulation platform as well as real-world testbed of heterogeneous IoTs (devices and users)}. {\bf NOTE: mention a realization of application scencario(s)}.


\noindent {\bf Broader Impacts}: The salient broader impacts of the proposed project are:

{\bf NOTE: need to expand}

This research effort aims to contribute to the rapid development of national and local capacity to respond to critical events and to develop technology to advance the ability and scope of Internet of Things to detect and monitor events in applications of high priority, such as surveillance and habitat monitoring. The proposed research activity is a multidisciplinary effort involving researchers with expertise in distributed databases, security, and data management and analytics. This work will also contribute to, and complement efforts under way for, the design and management of reliable smart facilities. Leveraging on the existing cooperation between Illinois Institute of Technology, Northwestern University and Rutgers University. The PIs have a long record of working on collaborative research projects and have extensively involved undergraduate students, minority students and women in their research.



The proposed collaborative project will be executed by a team of six highly accomplished researchers and their students from three different institutions, who have expertise in all the complimentary areas relevant to the research activities proposed in this project, including: Human Computer Interface, Machine Learning, Information Systems and Data Sciences, Sensor and Sensor Networking, Security, and Privacy. The PI and Co-PIs have a long and well established record of collaborating on different federal government and industry funded  research projects. In the following we provide a brief introduction of the PIs and their relevant expertise.
\begin{itemize}
\vspace{-3mm} \item Ashfaq Khokhar, Illinois Institute of Technology: Wireless Sensor Networks, Data Mining, Cloud Computing
\vspace{-3mm} \item Farid Nait-Abdesselam, Illinois Institute of Technology: Network Security, Wireless Sensor Networks
\vspace{-3mm} \item Mubbasir Kapadia, Rutgers University: Human Computer Interface, Virtual Human Systems
\vspace{-3mm} \item Vladimir Pavlovic, Rutgers University: Machine Learning, Dynamic System Modeling, Computer Vision
\vspace{-3mm} \item Aleksander Kuzmanovic, Northwestern University: Networks Security, Denial-of-service Resiliency, Content Distribution
\vspace{-3mm} \item Goce Trajcevski, Northwestern University: Moving Objects Databases,Data Management in Sensor Networks
\end{itemize}




\section{State of the art}\label{soa}
We now present an overview of the state of the art, both from the perspective of the current properties of devices and networking technology features, as well as from the perspective of related works in several fields that, in one way or another, may be used to achieve our objectives however, the fall short of several important aspects.\\

\subsection{Whatever Proper title of this sub-section}
The vast majority of devices are endowed with electronic interfaces, made of buttons and the like, inviting users to operate selections from among a limited set of options pre-featured by the manufacturers. They may concern options, such as, ventilation on/off button in an oven, a selection of delicate option in a washing machine, air-condition on/off button in a car, and so on. While these older devices are considered somehow efficient and simple, the trend is towards more technological and ubiquitous devices.
The operational innovation of our approach is to make a move of the interface a step ahead and a step farther so as to realize a deep, remote and smart control. In this case, the manufacturer makes all/most of the operational parameters, such as spin speed and duration, totally manageable and configurable by the user. In turn, the user drives these parameters via software, hence makes them obeying to automatic procedures/instructions that have been elaborated and optimized in advance, with the help of the manufacturer, other contributors, and/or the networked intelligence.
Thus, we need to have this new hardware/software interface managing signals/data back and forth between actuators/sensors and a logic unit with the result of detaching the user from any physical contact with the device (thing). To this end, in the project we will use open source electronic boards, such as Arduino boards (https://www.arduino.cc/), Parallella boards (https://www.parallella.org/) and/or Raspberry-Pi boards (https://www.raspberrypi.org/). They are powerful prototyping boards based on flexible, open and easy-to-use hardware and software. These devices are wirily and/or wirelessly connected to the Internet from which they can be easily visible and accessible. We foresee that their exploitation will lead microelectronic companies to develop specific chips for a massive diffusion of this new technology in a near future.
In order to realize the vision of an Ambient Intelligence in a future network and service environment, heterogeneous wireless sensor and actuator networks have to be integrated into a common framework of global scale and made available to services and applications via universal service interfaces \cite{SSR05}. The goal is to reach a distributed open architecture with interoperability of heterogeneous systems, neutral and easy access, clear layering and resilience. It should provide the necessary network and information management services to enable reliable, secure and accurate interactions with the physical environment.
The idea is to provide an integrated platform that offers unified data access, processing and services on top of existing ubiquitous services of the Internet of Things to integrate heterogeneous sensors/actuators in a uniform way. From an application perspective, a set of basic services encapsulates sensor/actuator network infrastructures hiding the underlying layers with the network communication details and heterogeneous sensor hardware and lower level protocols.\\

A heterogeneous networking environment indeed calls for means to hide the complexity from the end-user, as well as applications, by providing intelligent and adaptable connectivity services, thus providing an efficient application development framework. To face the coexistence of many heterogeneous set of things, a common trend in Internet of Things applications is the adoption of an abstraction layer capable of harmonizing the access to different devices with a common language and procedures \cite{ERA09}. Standard interfaces and data models ensure a high degree of interoperability among multiple systems. However, typical drawbacks of misconfigurations and traffic congestions are normally exasperated by the node heterogeneity. These drawbacks will be overcome in the project through the adoption of islands architecture. On the one hand, smart gateways in their locations will hide all the complexities of the underlying standards. Hence, wireless connections of the devices/machines to the web/Internet, using communication technologies like Zigbee \cite{Zigbee12}, Z-wave \cite{Zwave12}, Wi-Fi \cite{Wifi}, plus the mapping of every device to a unique ID, will provide full, intelligent and secure control over it. With these smart boxes/gateways, technological limitations will completely disappear and the devices will become identifiable only through their functionalities with clear and consistent APIs. On the other hand, the middleware fog constitutes an abstraction level where all the devices, viewed as entities sharing their functionalities (avatars), are transparently managed and used. The goal is to manage the collaboration between heterogeneous devices through a simple API level in conjunction with the mentioned communication protocol able to reach the peer within the location singularly.\\

Most existing solutions for middleware adopt in general a service-oriented design tailored mainly to support a network topology of sensors that is both unknown and dynamic. But while some projects focus on abstracting the sensors in the network as services (such as in HYDRA \cite{ZH09, ERA10, ZH08}, SENSEI \cite{PBEV09}, SOCRADES \cite{GTKSS10}, and COBIS \cite{cobis07}), other projects devote more attention to data/information abstractions and their integrations with services (among which are SOFIA \cite{HLBT10}, SATware \cite{MDMV09}, and Global Sensor Networks GSN \cite{AHS07}). A common thread throughout all of these solutions, however, is that they handle the challenge of an unknown topology through the use of discovery methods that are largely based on the well-known traditional service/resource discovery approaches of the existing Internet, ubiquitous environments and wireless sensor and actuator networks \cite{AM08, ZMN05, MRPM08}. For instance, SOCRADES provides discovery on two levels, the sensor level and the service level, which can employ either standard web-service discovery or a RESTful discovery mechanism (for RESTful services). COBIS, on the other hand, uses its own service description language COBIL 2 (Collaborative Business Item Language), where service functions and keywords are annotated with a verbal description.
Another point of agreement in the state-of-the-art of middleware solutions is in the widespread use of semantics and metadata to overcome heterogeneity challenges. Indeed, it is standard practice to use ontologies to model sensors, their domains, and sensor data repositories  \cite{ERA10, ELS07, LZ05}. Some projects even go a step further and also include context information \cite{PBEV09}, or service descriptions \cite{ZH09, ERA10, ZH08}. And as a type of service composition, many projects support the concept of virtual/semantic sensors (for instance, in HYDRA, GSN and SATware), i.e. entities that abstract several aggregated physical devices under a single service. A different implementation of a similar idea, though, is provided in the SATware project, where virtual sensors actually correspond to transformations applied to a set of raw sensor streams to produce another semantically meaningful stream.\\

Regarding scalability, most projects address this challenge by pursuing modifications in the underlying sensor/actuator network topology. Sometimes, this is done by adopting fully-distributed infrastructures (such as in COBIS and SOFIA), and sometimes through an architecture of peer-to-peer clusters (e.g., GSN). In our view, however, while these approaches work well for the existing Internet, where traffic is made up of a relatively small amount of service interactions, they will not fit for the complex weave of interactions that will be commonplace in the Internet of Things. In the Internet of Things, a large number of requests will involve intricate coordination among millions of things and services, whereas on today's Internet most requests are largely point-to-point. Therefore, the number of packets transmitted in the network will grow strongly and nonlinearly as the number of available services increases. In such an environment, performing even a simple service discovery may exceed acceptable time, processing, and memory constraints. //NOTE: the last paragraph needs to be tied (cf. (2)/Fig.2 in the intro with the problems related to ''Possible Worlds Semantics'' and the efficient pruning the needs to be performed

\subsection{Data and Process Integration}
Heterogeneous data integration is a topic which, in addition to the database community broadly \cite{Cohen98,HalevyRO06}, has also been investigated by more specialized research foci: -- bioinformatics \cite{KirstenR06,MostafaviM10}; information management and text retrieval \cite{CouletGDAMS11,HammerGIPUW95,Torlone08}; sensing/actuation and tracking \cite{AvciTS16,ChawatheKRS04,SalarianCN12} (to name but a few). While the role and the impact of the semantics have been recognized and addressed \cite{BergamaschiCVB01,CastanoFMR04}, the existing results still have limited cross-contextual support. For instance, the works investigating the benefits of semantics for resource discovery (cf. \cite{CastanoFMR04}) do not take into consideration the role of different devices that may enter and/or leave a particular ''geo-social'' network dynamically, as well as their impact on real-time adjustments (as well as the impact of their cessation).
To-Do -- Goce:
Overview of:

a.	Data integration + distributed query processing; workflows~\cite{LiCLWPZB12,PandeyB12,PoolaRB16}; Wireless Actuator Networks

b.	Analytics and platforms in-context (Cloud/Hadoop; NoSQL; Warehousing)~\cite{ToosiCB14}


\section{Proposed research overview}\label{research-overview}

We now proceed with identifying the main challenges that the proposed project will address, with a brief overview of the proposed research tasks (research challenges), followed by an introduction of the team and their expertise.\\

Research Challenge 1 (RC-1): Data collection and processing\\
…\\
\\

Research Challenge 2 (RC-2): Knowledge discovery\\
…\\
\\

Research Challenge 3 (RC-3): Security, privacy (anonymity) and trust\\

The Internet of Things is foreseen to bring a multitude of services with a vision of creating a smart self-configuring and interconnected world for the benefit of end users. With the extensive research and development of computer, communication and control technologies, it is possible to connect all things to the Internet such that the so-called Internet of Things (IoT) can be formed. These things may be equipped with devices such as sensors, actuators, and tags, in order to allow people and things to be connected anytime and anywhere, with anything and anyone. IoT will enable collaborations and communications among people and things, and among things themselves, which expand the current Internet and will radically change our personal, corporate, and community environments. However, a plethora of security, privacy, and trust challenges need to be addressed in order to fully realize this vision [20, 21, 22, 23]. When more and more things connect to the Internet, security and privacy issues become more serious, especially in the case that these things are equipped with actuators and can support control.
For better protection of secure communication and user privacy, including location, identity and behavior habits, it is necessary to develop anonymous communication theories, methods and key technologies of anonymous communication systems in all varieties of application environments. Anonymous communication is used to hide communication participants or communication relations to achieve effective protection for network nodes and user identities. Anonymous communication can address potential network security issues, and becomes one of the hot topics in the field of network and information security. With reference to security, data anonymity, confidentiality, and integrity need to be guaranteed, along with providing authentication and authorization mechanisms in order to prevent unauthorized users (i.e., humans and devices) to access any system. Concerning privacy requirement, both data and user personal information have to be confidentially manipulated since devices may manage sensitive information (e.g., user habits, locations, etc). Finally, trust is also a fundamental issue since the IoT environment is characterized by multiple devices that have to process and handle the data in compliance with user needs and rights.\\

To realize a trusted, secure, and privacy preserving social network of things, we plan to address in this proposal the following security, privacy, and trust management related tasks:\\

T-3.1: Define a new scheme of anonymizing things and representing them as avatars in the social network of things.\\

T-3.2: Authentication of data collected in fogs in the presence of malicious attackers, despite the attack surface being very broad, ranging from PHY layer to application layer.\\

T-3.3: Achieving end-to-end secure and privacy-preserving information flow monitoring between users and things, covering all sort of things and access networks using physiological and physical layer propertie.\\

T-4.3: Design of secure, fast and friendly authentication schemes to allow users of the social network of things to access the data of interest through diverse mobile devices such as smartphones and tablets.\\

\section{Research tasks}\label{research-tasks}
We now proceed with a detailed presentation of the main research of the proposed project. 


\subsection{Data collection and processing} 
(Goce + Mubbasir)\\
Data cleaning, data collection, sharing, processing, analytics\\
How can we secure the collected data and how to make it available?\\
APIs and libraries for building applications 


Advances in global positioning technologies
(GPS)~\cite{MannuciA04} enabled a fusion of spatial~\cite{pelekis-r2,HjaltSamet99,SpatialShashi03}
and temporal~\cite{TemporalBook} databases, extending them
to the field of spatio-temporal and mobile data
management~\cite{MobiEyes2,IndexJensen06,OurGeoInf07,MokbelA08}. The main relevance for the 
proposed project is that the popular query categories 
are \textit{continuous} (i.e., their answers may have to be
re-evaluated based on the changes in the motion of the entities);
and/or \textit{persistent} (i.e., their answers may need to be
re-evaluated based both on the changes of the motion as well as
the history of such
changes)~\cite{OurGeoInf07}. From a complementary perspective, research works in energy-efficient 
tracking~\cite{AvciTS16,LiuS11,RenLC11} and query processing~\cite{MaddenTinyDB06} in Wireless Sensor Networks (WSN)
brought about the concepts of distributed/localized processing (i.e., minimizing the communication) along with the selection of
notes in-charge of a particular data-gathering and processing -- e.g., {\it tracking principals}~\cite{GhicaTZ10}, {\it cluster-heads}~\cite{Kulik99,Tavli06}, etc.

In the settings focal to the proposed project and management of evolving HACADAs, both bodies of existing works -- for which the co-PIs have research contributions and experience~\cite{AvciTS16,OurGeoInf07,GhicaTZ10,MyTODS04,ZhouT+12} -- can be used as leverage, however, there are quite a few additional unique challenges that we need to address.

\paragraph{Operands, Operations and Placements}

We assume that, at minimum, each WiT object will have the attributes described in Table~\ref{tab:table1}:

\begin{table}[h!]
  \centering
  \caption{WiT structure}
  \label{tab:table1}
  \begin{tabular}{l|l|l|l}
    IP\_Id & [(Value$_i$, Description$_i$)] & Location & Other Description\\
    \hline
    Unique IP & Types of values (sensed, transmitted, descriptive) & geo-location & Semantics; Operational Mode; etc.\\
  \end{tabular}
\end{table}

At a first approximation, a particular HACADA can be (logically) considered as a triplet $H_i =${\it ([WiT]$_i$, R(WiT$_i$,Wit$_j$), [$C_g$])}, where {\it [WiT]$_i$} denotes the list of its WiT constituents/objects; {\it R(WiT$_{ij}$,Wit$_{ik}$)} encodes a (possible) relationship  between the $j$-th and $k$-th WiT; and {\it [$C_g$]} denotes a (possible) list of {\it global constraints} such as {\it C1} introduced in Sec.~\ref{dat-cp} or other kinds, limiting the number of possible transitions among states.

One of the primary tasks of the proposed work will be to {\it define a set of operations} over the universe of HACADAs, along with efficient algorithms for their processing. But few examples:

\noindent $\bullet$ {\it merge($H_i$,$H_j$)} -- this operand will merge the representations $H_i$ and $H_j$ into a single HACADA $H_j$. The need to execute this operation may be due to optimization of various analytics tasks (e.g., detecting association rules).

\noindent $\bullet$ {\it split(H,$H_i$,$H_j$,A)} -- this operand will split an existing instance of a HACADA ($H$) into two new instances, based on a criterion pertaining to attribute $A$.

Even the two simple examples above have a lot of inherent complexities. Firstly, one may observe that different overloadings are possible: for instance, {\it merge($H_i$,$H_j$)} can also have a signature {\it merge($H_i$,$H_j$, $H_k$)}, indicating that a brand new HACADA instance is to be created -- and some of the traditional techniques and approaches (cf.~\cite{GoguenM92,JouannaudKKM92}) may need to be revisited for incorporating them in the IoT context~\cite{Adaikkalavan05,snoop11}. Secondly, context-based policies will need to be considered which, in turn, may affect the implementation: for example, should any of the input-operands continue to exist as an independent instance. Thirdly, as much as one can attempt to rely on existing techniques for integrating heterogeneous data~\cite{BergamaschiCVB01,Cohen98,CouletGDAMS11,HalevyRO06,SalarianCN12} -- how can different networking and/or security requirements be brought ``in concert'' with the specification of the declarative and/or procedural semantics of the operators?

An important question in its own right for this part of the proposed research is how can one selectively {\it place the operands and operations' execution} for the purpose of efficient execution of particular operations {\it and} how are the outputs of the operations to be placed, having in mind not only representation-related issues, but also the possibility of analytics and/or security based requirements. There are plethora of works from distributed query processing through processing aggregates in sensor networks, to resources re-allocation~\cite{DewanSHH94,GuoPZA14,JinC06,LiuOBC12,Nisheeth04,Synopsis} to leverage upon -- however, one cannot expect that a straightforward adaptation and/or extension of the existing techniques can yield good performance, especially in dynamic scenarios of multiple WiTs from completely heterogeneous sources entering and exiting the working-context. We will attempt to couple the existing works with our recent results on on-demand resource guidance in mobile sensor networks and detection of motion trends~\cite{AvciTTSZ16,MohamedKT15} in order to ensure that the processing of the operands of interests, as well as evaluation of queries/predicates of interest is done in an optimized and balanced manner~\cite{omCom16-1}.




\paragraph{Uncertainty and Data Compression in Evolving Contexts}

%%of
%%spatio-temporal queries: \textit{range, (k) Nearest-Neighbor
%%((k-)NN)}~\cite{MyTODS04,XiongMA05,YuPK05} are typically: (1)
%%\textit{continuous} (i.e., their answers may have to be
%%re-evaluated based on the changes in the motion of the entities);
%%and/or (2) \textit{persistent} (i.e., their answers may need to be
%%re-evaluated based both on the changes of the motion as well as
%%the history of such
%%changes)~\cite{MobiEyes2,MokbelA08,OurGeoInf07}. 

One of the challenges to be addressed in the proposed project is how to properly incorporate the uncertainty in all the aspects of the query/operations processing. Its sources can be plentiful -- from errors in the values sensed, to errors due to attempting to represent a continuous phenomenon with a discrete samples and use of interpolation in-between~\cite{DevendranL14,GoodchildZK09,HunterG96}. A specific source of uncertainty is the quest for a compactness in the representation -- i.e., data compression, which is sometimes essential (like in the settings of streaming data)~\cite{CormodeMYZ12,CormodeGHJ12}. 
However, unless it is properly captured in the very syntax of the predicates and operators, as well as in the processing algorithms -- its impact can be unpredictable~\cite{MyTODS04}. Throughout the proposed work, we will capitalize on the works coupling aggregation and compression in WSNs~\cite{LinGKL05,KadayifK04,PattemKG08} and our recent results on fusing uncertain data from heterogeneous sources~\cite{ZhangTL16} to formalize the representation of the uncertainty when managing the instances of HACADAs and the execution of the novel operands.

An important component of our research will be how to pro-actively steer the collaborative orchestration of the processes of data generation, compression, (re)placement -- along with queries processing and actuation. We will leverage on our works on proactive management of resources in WSNs~\cite{McClurgTY12} and attempt to apply the concept of evolving triggers~\cite{old-r35} to balance the impact of the (bounded) uncertainty on the quality of the service/experience. This part of the research will be coupled with the challenges addressed in the context of detecting the placement of the predicates whenever multiple levels of granularity may need to be maintained about a data of evolving nature and with semantic annotations~\cite{VaismanZ09,TrajcevskiDVAZT15}.



\paragraph{Knowledge Representation and Reasoning}

The processed data from above will be represented as a set of discrete time series, or \emph{fluents} which encode the states and properties of different \avatar which change over time. \todo{give example}. To facilitate efficient reasoning and knowledge discovery in a dynamic smart and inter-connected system of avatars, we will consider a level of abstraction, termed as \emph{events }. Events constitute meaningful interactions between two or more \avatar and serve as the basis for identifying relationships between them towards creating social networks of avatars. \todo{give example}. We define the specific terms and concepts we will use for event-centric knowledge representation and reasoning below.

\noindent \textbf{Smart Avatars} The notion of smart objects~\cite{Kallmann:1999:DIS:323663.323683} has been popularly used in the graphics and animation community to embed intelligence and semantics in virtual objects. We extend this formalism to represent both IoT objects as well as IoT users. This unified formalism allows us to seamless consider sensors, controllers, actuators, and human users within the same social community.  We define a smart avatar $\smartobj{} \in \world$ as $\smartobj{} = \langle \affordances, \state{} \rangle$ with a set of advertised affordances $\affordances$ and a state \state{}. An affordance $\affordance(\affordanceowner, \affordanceuser) \in \affordances$ is an advertised capability offered by a smart avatar that manipulates the states of the owner of an affordance $\affordanceowner$ and a smart avatar user $\affordanceuser$.

\noindent \textbf{State.} The state $\state{} = \langle \attributeSymbol, \relationshipSymbol \rangle$ of a smart object $\smartobj{}$ comprises a set of attribute mappings \attributeSymbol (fluents), and a collection of pairwise relationships $\relationshipSymbol$ with other avatars. With this representation, we can make logical inferences between objects using a declarative PROLOG-like knowledge reasoning engine.
%An attribute \attributemapping{i}{j} is a bit that denotes the value of the $j^{th}$ attribute for $\smartobj{i}$.



%A specific relationship $\relationship{\cdot}{\cdot}{a}$ is a sparse matrix of $|\world| \times |\world|$, where \relationship{i}{j}{a} is a bit that denotes the current value of the $a^{th}$ relationship between \smartobj{i} and \smartobj{j}.


%\noindent \textbf{Rules.}  A rule $\Rule_x(i,j) \in \Rules$ between two smart objects \smartobj{i}, \smartobj{j} is true or false, depending on the states and relationships of both objects. Rules allow for logical inference between objects and are used for reasoning such as evaluating whether a character can access a particular room, or manipulate another smart object based on the current world state. They are defined and solved using a declarative PROLOG-like interface

\noindent \textbf{Events.} Events are used to encode context-specific interactions between two or more smart avatars, and provide an appropriate level of abstraction for knowledge discovery  An event is formally defined as $\event{} = \langle \tree,  \precondition{}, \postcondition{} \rangle$. A precondition $\precondition{}: \groupState{\smartObjectGroup{}} \leftarrow \{\true, \false \} $ is a logical expression on the compound state \groupState{\smartObjectGroup{}} of a particular set of smart avatars $\smartObjectGroup{}: \{ \smartobj{1}, \smartobj{2}, \ldots \smartobj{|\roleSet{}|} \}$ that checks the validity of the states of each smart object. \precondition{} is represented as a conjunction of clauses $\clause{}{} \in \precondition{}$ where each clause \clause{}{} is a literal that specifies the desired attributes of smart objects, and relationships between pairs of participants. A precondition is fulfilled by $\smartObjectGroup{} \subseteq \world $ if $\preconditionFunction{\event}{\smartObjectGroup{}} = \true$. The event postcondition $\postcondition{} : \groupState{} \rightarrow \nextGroupState{}$ transforms the current state of all event participants \groupState{} to \nextGroupState{} by executing the effects of the event. We can extend this definition to model non-deterministic, fuzzy events with a probabilistic notion of success or failure. Events may optionally have a controller which defines the series of affordance activations within the smart avatars to produce its desired outcome. We represent this control logic using an extended version of Behavior Trees that facilitate parameterization. Parameterized Behavior Tree (PBT)~\cite{Shoulson:2011:PBT:2177817.2177835} are an effective model for representing coordinated control logic  between multiple smart avatars.

PI Kapadia has extensive prior experience in developing event-centric knowledge bases for inference and reasoning in virtual worlds~\cite{Shoulson:2013:EPA:2522628.2522629,2015-fdg-bstl,Kapadia:2015:CAI:2699276.2699279,Kapadia:2016:CCN:2982818.2982846} which will be extended to represent the IoT domain.


%ADAPT~\cite{Shoulson:2013:EPA:2522628.2522629}
%BSTL~\cite{2015-fdg-bstl,Kapadia:2015:CAI:2699276.2699279}
%behavior authoring~\cite{behaviorCGA,Kapadia:2011:BAC:1944745.1944779}
%story world~\cite{AIIDE1511583}
%CANVAS~\cite{Kapadia:2016:CCN:2982818.2982846}





%$\roleSet{} = \{ \role{i} \}$  define the desired roles for each participant. \role{i} is a logical formula specifying the desired value of the immutable attributes \attributemapping{\cdot}{j} for \smartobj{j} to be considered as a valid candidate for that particular role in the event.


%An event instance $\instance = \langle \event, \smartObjectGroup{} \rangle$ is an event \event populated with an ordered list of smart object participants \smartObjectGroup{}.  $\preconditionFunction{\event}{\smartObjectGroup{}} = \true$. The event postcondition $\postcondition{} : \groupState{} \rightarrow \nextGroupState{}$ transforms the current state of all event participants \groupState{} to \nextGroupState{} by executing the effects of the event. When an event fails, $\nextGroupState{} = \groupState{}$. An event instance $\instance = \langle \event, \smartObjectGroup{} \rangle$ is an event \event populated with an ordered list of smart object participants \smartObjectGroup{}.

%? which represent the specific values of avatar fluents at specific points. Events occur on a set of avatars which are the participants, and produce a change in state of these avatars which are defined as event postconditions. Events may additionally be pre-conditioned on the participant states to indicate what conditions must be satisfied in order for an event to successfully execute.

This symbolic representation of avatars and their dynamic states in terms of key events will facilitate the development of an event calculus ? allowing us make inferences about the relationships and properties of these avatars using first and second-order logic. This reasoning will be used as the basis for knowledge discovery described in Section XX. Knowledge discovery will entail the identification of relationships between the properties of different avatars, clustering of avatar to create compound entities, and the discovery of salient events.


 

\subsection{Knowledge discovery: social network of avatars (virtual world)}
(Ashfaq + Vladamir)\\
….\\
…\\

\subsection{Security, privacy and anonymity}
(Farid + Alex)\\
…\\
Data security\\
System level security\\
Privacy, anonymity\\
\\
Greedy behavior (?)\\
Key sharing problem\\ 
Privacy, trust and anonymity \\

\subsection{Implementation and validation }
(All)\\
\\
Letters of Collaboration)\\
Campus wide implementation \\
Letters of Commitment\\
Shall we rely on KAA?\\
APIs: flexibility of formats/schemas\\


\section{Broader Impact}\label{broader-impact}
From Rutgers:

The proposed highly interdisciplinary project combines computer vision, human modelling and simulation, optimization, robotics and autonomy, and control theory, and has the potential to impact several research areas. The proposed sensing, simulation, and optimization framework will have wide applicability in various scientific areas involving decision-making in heterogeneous and dynamic networks comprising human-operated, and autonomous sensors, with concepts that could easily generalize to event understanding in complex social, economical or cyber-biological systems, thus impacting multiple societal applications including urban infrastructure, emergency response, safety, and quality of life.

\noindent \textbf{Datasets.} We will release the reconstructed time-stamped behaviors that are collected as part of our real-world experiments in the classroom and the dining hall. This real-world dataset will include both the original, unprocessed, noisy, incomplete trajectories, as well as the processed trajectories for researchers to use and compare, when developing their own algorithms. We will also release synthetic datasets from our simulation experiments using the data-driven crowd simulator that will be developed as part of this research. All crowd datasets (both real and synthetic) will be captured for both the original, un-optimized environments, as well as the optimized layouts. We will also release extended designs actuated of environmental elements such as mobile queue separators and mobile furniture.

\noindent \textbf{Open-Source Software.} The PIs have an established record of releasing software packages related to crowd modeling and simulation ~\cite{steersuite,10.1109/TVCG.2014.251}. We will build on top of these foundations and extend our existing open-source platforms to include software solutions for: (1) crowd trajectory reconstruction, (2) data-driven crowd simulation, (3) static and dynamic analysis of environments, (4) crowd-aware environment optimization, and (5) environment reconfiguration planning.

\noindent \textbf{Kaggle Competitions}. We plan to organize 2 public data analytics competitions using the collected data, hosted on kaggle.com. Kaggle is a platform for predictive modelling and analytics competitions, which is often used in academia to incentivize students to tackle challenging problems through a combination of research, system building, and peer-competitions. The goals will include trajectory estimation and reconstruction, data-driven crowd modelling, and computer-assisted designs of environments.

\noindent \textbf{Curriculum Development and Outreach.} The PIs will generate educational material on the autonomy, hardware design, sensing and simulation aspects. PI Kapadia will design a graduate seminar on \emph{Crowd-Aware Smart Environments} that will introduce students to concepts in crowd simulation, and environment optimization. Advanced undergraduate students will also be permitted to enroll. PI Pavlovic will introduce an assignment in the machine learning course where students will learn generative models of crowd movement. The PI's will collectively organize a workshop on crowd-aware cyber-physical environments during CPS week (tentatively April 2017), which will bring together researchers from various disciplines including computer vision, machine learning, simulation, robotics, and potential adopters of the technology. Our research environment will provide interdisciplinary training ranging from distributed sensing and control, simulation and optimization, to robot motion planning.

\noindent \textbf{Under-Represented Groups and K-12 Level}. Our project can help attract a diverse group of students and broaden the diversity of students recruited into these and other STEM disciplines. Rutgers office of Enrollment Management conducts the Rutgers Future Scholars Program (RFS), which provides mentoring activities to grade 9-12 minority students from disadvantaged backgrounds and full scholarships to undergraduate programs. We will closely work with this office to participate in outreach, enrichment, and mentoring activities. We will also leverage other programs to recruit and mentor students from under-represented groups: the summer undergraduate Project RiSE (Research in Science and Engineering) for undergraduates from underrepresented populations for 8/1-week intensive summer research internships; and the Aresty Center for Rutgers Undergraduate Research- places Rutgers undergraduates in research laboratories on campus. PI Pavlovic has a multi-year track record of working with over 14 minority and female students through Aresty and RiSE, as well as multiple REUs. We will leverage the extensive diversity programs at Rutgers to recruit and support women and underrepresented minorities.


\section{Results from current and recent prior NSF support}\label{prior-projects}


{\bf Aleksandar Kuzmanovic} has done extensive work in 
congestion control \cite{ecn05sigcomm,ecn05rfc,web06par,tcp-lp,tcp-lpToN,hstcp-lp,extr08ccr},
DoS resiliency \cite{receiver07jcn,friendly07infocom,dos05p2p,cache06icnp,tcpoison07icnp,shrew03sigcomm,shrewToN,aj08imc,adver10infocom},
measurements \cite{akamai06sigcomm,pong07sigmetrics,pong08icnp,KuzKni01,kuzkni-tpds,akamai08icdcs,aj08imc,google08sigcomm,serendipity09imc,p2p10infocom,geo11nsdi}, and Web \cite{geo11nsdi,migration11wi,spam13pam,selective12wi,adver11icdcs,myths10wi,ad10www,ad10infocom,myths14tweb,synthoid14wi,geoecho14wi,mosaic13sigcomm,glance13wow,fusion11,fusion13,move12www,vision12}.
He is currently the PI of CNS-1319086, \$473,445, 8/30/13 through 8/29/17, \emph{NetS: Small: Endpoint User Profile Control}. The \textbf{Intellectual Merit} of this proposal lies in developing auditing mechanisms on the web, and the key
\textbf{Broader Impact} of this proposal is the deployment of a set of first such tools for the Web; the list of publications resulted from this project is \cite{spam13pam,migration11wi,selective12wi,adver11icdcs,myths10wi,ad10www,ad10infocom,myths14tweb}, and the list of research products from this project is available at \url{http://networks.cs.northwestern.edu/audit-content/}.

...\\

...\\


\section{Data management plan}\label{data-management-plan}
\section{Data Management Plan}

\subsection{Types of Data} 
The proposed research will generate the following material and data that might be of use to the general scientific community: 
\begin{itemize}
\item Open-source software, either in the form of stand-along packages or modules for existing packages, such as the following: 
	SteerSuite, developed by Co-PI Mubbasir Kapadia, which is an open framework for developing, evaluating and sharing steering algorithms,
\item Robot Operating System (ROS), a popular middleware for communicating with robotic devices and sensors, 
\item Open Motion Planning Library (OMPL), a software package developed by Rice University for sharing motion planning algorithms 
\end{itemize}

Data collected from experiments evaluating different approaches for sensing and modeling heterogeneous cyber-human IoTs, as well as for the control of IoTs. The experiments will correspond to simulated challenges, using models of real systems and crowds, as well as physical experiments in office and classroom environments given the platforms developed as part of this project. The data will correspond to the available information to the overall system (e.g., the crowd’s configuration), and the choices made by different algorithms. 

\subsection{Standards and Formats} 
The input files and software developed will be compatible with popular protocols in the research community, such as those followed by the ROS software package. For instance, the control and planning methods will be decomposed into multiple processes that will communicate using the messaging infrastructure of ROS. This will also allow the processes to be compatible with existing and future packages developed by different research teams that follow a similar protocol and use the same middleware. 
The format for the data will be documented and provided in parallel with the data. An effort will be made to follow standards in the motion planning community (e.g., current format used by OMPL) and it is expected that the data will be stored in YAML files that will be compatible with ROS’ parameter server and easily readable by people. 

\subsection{Provisions for Archiving and Preservation}
The electronic data will be stored on multiple workstations and notebooks used by the PIs and the graduate students involved in the project. They will be backed up through the use of versioning control systems (Mercurial, Subversion) on the servers of the Computer Science Department at Rutgers University. These servers are backed up regularly off-site. The software modules will be also stored externally at online repositories, such as BitBucket. 

No additional funding or special institutional commitment or special preparation for long-term preservation will be required for the preservation of the data as described above. 

\subsection{Access and Re-Use Policies and Provisions}
The research groups websites’ will provide access to data as they will become available, as well as the software modules that will be developed, documented and prepared for sharing. Sharing of data and software modules will take place no longer than a year after their generation. If requested, access to data or software will be provided earlier via contact with the PIs. 

The software will be using the BSD license, an open-source software license. BSD is a permissive free soft- ware license, which introduces only minimal requirements about how the software can be redistributed. 

There is no anticipation of significant ethical or intellectual property or confidentiality or copyright issues involved with the acquisition of the data. In the event that discoveries or inventions are made in direct connection with this data, access to the data will be granted upon request once appropriate invention disclosures and/or provisional patent filings are made. 

\subsection{Plans for Transition or Termination of the Data Collection}
The data and software modules will be preserved for at least three years beyond the award period, as required by NSF guidelines. The Principal Investigator, assisted by the Information Technology staff of the Computer Science Department at Rutgers University, will be responsible for managing the data beyond the expiration of the research project. No special preparation is needed for migrating, deleting or transitioning the data to another media into the future. 

The data acquired and preserved in the context of this proposal will be further governed by Rutgers University’s policies pertaining to intellectual property, record retention, and data management. 

\section{Project management plan including timeline}\label{project-management-plan}

The collaboration efforts and the management structure are organized around the objectives of the project, the outcome assessment, and research team expertise. The collaborative effort will extend beyond the scientific research to teaming on testbed development, student supervision and training, and educational activity.

\subsection{Research team members and their expertise}
The research project pools diverse expertise from Illinois Institute of Technology (IIT), Northwestern University, and Rutgers University needed for the successful completion of the proposed interdisciplinary research activity. The project team has a prior history of successful collaborations. The team members and their complementary research expertise are as follows:\\

\textbf{Ashfaq Khokhar (AAK), IIT}, (Routing and MAC Layer in WSN, Data Security, Privacy Preserving Data Mining, Power Efficient Scalable Computing),  

\textbf{Farid Nait-Abdesselam (FN), IIT}, (Wireless Networks, Protocol Security, Data Analytics), 

\textbf{Mubbasir Kapadia (MK), Rutgers University}, (Machine Learning, Computer Vision), 

\textbf{Vladimir Pavlovic (VP), Rutgers University}, (xxx), 

\textbf{Aleksander Kuzamanovic (AK), Northwestern University} (xxx),

\textbf{Goce Trajcevski (GT), Northwestern University}, (Moving Objects Databases, Uncertainty in Mobile Data, Active Databases and Triggers). \\

The PIs involved in this project have a history of research collaboration on jointly funded projects. Dr. Khokhar, Dr. Trajcevski, and Dr. Nait-Abdesselam (then an international partner from France) have collaborated on a Large NeTS-NSF award, have co-advised PhD students, and have co-authored several conference and journal publications. 
\colorbox{red}{ADD OTHERS HERE…}\\
The collaborative effort will extend beyond the scientific research to teaming on testbed development, student supervision and training, course development, workshop organization, and performance evaluation of the technology developed.

\subsection{PI research collaboration} 
If the team awarded the funding, a kick-off meeting will be held within a week of the announcement for all investigators, participating senior research personnel, and graduates students. A broad overview of the research activities will be presented and yearly research targets and deliverables will be set. We will hold biweekly meetings with the attendance of the PIs and Ph.D students from all the campuses. Towards this end we plan to use social media tools such as Google Hangouts and Zoom.

The major research tasks, group interactions, research outcomes, deliverables, and possible applications likely to emerge out of the proposed research are depicted in Figure~\ref{CM-1}. The activities in the four research threads and their linkages are captured in the four rectangular boxes on the left. We will have the following collaborative roles for the PIs in the research activities of the project effort, with the team lead identified in bold:\\

\textbf{Trajcevski}, Kapadia, Pavlovic: Data Processing and Information Modeling 

\textbf{Pavlovic}, Khokhar, Kuzmanovic, Nait-Abdesselam: Data Analytics

\textbf{Kuzmanovic}, Khokhar, Nait Abdesselam: Security and Proivacy

\textbf{Nait-Abdesselam}, Kapadia, Trajcevski: Testbed Implementation and Validation \\

These roles are also identified (with the abbreviations of their first and last names) as white ovals in Figure~\ref{CM-1}. Also, we envision realization of the objectives defined for the research activities will produce three outcome components (i) Efficient IoT Representations, (ii) Privacy Preserving Data Analytics, and (iii) Private Secure Connectivity among heterogeneous devices and users. These in turn will be assembled into the overall deliverable of comprehensive software system and testbed for realizing Dynamic Data Driven Cyber Human Networks consisting of IoT avatars. Research in all key areas will proceed in parallel and overlapping phases, and the planned progress over a three-year timeline is shown in quarters at the bottom in Figure~\ref{CM-1}. 

\begin{wrapfigure}{R}{0.70\textwidth} \vspace{-3mm}
	\centerline{\includegraphics[width=0.61\textwidth]{./Timeline-v1.jpg}}
	\vspace{-3mm} \caption{Relationship of major research tasks, group interactions, research outcomes, deliverables, and possible applications likely to emerge out of the proposed research}
	\label{CM-1}
	\vspace{-3mm}
\end{wrapfigure}

The management structure is organized around the main research themes identified in Section 4.  Prof. Khokhar will assume overall responsibility for coordination of research activity. He will be in regular contact with the other investigators and personnel. He has extensive experience in managing large domestic and international teams and has organized NSF-sponsored workshops. The progress of then project will be monitored by a Project Steering Committee, consisting of one Co-PI from each institution. Three working groups will oversee the research on the main research threads described in the proposal, and will be responsible for assessing the research outcomes. Due to the need for collaborative work on the research activities, each investigator will serve on at least two of the three working groups.


\subsection{PI research collaboration and student supervision} 
Upon the award of the project, a kick-off meeting will be held within a week of the announcement for all investigators, participating senior research personnel, and graduates students. A broad overview of the research activities will be presented and yearly research targets and deliverables will be set. We will hold monthly cross-institutional and bi-weekly intra-institutional meetings with the attendance of the PIs and Ph.D students. 

In the first eight quarters, the proposed research activity will include participation of \textit{one postdoc and three graduate students} supported by the project and several undergraduate students carrying out Honors College activity. During the last eight quarters, the project will support \textit{seven graduate students and a Post-doctoral} researcher. The students will receive research training in diverse emerging topics such as Information abstraction, data analytics, and robust privacy and security of networks and its components. Since the work requires collaboration beyond the immediate field of interest, the students will be co-supervised by faculty from participating institutions and will gain extensive experience in inter-disciplinary research activity.

\subsection{Dissemination plan} 
A project website will be established to disseminate the results and the code developed during the project. We will organize at least one workshops in a leading conference on IoTs promote our research among peers working on related research issues. In addition we will work to organize special issues of journals on these topics. The travel cost budgeted in the proposal is related to these dissemination activities

\subsection{Management of testbed development and technology transfer} 
The protocols and techniques developed in this project will be evaluated using an experimental testbed consisting of a network of IoT devices. This effort will be managed jointly by Dr. Khokhar and Dr. Trajcevski. Furthermore multiple commercial organizations working the domain of smart sensor systems and applications has indicated strong interest in the proposed project. For additional details please we refer to the letters of collaboration uploaded as single copy documents.


%%%\newpage

%%%\mypage{M}

%%%\section{Management Plan and Validation}%%
%%%\input{ManagementPlan-v5.tex}
%%%\label{management}

%%%%%%%%%%%Take a look at Phase Transition Phenomena in Wireless Ad-Hoc Networks

\newpage
\mypage{References }

\bibliographystyle{plain}
\bibliography{goce,farid}
%\bibliography{comp11mc-GT-Comprehensive-2011,cps_raf22,pdinda,CPS2012-nikos-12,WSN22,Mokbel,shekhar-refcs,terveen,mining,vipin-references,minnesota-refcs,Prior-bib}

\end{document}

