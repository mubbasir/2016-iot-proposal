We now present a detailed discussion of the main research topics and approaches to be undertaken by the proposed project.


\subsection{Data collection and processing}\label{data-cp}
%%(Goce + Mubbasir)\\
%%Data cleaning, data collection, sharing, processing, analytics\\
%%How can we secure the collected data and how to make it available?\\
%%APIs and libraries for building applications

<<<<<<< HEAD
To motivate this part of the project, consider the following rule: {\it R1: Whenever more than 60\% of the WiT in an ensemble represented as a single HACADA have attributes from a particular class, generate a separate instance to store their data/description}, along with the following constraint: {\it C1: No more than 100 objects should be stored in a MIB, and the total geo-coverage of their location should not span more than 1 mile radius}. Essentially, $R1$ states that if we have a mix of 60 (descriptions of) home-appliances, when 49 of them happen to be refrigerators, they should have their own HACADA instance. However, if there were already 100 HACADAs stored at a particular MIB then, because of $C1$ we may need to re-assign the physical storage of the corresponding representations. The co-existence of $R1$ and $C1$ has to be {\it entangled}~\cite{GuptaKRBGK11} with the dynamics of various operational entities -- from static (e.g., in-home) devices but of different, heterogeneous types, to entities described with various contexts such as user's location, motion-type and type of a mobile device.
=======
<<<<<<< Updated upstream
>>>>>>> origin/master

Advances in global positioning technologies
(GPS)~\cite{MannuciA04} enabled a fusion of spatial~\cite{pelekis-r2,HjaltSamet99,SpatialShashi03}
and temporal~\cite{TemporalBook} databases, extending them
to the field of spatio-temporal and mobile data
management~\cite{MobiEyes2,IndexJensen06,OurGeoInf07,MokbelA08}. The main relevance for the 
proposed project is that the popular query categories 
are \textit{continuous} (i.e., their answers may have to be
re-evaluated based on the changes in the motion of the entities);
and/or \textit{persistent} (i.e., their answers may need to be
re-evaluated based both on the changes of the motion as well as
the history of such
changes)~\cite{OurGeoInf07}. From a complementary perspective, research works in energy-efficient 
tracking~\cite{AvciTS16,LiuS11,RenLC11} and query processing~\cite{MaddenTinyDB06} in Wireless Sensor Networks (WSN)
brought about the concepts of distributed/localized processing (i.e., minimizing the communication) along with the selection of
notes in-charge of a particular data-gathering and processing -- e.g., {\it tracking principals}~\cite{GhicaTZ10}, {\it cluster-heads}~\cite{Kulik99,Tavli06}, etc.

In the settings focal to the proposed project and management of evolving HACADAs, both bodies of existing works -- for which the co-PIs have research contributions and experience~\cite{AvciTS16,OurGeoInf07,GhicaTZ10,MyTODS04,ZhouT+12} -- can be used as leverage, however, there are quite a few additional unique challenges that we need to address.

\paragraph{Operands, Operations and Placements}

We assume that, at minimum, each WiT object will have the attributes described in Table~\ref{tab:table1}:

\begin{table}[h!]
  \centering
  \caption{WiT structure}
  \label{tab:table1}
  \begin{tabular}{l|l|l|l}
    IP\_Id & [(Value$_i$, Description$_i$)] & Location & Other Description\\
    \hline
    Unique IP & Types of values (sensed, transmitted, descriptive) & geo-location & Semantics; Operational Mode; etc.\\
  \end{tabular}
\end{table}

At a first approximation, a particular HACADA can be (logically) considered as a triplet $H_i =${\it ([WiT]$_i$, R(WiT$_i$,Wit$_j$), [$C_g$])}, where {\it [WiT]$_i$} denotes the list of its WiT constituents/objects; {\it R(WiT$_{ij}$,Wit$_{ik}$)} encodes a (possible) relationship  between the $j$-th and $k$-th WiT; and {\it [$C_g$]} denotes a (possible) list of {\it global constraints} such as {\it C1} introduced in Sec.~\ref{dat-cp} or other kinds, limiting the number of possible transitions among states.

One of the primary tasks of the proposed work will be to {\it define a set of operations} over the universe of HACADAs, along with efficient algorithms for their processing. But few examples:

\noindent $\bullet$ {\it merge($H_i$,$H_j$)} -- this operand will merge the representations $H_i$ and $H_j$ into a single HACADA $H_j$. The need to execute this operation may be due to optimization of various analytics tasks (e.g., detecting association rules).

\noindent $\bullet$ {\it split(H,$H_i$,$H_j$,A)} -- this operand will split an existing instance of a HACADA ($H$) into two new instances, based on a criterion pertaining to attribute $A$.

Even the two simple examples above have a lot of inherent complexities. Firstly, one may observe that different overloadings are possible: for instance, {\it merge($H_i$,$H_j$)} can also have a signature {\it merge($H_i$,$H_j$, $H_k$)}, indicating that a brand new HACADA instance is to be created -- and some of the traditional techniques and approaches (cf.~\cite{GoguenM92,JouannaudKKM92}) may need to be revisited for incorporating them in the IoT context~\cite{Adaikkalavan05,snoop11}. Secondly, context-based policies will need to be considered which, in turn, may affect the implementation: for example, should any of the input-operands continue to exist as an independent instance. Thirdly, as much as one can attempt to rely on existing techniques for integrating heterogeneous data~\cite{BergamaschiCVB01,Cohen98,CouletGDAMS11,HalevyRO06,SalarianCN12} -- how can different networking and/or security requirements be brought ``in concert'' with the specification of the declarative and/or procedural semantics of the operators?

An important question in its own right for this part of the proposed research is how can one selectively {\it place the operands and operations' execution} for the purpose of efficient execution of particular operations {\it and} how are the outputs of the operations to be placed, having in mind not only representation-related issues, but also the possibility of analytics and/or security based requirements. There are plethora of works from distributed query processing through processing aggregates in sensor networks, to resources re-allocation~\cite{DewanSHH94,GuoPZA14,JinC06,LiuOBC12,Nisheeth04,Synopsis} to leverage upon -- however, one cannot expect that a straightforward adaptation and/or extension of the existing techniques can yield good performance, especially in dynamic scenarios of multiple WiTs from completely heterogeneous sources entering and exiting the working-context. We will attempt to couple the existing works with our recent results on on-demand resource guidance in mobile sensor networks and detection of motion trends~\cite{AvciTTSZ16,MohamedKT15} in order to ensure that the processing of the operands of interests, as well as evaluation of queries/predicates of interest is done in an optimized and balanced manner~\cite{omCom16-1}.




\paragraph{Uncertainty and Data Compression in Evolving Contexts}

%%of
%%spatio-temporal queries: \textit{range, (k) Nearest-Neighbor
%%((k-)NN)}~\cite{MyTODS04,XiongMA05,YuPK05} are typically: (1)
%%\textit{continuous} (i.e., their answers may have to be
%%re-evaluated based on the changes in the motion of the entities);
%%and/or (2) \textit{persistent} (i.e., their answers may need to be
%%re-evaluated based both on the changes of the motion as well as
%%the history of such
%%changes)~\cite{MobiEyes2,MokbelA08,OurGeoInf07}. 

One of the challenges to be addressed in the proposed project is how to properly incorporate the uncertainty in all the aspects of the query/operations processing. Its sources can be plentiful -- from errors in the values sensed, to errors due to attempting to represent a continuous phenomenon with a discrete samples and use of interpolation in-between~\cite{DevendranL14,GoodchildZK09,HunterG96}. A specific source of uncertainty is the quest for a compactness in the representation -- i.e., data compression, which is sometimes essential (like in the settings of streaming data)~\cite{CormodeMYZ12,CormodeGHJ12}. 
However, unless it is properly captured in the very syntax of the predicates and operators, as well as in the processing algorithms -- its impact can be unpredictable~\cite{MyTODS04}. Throughout the proposed work, we will capitalize on the works coupling aggregation and compression in WSNs~\cite{LinGKL05,KadayifK04,PattemKG08} and our recent results on fusing uncertain data from heterogeneous sources~\cite{ZhangTL16} to formalize the representation of the uncertainty when managing the instances of HACADAs and the execution of the novel operands.

An important component of our research will be how to pro-actively steer the collaborative orchestration of the processes of data generation, compression, (re)placement -- along with queries processing and actuation. We will leverage on our works on proactive management of resources in WSNs~\cite{McClurgTY12} and attempt to apply the concept of evolving triggers~\cite{old-r35} to balance the impact of the (bounded) uncertainty on the quality of the service/experience. This part of the research will be coupled with the challenges addressed in the context of detecting the placement of the predicates whenever multiple levels of granularity may need to be maintained about a data of evolving nature and with semantic annotations~\cite{VaismanZ09,TrajcevskiDVAZT15}.



\paragraph{Knowledge Representation and Reasoning}

The processed data from above will be represented as a set of discrete time series, or \emph{fluents} which encode the states and properties of different \avatar which change over time. \todo{give example}. To facilitate efficient reasoning and knowledge discovery in a dynamic smart and inter-connected system of avatars, we will consider a level of abstraction, termed as \emph{events }. Events constitute meaningful interactions between two or more \avatar and serve as the basis for identifying relationships between them towards creating social networks of avatars. \todo{give example}. We define the specific terms and concepts we will use for event-centric knowledge representation and reasoning below.

\noindent \textbf{Smart Avatars} The notion of smart objects~\cite{Kallmann:1999:DIS:323663.323683} has been popularly used in the graphics and animation community to embed intelligence and semantics in virtual objects. We extend this formalism to represent both IoT objects as well as IoT users. This unified formalism allows us to seamless consider sensors, controllers, actuators, and human users within the same social community.  We define a smart avatar $\smartobj{} \in \world$ as $\smartobj{} = \langle \affordances, \state{} \rangle$ with a set of advertised affordances $\affordances$ and a state \state{}. An affordance $\affordance(\affordanceowner, \affordanceuser) \in \affordances$ is an advertised capability offered by a smart avatar that manipulates the states of the owner of an affordance $\affordanceowner$ and a smart avatar user $\affordanceuser$.

\noindent \textbf{State.} The state $\state{} = \langle \attributeSymbol, \relationshipSymbol \rangle$ of a smart object $\smartobj{}$ comprises a set of attribute mappings \attributeSymbol (fluents), and a collection of pairwise relationships $\relationshipSymbol$ with other avatars. With this representation, we can make logical inferences between objects using a declarative PROLOG-like knowledge reasoning engine.
%An attribute \attributemapping{i}{j} is a bit that denotes the value of the $j^{th}$ attribute for $\smartobj{i}$.



%A specific relationship $\relationship{\cdot}{\cdot}{a}$ is a sparse matrix of $|\world| \times |\world|$, where \relationship{i}{j}{a} is a bit that denotes the current value of the $a^{th}$ relationship between \smartobj{i} and \smartobj{j}.


%\noindent \textbf{Rules.}  A rule $\Rule_x(i,j) \in \Rules$ between two smart objects \smartobj{i}, \smartobj{j} is true or false, depending on the states and relationships of both objects. Rules allow for logical inference between objects and are used for reasoning such as evaluating whether a character can access a particular room, or manipulate another smart object based on the current world state. They are defined and solved using a declarative PROLOG-like interface

\noindent \textbf{Events.} Events are used to encode context-specific interactions between two or more smart avatars, and provide an appropriate level of abstraction for knowledge discovery  An event is formally defined as $\event{} = \langle \tree,  \precondition{}, \postcondition{} \rangle$. A precondition $\precondition{}: \groupState{\smartObjectGroup{}} \leftarrow \{\true, \false \} $ is a logical expression on the compound state \groupState{\smartObjectGroup{}} of a particular set of smart avatars $\smartObjectGroup{}: \{ \smartobj{1}, \smartobj{2}, \ldots \smartobj{|\roleSet{}|} \}$ that checks the validity of the states of each smart object. \precondition{} is represented as a conjunction of clauses $\clause{}{} \in \precondition{}$ where each clause \clause{}{} is a literal that specifies the desired attributes of smart objects, and relationships between pairs of participants. A precondition is fulfilled by $\smartObjectGroup{} \subseteq \world $ if $\preconditionFunction{\event}{\smartObjectGroup{}} = \true$. The event postcondition $\postcondition{} : \groupState{} \rightarrow \nextGroupState{}$ transforms the current state of all event participants \groupState{} to \nextGroupState{} by executing the effects of the event. We can extend this definition to model non-deterministic, fuzzy events with a probabilistic notion of success or failure. Events may optionally have a controller which defines the series of affordance activations within the smart avatars to produce its desired outcome. We represent this control logic using an extended version of Behavior Trees that facilitate parameterization. Parameterized Behavior Tree (PBT)~\cite{Shoulson:2011:PBT:2177817.2177835} are an effective model for representing coordinated control logic  between multiple smart avatars.

PI Kapadia has extensive prior experience in developing event-centric knowledge bases for inference and reasoning in virtual worlds~\cite{Shoulson:2013:EPA:2522628.2522629,2015-fdg-bstl,Kapadia:2015:CAI:2699276.2699279,Kapadia:2016:CCN:2982818.2982846} which will be extended to represent the IoT domain.


%ADAPT~\cite{Shoulson:2013:EPA:2522628.2522629}
%BSTL~\cite{2015-fdg-bstl,Kapadia:2015:CAI:2699276.2699279}
%behavior authoring~\cite{behaviorCGA,Kapadia:2011:BAC:1944745.1944779}
%story world~\cite{AIIDE1511583}
%CANVAS~\cite{Kapadia:2016:CCN:2982818.2982846}





%$\roleSet{} = \{ \role{i} \}$  define the desired roles for each participant. \role{i} is a logical formula specifying the desired value of the immutable attributes \attributemapping{\cdot}{j} for \smartobj{j} to be considered as a valid candidate for that particular role in the event.


%An event instance $\instance = \langle \event, \smartObjectGroup{} \rangle$ is an event \event populated with an ordered list of smart object participants \smartObjectGroup{}.  $\preconditionFunction{\event}{\smartObjectGroup{}} = \true$. The event postcondition $\postcondition{} : \groupState{} \rightarrow \nextGroupState{}$ transforms the current state of all event participants \groupState{} to \nextGroupState{} by executing the effects of the event. When an event fails, $\nextGroupState{} = \groupState{}$. An event instance $\instance = \langle \event, \smartObjectGroup{} \rangle$ is an event \event populated with an ordered list of smart object participants \smartObjectGroup{}.

%? which represent the specific values of avatar fluents at specific points. Events occur on a set of avatars which are the participants, and produce a change in state of these avatars which are defined as event postconditions. Events may additionally be pre-conditioned on the participant states to indicate what conditions must be satisfied in order for an event to successfully execute.

This symbolic representation of avatars and their dynamic states in terms of key events will facilitate the development of an event calculus ? allowing us make inferences about the relationships and properties of these avatars using first and second-order logic. This reasoning will be used as the basis for knowledge discovery described in Section XX. Knowledge discovery will entail the identification of relationships between the properties of different avatars, clustering of avatar to create compound entities, and the discovery of salient events.




\subsection{Knowledge discovery: social network of avatars (virtual world)}
(Ashfaq + Vladimir)

Tasks:
\begin{itemize}
	\item Build spatio-temporal avatar profiles
	\item Group avatars into groups according to similar profiles. I.e., construct social network of profiles and do graph clustering.
	\item Predict avatar behavior from group profiles
\end{itemize}


We will use Gaussian Process dynamic models (GPDM) as profiles. GPDMs can be used for both modeling isolated avatars and interactions between avatars.  We can use a framework similar to \cite{Shen2015-jr,Shen2015-ft,Shen2012-vi}.  Then extend using our trajectory refinement and optimization approaches \cite{yoon2016}.

However, we need to consider the fact that this system is heterogeneous, i.e., things and humans.  So should have a hierarchical model, one level for things, one for humans, then merge on higher level.



Modeling the state of the cyber-human IoT avatar world is a critical component necessary for optimal IoT behavior prediction and, subsequently, decision making.  This research task focuses on novel data-driven computational models and algorithms that will enable efficient, scalable and accurate estimation of the IoT world state from a (social) network of individual IoT avatars.  In particular, we aim to solve this task by exploiting the parallel between data-driven modeling in IoT avatar worlds and modeling of complex heterogeneous crowd behaviors.


{\bf Prior Art}.  Consider the context of crowds, where the goal is not only to estimate the location of individuals in a group but also to predict group behavior, short (immediate flow) and long-term movement (e.g., crowd moving toward exit A), as well as the interaction of the crowd with the environment (e.g., residents entering of leaving an apartment).  This problem is similar to the one we are facing. However, in the IoT context, the state will not only be represented by physical spatio-temporal locations of entities but also by their general "characteristics", such as the power usage or temperature time course, etc.  Crowd state estimation may focus on macroscopic modeling (coarse flow), mesoscopic (blobs, groups of individuals), or microscopic level (individuals within a group)~\cite{zhan2008}.  Models such as the Discrete Choice~\cite{antonini2006} and the social force (SF)~\cite{helbing2005, pellegrini2009you} model were primarily designed in the context of analytics and simulation. Combining simple person sensing (tracking) with crowd modeling is essential to address the interaction of entities.  Those approaches, e.g.,\cite{bera2014,bera2015,6907095}, use crowd simulators to constrain the dynamics of measured individuals' locations by taking into account collisions.  Recent data-driven approaches~\cite{zhou2012,wang2013} aim to produce more realistic estimates of behaviors but are limited to simple environments and motion patterns and relatively short-term predictions.  Broad environments such as large residential areas, necessitate distributed models to enable sufficient coverage and efficient computation. However, no work to-date considers the detailed aspect of the joint crowd-environment state estimation in long-term planning scenarios, particularly when the states of each entity become truly multidimensional and the crowd is heterogeneous, as is the case in our social networks of IoTs.


{\bf Proposed Research}. This research focuses on the key questions of how to estimate, and subsequently predict, the joint state of the social network of IoT system.  %The estimates must yield individual states of citizens in the crowd (location, etc.), indications of physical interactions with the environment and other citizens (collisions) as well as social interactions (members of the same social group), and the ability to forecast short and long term behavior.  Because of the distributed and multimodal nature of the system, the sensor measurements will be collected and preprocessed locally and then fused to provide a comprehensive view of the system, along with the quantification of uncertainty, necessary for the subsequent stages.
We assume that the sporadic, potentially interrupted, measurements of IoT profiles, called tracklets, can be obtained from multiple sensors using existing acquisition algorithms. However, the observed tracklets will include noise and significant amount of missing information. State of the environment will be similarly measured using other static or dynamic sensors.  The state estimation will then be formulated as: {\bf Task 1: Fusion}.  Fuse and link the local measurements from multiple distributed sensors to identify correspondences between measurements of the same target process.  {\bf Task 2: IoT Crowd State Estimation}. Using the fused/linked measurements estimate the state of the IoT crowd and the environment, taking into account their social or physical interactions. {\bf Task 3: Data-driven Estimation}.  Include historical and/or simulation data to constrain and improve local and global state estimation.  To solve these Tasks we propose a global optimization framework that builds upon our preliminary work in~\cite{yoon2016}. We consider the following setup of the problem.

Consider the crowd-environment state space $\mathcal{X} = \{ ( \mathbf{T}_{i}, l_{i} ), \mathbf{Z} \}$, $\forall i \in (1, J)$  defined as the union of the set of $J$ IoT agent "trajectories" $\mathbf{T}_{i}$ and the environmental configuration $\mathbf{Z}$. The trajectories have the associated agent ID $l_{i} \in \{ 1, ... , J \}$ and temporally ordered agent states
$\mathbf{T}_{i} = \{ \mathbf{t}_{i}^{t} = (x_{i}^{t}, y_{i}^{t}) \},\quad \forall t \in (1, N_{i}),\quad x_{i}^{t}, y_{i}^{t} \in \mathbb{R},$ where $N_{i}$ denotes the number of track points available for the trajectory $i$. The environment is modeled as a set of 2D linear segments, with state $\mathbf{Z} = \{ (z_{k, x_1}^{t}, z_{k, x_2}^{t}, z_{k, y_1}^{t}, z_{k, y_2}^{t}) \},\quad \forall k=(1, N_{k}),$ where e.g., $z_{k,x_1}^{t}, z_{k,x_2}^{t}$ denote the horizontal minimum and maximum bound of the $k$-th linear obstacle at time $t$. For example, a rectangular wall is a set of four linear obstacles.  Our goal will be to reconstruct $\mathbf{T}_{i}$ and $\mathbf{Z}$ from the measurements collected by distributed IoT crowd sensors, tracklets $\mathcal{O} = \{ \mathbf{O}_i \}, i=1,\ldots,M$, and the environment sensors $\mathcal{E} = \{ \mathbf{E}_i \}, i=1,\ldots,N$. Associated with each measurement will be a quantifier of uncertainty, the measurement noise precision $u_{i}^{t}$ (e.g., supposing a Gaussian noise model). 

{\bf Task A.1 -}  We will assume that the association problem,associating each tracklet $\mathbf{O}_i$ with an agent $j \in \{ 1, ... , J \}$, need only be solved for the agents while the associations of measurements with the environment states are known and remain constant.    Typically, local sensor can produce tracklets of reasonable duration along with the profile features $\mathbf{f}_i$ describing this agent (e.g., device model, personal social profile). The linking process amounts to solving a combinatorial min-flow problem on the graph defined by tracklet compatibility costs such as the tracklet profile feature differences (e.g., $\|\mathbf{f}_i - \mathbf{f}_j\|$), spatio-temporal proximity of tracklets, etc. This is a challenging task, with approaches spanning the spectrum of exact solutions based on e.g., the Hungarian algorithm, to more tractable approximations c.f.,~\cite{tron2011b,wang2013,mazzon2013,alahi2014socially} or stochastic solutions such as JPDAF and its extensions, c.f.,~\cite{fortmann1980multi,schulz2003people}.  In this project we will first investigate solutions based on the min-flow problem formulation, which have shown reasonable performance based on our preliminary studies~\cite{yoon2016}.  We will also compare these solutions to stochastic approximations of the JPDAF type, given their ability to handle online and distributed settings~\cite{kamal2013information}.

{\bf Task A.2 -}  Given the estimated associations from Task 1, the goal here is to reconstruct $\mathbf{T}_{i}$ and $\mathbf{Z}$ from the now associated measurements $\mathbf{O}_i, i=1,\ldots,J$, and the environment sensors $\mathbf{E}_i, i=1,\ldots,N$. To illustrate our proposed approach we will focus on the specific case when the environmental configuration is known and static.  We will then first define the agent state estimation as a global multiobjective energy minimization problem.  The choice of the multiple objectives will be driven by specific and unique problem (crowd/environment) considerations.  For example, we will initially consider the following key sub-objectives, summarized in Tab.~\ref{tab:crowd_obj}:


\begin{wrapfigure}{r}{0.4\textwidth}
\vspace{-20pt}
\includegraphics[width=0.38\textwidth]{wacv16_bottleneck.pdf}
\vspace{-10pt}
\caption{\small Preliminary results~\cite{yoon2016} of global microscopic trajectory refinement when 20\% of the tracklets are missing.  Black are the trajectories of 40 agents in a bottleneck benchmark simulated using the SF model, red indicate missing data, blue are refined trajectories after optimization.}
\vspace{-15pt}
\label{fig:bottleneck}
\end{wrapfigure}

\begin{table}
\centering
\caption{Multiple objective terms for global crowd-environment state estimation.}\label{tab:crowd_obj}
\tiny
\begin{tabular}{|l|l|l|l|l|} \hline
	Measurement Compatibility & Kinetic energy & Physical constraint & Social constraint & Environmental constraint \\ 
	
	$E_{gt}(\mathbf{t}_{i}^{t})$ &  
	$E_{kn}(\mathbf{t}_{i}^{t} , \mathbf{t}_{i}^{t+1})$ & 
	$E_{mv}(\mathbf{t}_{i}^{t}, \mathbf{t}_{i}^{t+1})$ &
	$C_{s}(\mathbf{t}_{i}^{t}, \mathbf{t}_{j}^{t}, \mathbf{t}_{i}^{t+1}, \mathbf{t}_{j}^{t+1}, r_{i}, r_{j})$ &
	$C_{e}(\mathbf{t}_{i}^{t}, \mathbf{t}_{i}^{t+1}, r_{i}, \mathbf{z}_{k_{1}}^{t}, \mathbf{z}_{k_{2}}^{t})$ \\ \hline
	
	$ u_{i}^{t} \| \mathbf{t}_{i}^{t} - \mathbf{o}_{i}^{t} \|^{2} $ &
	$ c_{kn} \| \mathbf{t}_{i}^{t+1} - \mathbf{t}_{i}^{t} \|^{2} $  &
	$	\begin{cases}
		0 & \text{if } \| \mathbf{t}_{i}^{t+1} - \mathbf{t}_{i}^{t} \| \le c_{mv}, \\ 
		\infty & \text{otherwise.}
		\end{cases}$ &
		
	$ \begin{cases}
		0 & \text{if } \| \alpha ( \mathbf{t}_{i}^{t+1} - \mathbf{t}_{j}^{t+1} ) + (1 - \alpha) ( \mathbf{t}_{i}^{t} - \mathbf{t}_{j}^{t} ) \|  \\
		&\quad \ge (r_{i} + r_{j}), \quad \forall \alpha \in [0, 1] \\
		\infty & \text{otherwise }
		\end{cases} $ &
	$ \begin{cases}
		0 & \text{if } \| ( \alpha \mathbf{t}_{i}^{t+1} + (1 - \alpha) \mathbf{t}_{i}^{t} ) - ( \beta \mathbf{z}_{k_{1}}^{t} + (1 - \beta) \mathbf{z}_{k_{2}}^{t} ) \|  \\
		&\quad \ge r_{i}, \quad \forall \alpha, \beta \in [0, 1] \\
		\infty & \text{otherwise }
		\end{cases} $ \\ \hline
\end{tabular}
\normalsize	
\end{table}
\emph{Compatibility with Associated Measurements}: dependency of the estimate agent trajectory on the measured tracklets; \emph{Kinetic energy}: the agents whose objective is to reach the goal position following the minimum travelled distance; \emph{Physical constraint}:  the human body set limits on the maximum walking speed $c_{mv}$ of an agent. Depending on the context, these constraints may also include the minimum speed and other biomechanical limitations, c.f.,~\cite{bento2013}; \emph{Social constraint}: Avoidance of collisions with other agents encoded through pairwise constraint functions that depend on the size of agens $r_{i}$; \emph{Environmental constraint}: Collisions between agents and the environment.


Combining the multiple objectives, our global objective will determine each track point of each $i$-th agent by solving the optimization problem
% \small
% \begin{align}
% \argmin_{ \mathbf{t}_{i}^{t} }& \sum_i \sum_{t=1}^{N_{i}} E_{gt} (\mathbf{t}_{i}^{t}) + \sum_{t=1}^{N_{i}-1} E_{kn} (\mathbf{t}_{i}^{t}, \mathbf{t}_{i}^{t+1}) + \sum_{t=1}^{N_{i}-1} E_{mv} (\mathbf{t}_{i}^{t}, \mathbf{t}_{i}^{t+1}) +\sum_{t=1}^{N_{i}} \sum_{i \neq j} C_{s}(\mathbf{t}_{i}^{t}, \mathbf{t}_{j}^{t}, \mathbf{t}_{i}^{t+1}, \mathbf{t}_{j}^{t+1}, r_{i}, r_{j})
% +\sum_{t=1}^{N_{i}} \sum_{(i,k)} C_{e}(\mathbf{t}_{i}^{t}, \mathbf{t}_{i}^{t+1}, r_{i}, \mathbf{z}_{k_{1}}^{t}, \mathbf{z}_{k_{2}}^{t}) \nonumber
% %\\
% %&+\sum_{t=2}^{N_{i}-1} E_{av} (\mathbf{t}_{i}^{t-1}, \mathbf{t}_{i}^{t}, \mathbf{t}_{i}^{t+1}, \mathbf{\tilde{t}}_{m}^{t-1}, \mathbf{\tilde{t}}_{m}^{t}, \mathbf{\tilde{t}}_{m}^{t+1})
% \end{align}
% \normalsize
%\small
%\begin{align}
$\arg\min_{ \mathbf{t}_{i}^{t} } \sum_i \sum_{t=1}^{N_{i}} E_{gt} + \sum_{t=1}^{N_{i}-1} E_{kn}$ $ + \sum_{t=1}^{N_{i}-1} E_{mv} +\sum_{t=1}^{N_{i}} \sum_{i \neq j}  $ $ +\sum_{t=1}^{N_{i}} \sum_{(i,k)} C_{e}.$
%\nonumber
%\end{align}
%\normalsize
However, direct minimization of this global objective is infeasible. In particular, collision constraints are non-convex and we desire to solve this task in a distributed manner, enabling local data processing. Additive objectives of this type are amenable to general distributed optimization using the alternating direction method of multipliers (ADMM). A particular version of that approach, investigated in~\cite{bento2013,bento2015,yoon2016}, focuses on a message-passing solution to ADMM, specifically suitable for networked sensor settings.  Here the global objective can be considered as a consensus optimization problem, where the consensus constraints stem from the need to satisfy pairwise agent-agent and agent-environment interaction "rules."  In Fig.~\ref{fig:bottleneck} we illustrate the effectiveness of this approach from our preliminary studies in~\cite{yoon2016}.  Our method is able to effectively reconstruct large portions of the missing information, while ensuring the reconstructed trajectories are without collisions and discontinuities, and still preserve the original essence of the crowd.  While this approach is appealing, its convergence may be slow (typically on order of minutes).  We will consider recent ADMM acceleration approaches particularly suitable for this setting, including our own distributed adaptive penalty Fast ADMM~\cite{song2016aaai}.  Another challenge will be to extend this point-based estimation approach to a fully probabilistic setting, with quantifiers of posterior uncertainty.  We propose to tackle this task using our recent work on ADMM for probabilistic models~\cite{yoon2012,behnam2016}, which can also be used to construct online versions of the global optimization approach.  We will compare these approaches to state-of-the-art distributed dynamical system methods, including~\cite{Das2013-mk,kamal2013information} and the recent generative-based models of~\cite{zhou2012}.

{\bf Task A.3 -}  One of the main limitations of the global optimization approaches in the limited physical and social constraints (e.g., kinetic, repulsion, etc.) that may not be fully representative of the actual IoT entity behavior.  While one can overcome these drawbacks when the tracklets are accurate and densely measured, sensor failures, etc. can result in poor state estimates.  We propose to use data-driven IoT profile priors to improve the state estimation approach.  To construct such priors we propose to use both the data from actual IoTs as well as the data simulated in our simulators.  However, such "raw" sources of data typically produce dense measurements that do not capture the "essence" of the behaviors, a desideratum for good generalization.  We thus propose to first construct summaries of behavior patterns from the raw data using methods such as dynamic trajectory clustering~\cite{johnson1996,morris2011,zhou2011,cancela2014,xu2015}.  These summaries $\mathcal{\tilde{D}} = \{ \mathbf{\tilde{T}}_{m}, \mathbf{Z} \}$ will be specific to environment configurations $\mathbf{Z}$.  Given the summaries, we will define an additional energy term $E_{av} (\mathbf{t}_{i}^{t-1}, \mathbf{t}_{i}^{t}, \mathbf{t}_{i}^{t+1}, \mathbf{\tilde{t}}_{m}^{t-1}, \mathbf{\tilde{t}}_{m}^{t}, \mathbf{\tilde{t}}_{m}^{t+1}) = \sum_m s_{m,i}^t \| \theta_{i}^{t} - \tilde{\theta}_{m}^{t} \|^{2}$ that models the compatibility of $\mathbf{t}$ with the cluster centers $\mathbf{\tilde{t}}$, where $s_{m,i}^t$ are the compatibility weights.  This energy term will be added as another subobjective to the global optimization in Task 2. Compatibility weights can be either set apriori to estimated in the optimization process (e.g., in an EM-type of recursive estimation).  We will contrast these approaches to other learning-based crowd sensing models, including~\cite{zhou2012} and~\cite{bera2015}. 

Note that all three Task possess clear dependencies.  Therefore, we will also investigate the utility of a closed-loop estimation system where the results of Task 2 and Task 3 can be used to improve the associations in Task 1.
 
%There has been several approaches to incorporate motion prior into tracker to improve accuracy~\cite{hu2008,ali2008,rodriguez2009}. Notable contributions by~\cite{pellegrini2009,eth_biwi_00785} introduce the SF model~\cite{PhysRevE.51.4282} to model the individual, dynamic social behavior in crowd. \cite{eth_biwi_01014} extended this model by proposing a multi-target tracking method that can uniformly include different types of information, e.g. appearance, physical constraints, and the social behavior of pedestrians. However, this model is not a global optimization thus an approximate inference strategy.  The works by Bera and Manocha~\cite{bera2014,bera2015} proposed a real-time iterative trajectory estimation algorithm for medium-density crowds using Kalman filters with a multi-agent motion model based on velocity obstacles~\cite{berg2008}. The key idea is to separate tracking part and waypoint estimation part, and exploit the motion model to do the latter. This framework was further extended to provide real-time, adaptive tracking for crowded scenes~\cite{6907095}.

%It should also be noted that crowd movement simulation is a complementary area to the crowd tracking, and many methods were proposed in the computer graphics community~\cite{kapadia2015virtual}. As seen in the above example prior works, crowd simulators can be used as motion priors for trackers~\cite{PhysRevE.51.4282,berg2008}. Moreover, the output of crowd trackers can be used to train data-driven crowd simulation models, e.g.~\cite{journals/cgf/LernerCL07} as obtaining large scale ground truth information for training crowd models is one of the biggest challenges~\cite{musse2010}.

%{\bf Trajectory Clustering.} Trajectory clustering is a good way to find consistent motion patterns of trajectories. It often done by iterative, unsupervised algorithms~\cite{johnson1996,zhou2011,cancela2014}. Patterns of the clusters have been described as distribution~\cite{johnson1996}, semantic path~\cite{zhou2011,zhou2012,cancela2014} or some model parameters~\cite{morris2011,hu2013}. Probabilistic approaches such as Gaussian process~\cite{ellis2009}, Gaussian mixture with Hidden Markov Models~\cite{morris2011}, Dirichlet process mixture models~\cite{hu2013} are also popular and may provide notion of uncertainty of the outcome, but they often tend to be sensitive to initialization or pre-training~\cite{morris2011}. A recently introduced method~\cite{xu2015} thus proposes a multi-kernel, mean shift based method that is less sensitive to initialization and eliminate the need of pre-defining the number of clusters. 

%{\bf Environmental constraint: Motion pattern} So far, all constraints are directly imposed on observed, input trajectories. However, there are evidences from prior work that trajectory estimated by the motion models can help tracking~\cite{bera2015}. The issue here is that the model generated trajectories in prior works (a) do not always consider environmental configuration and (b) may be too fine-grained such that it conflicts with the real world tracking results if we impose too much constraint on the model generated trajectories. Therefore, we need an interm representation on the \emph{essence} of the crowd motion given the environmental configuration.

%For this purpose, we rely on clustering. Given set of trajectories $\mathcal{\tilde{D}} = \{ \mathbf{\tilde{T}}_{m}, \mathbf{Z} \}$, each trajectory representing a cluster, i.e. motion pattern (note that the trajectories are from the same environmental configuration $\mathbf{Z}$), where $m \in \{ 1, ... , M \}$ denotes the cluster index, we consider to minimize the angular velocity between the trajectory to be refined and the most similar cluster trajectory as
% \begin{align}
% E_{av} (\mathbf{t}_{i}^{t-1}, \mathbf{t}_{i}^{t}, \mathbf{t}_{i}^{t+1}, \mathbf{\tilde{t}}_{m}^{t-1}, \mathbf{\tilde{t}}_{m}^{t}, \mathbf{\tilde{t}}_{m}^{t+1})
% 	&= \| \theta_{i}^{t} - \tilde{\theta}_{m}^{t} \|^{2}
% \end{align}
% where
% \begin{align}
% \theta_{i}^{t}
% 	&= \frac{ ( \mathbf{t}_{i}^{t} - \mathbf{t}_{i}^{t-1} ) \times ( \mathbf{t}_{i}^{t+1} - \mathbf{t}_{i}^{t} )  }{ \| \mathbf{t}_{i}^{t} - \mathbf{t}_{i}^{t-1} \|^{2} } \\
% \tilde{\theta}_{i}^{t}
% 	&= \frac{ ( \mathbf{\tilde{t}}_{m}^{t} - \mathbf{\tilde{t}}_{m}^{t-1} ) \times ( \mathbf{\tilde{t}}_{m}^{t+1} - \mathbf{\tilde{t}}_{m}^{t} )  }{ \| \mathbf{\tilde{t}}_{m}^{t} - \mathbf{\tilde{t}}_{m}^{t-1} \|^{2} }
% \end{align}
% such that
% \begin{align}
% m = \argmin_{m \in \{1, ... , M\}} \sum_{t=1}^{N_{i}} \| \mathbf{t}_{i}^{t} - \mathbf{\tilde{t}}_{m}^{t} \|^{2}.
% \end{align}
% We will discuss how to learn $\mathbf{\tilde{T}}_{m}$ clusters later.



\subsection{Security, privacy and anonymity}



Security poses a great challenge for IoT. In addition to security threats common for networked systems in general, IoT systems are further confronted by the fact that they necessarily rely on wireless signals for communication. We propose to address challenges stemming from the \emph{physical layer} attacks on IoT systems. In particular, while Multi-User Multiple Input Multiple Output (MU-MIMO) systems are considered the ultimate next step in the wireless bandwidth race (because they leverage spatial multiplexing to send independent data streams to multiple terminals simultaneously, effectively improving spatial reuse), they could be utilized as an effective and sophisticated \emph{denial-of-service} tool. Indeed, by creating controlled signals in an area, it can selectively disrupt performance of Wi-Fi networks and sensors in an area, while still making it hard to detect or classify such behavior (precisely due to uneven signal properties at different locations). Such an attack is relevant in many scenarios. Consider a location densely populated with wireless APs. Can one of the APs turn itself into a “rogue” mode to disrupt the other APs, i.e., “clear the air”, and thus improve performance for itself, while staying largely “invisible” to the other APs? Another example is Wi-Fi backscatter, \emph{i.e.}, scenarios where a device backscatters Wi-Fi signals to communicate. In presence of the attack, such communication might be severely degraded, while at the same time it becomes hard to know that the attack is taking place. Given that Wi-Fi backscatter was recently proposed as a solution to communication with implanted medical devices, this scenario significantly raises the bar for addressing this emerging problem. Below, we first outline our current results in this domain, and the outline a research agenda.


{\bf Background and Current Work}

A fundamental requirement in MU-MIMO systems is that up-to-date Channel State Information (CSI), as perceived by the clients, must be obtained by the AP over short time scales. We show that both explicit and implicit \emph{client-fed} CSI feedback open the door to a fundamental vulnerability of MU-MIMO systems. We demonstrate that a single attacker (a Neo) can mount strategic attack, called Matrix-bending, that can dramatically reduce the throughput
of an MU-MIMO system. Moreover, we show that a small number of Neos (or a single Sybil-enabled Neo) can reduce the throughput of an MU-MIMO system to a level \emph{below} that of a corresponding single-user (SU-MIMO) system. Such a degradation translates to a significantly impaired user experience, \emph{e.g.}, increasing the number of perceivable freeze periods while video-streaming by a factor of $10$ and their total duration by a factor of $20$,
spending $34\%$ of the streaming duration frozen, in comparison to merely $1.7\%$ when using MU-MIMO under the same conditions~\cite{mu-mimoUXeffect,mimo13sigcomm}.

The key behind Matrix-bending is a strategically disruptive behavior in which a Neo reports its CSI with the goal of deliberately ``populating'' the clients' signal space as seen by the AP. By doing so, the Neo confuses the AP, which in turn fails to create ``good'' user clusters. In particular, by becoming a desired member of most client groups, and by driving the system towards generating extremely sub-optimal groups, the Neo effectively breaks client selection algorithms, leading to substantial throughput degradation


We demonstrate the following important features of the Matrix-bending attack:
($i$) Matrix-bending is \emph{a low-profile} attack. A Neo does not apply any jamming, and transmits data using only a single antenna. Moreover, it generates CSI that is statistically indistinguishable from the CSI generated by regular clients, and is capable of achieving its full impact while generating traffic at volumes that are more than an order of magnitude smaller than those generated by a regular client.
($ii$) A single Neo has the capacity to considerably decrease MU-MIMO's throughput gains, \emph{i.e}, by up to one third, even in the presence of $100$ legitimate clients. To bring an MU-MIMO system performance to the level of a corresponding single-user system, the number of Neos in the system need not be large, and does not depend on the number of regular clients in the system. Instead, the required number of Neos corresponds to the number of clients in a beamforming group, which is theoretically upper bounded by the number of antennas at an AP, and in practice often bounded by a smaller number, \emph{e.g.}, $4$ in 802.11ac~\cite{std11ac}.

We design and implement a real-time 802.11ac beamforming system on a software-defined radio
platform that allows us to evaluate Matrix-bending on over-the-air 5~GHz channels with complete access to CSI~\cite{warpProject}. We evaluate MU-MIMO client selection
algorithms and policies in the presence of Neos. We show that falsifying the CSI, \emph{i.e.}, the V-matrix, of a single Neo is sufficient to reduce a substantial portion of MU-MIMO's throughput gain, \emph{i.e.}, approximately by one third. Using $3$ Neos, the throughput degrades by approximately $3\times$, often experiencing performance below
the corresponding SU-MIMO system. We find that Matrix-bending's effect is most dominant over APs that attempt to maximize throughput,  while it is expectedly less pronounced for the schemes that do not focus on throughput.

We further study the potential defenses against Matrix-bending.  First, we evaluate random user selection algorithms. While they are less vulnerable to Matrix-bending attacks than other algorithms, such a feature comes with a significant cost: the throughput of such systems decreases below the levels achieved in SU-MIMO systems, even in the absence of a Matrix-bending attack. Second, we evaluate client CSI encryption. We find that Matrix-bending is quite effective in such scenarios when it has \emph{no knowledge about the CSI from other clients.} The key reason is that as long as the Neos' responses are near-orthogonal to each other, they will affect other users as well. Third, we find Matrix-bending to be equally effective when it has no knowledge about its own CSI. In particular, we evaluate a method that utilizes unknown sounding preambles which aim to prevent CSI manipulation. However, we show that all realistic preamble transformations still preserve the \emph{relative signal properties} such that the attack is still effective.

{\bf Research Agenda}

We propose to address challenges stemming from the \emph{physical layer} attacks on IoT systems. We will evaluate the attacks, and defenses, enabled by MU-MIMO systems. In particular, we propose research in the following directions:

\emph{Silent DoS.} In this scenario, the attacker can jam all communication between clients and an AP in both directions, while \emph{quiet-bubbling} the AP’s antennas, therefore preventing it from sensing the disturbance. The key research questions are how to effectively use \emph{implicit} CSI in order to quiet bubble the APs and to understand the impact on the AP’s performance.

\emph{Silent scrambler.} In this scenario, we consider Wi-Fi-enabled backscatter communication, where signals generated by a Wi-Fi source are backscattered by a client device. In this case, the attacker node (an 8-antenna 802.11ac AP) transmits a time-varying signal while quiet-bubbling the Wi-Fi source. The backscatter client-device responds to the scrambled signal it receives, thus disrupting the communication.
\emph{Power-killer attack.} In this scenario, we again assume the above Wi-Fi-enabled backscatter communication. The goal of this attack would be to null the signal at the backscatter device, such that the energy harvested by the backscatter client is insufficient to enable transmissions. The key research question is to determine critical time-scales and its impact on the effectiveness of the attack.

\emph{Countermeasures.} We propose to analyze countermeasures to the above attacks. The most promising approach for the silent DoS attack is to equip APs with an additional, passive antenna, that would collect the information from the environment. Then, we would further conduct statistical analysis to find critical time scales that enable the attack detection with appropriate confidence guarantees. For the silent-scrambler and power-killer attacks, we will pursue solutions that do not require additional hardware. In all cases, we will conduct experiments on a real-time 802.11ac beamforming system on a software-defined radio
platform on over-the-air 5~GHz channels~\cite{warpProject}.



(Farid + Alex)\\
\\
Data security\\
System level security\\
Privacy, anonymity\\
\\
Greedy behavior (?)\\
Key sharing problem\\
Privacy, trust and anonymity \\

=======
\subsection{Knowledge discovery: social network of avatars (virtual world)}
(Ashfaq + Vladamir)\\
….\\
…\\

\subsection{Security, privacy and anonymity}
(Farid + Alex)\\
…\\
Data security\\
System level security\\
Privacy, anonymity\\
\\
Greedy behavior (?)\\
Key sharing problem\\ 
Privacy, trust and anonymity \\

>>>>>>> Stashed changes
\subsection{Implementation and validation }
(All)\\
\\
Letters of Collaboration)\\
Campus wide implementation \\
Letters of Commitment\\
Shall we rely on KAA?\\
APIs: flexibility of formats/schemas\\
<<<<<<< Updated upstream

<<<<<<< HEAD
Given the interdisciplinary nature of the tasks in the proposed project, it is necessary to have well-defined mechanisms for assessing the validity and benefits of our findings. We will base our demonstration on variations of the two methodologically interdependent scenarios used in the introductory section.  We now discuss both the broad categories of evaluation environments, and the details of the particular evaluation methodologies. 
We will adopt a three-fold analysis of the different aspects of our system: (1) agent-based computer simulations, (2) experiments with real humans controlling avatars in shared virtual worlds, and (3) experiments with real humans in actual real-life environments. Each of these experimental paradigms allow us stress our system along a combination of the following three axes: (1) human stress, (2) replicating the noise inherent in using humans as sensors, and (3) scalability ? i.e., the number of participants (humans or agents). Agent-based simulations harness computation to simulate high-stress events in arbitrarily large crowds of computer agents, but are unable to accurately model humans. Shared online virtual worlds provide a unique opportunity to conduct experiments with real humans in controlled settings. While experiments with human subjects in the real world offer the closest possible parallel to reality, these experiments are limited in number of subjects and the extent of stress that we can induce on the participants. However, by leveraging combinations of all 3 paradigms, we are uniquely positioned to evaluate our models and technologies in a rigorous manner. 


\begin{wrapfigure}{R}{0.30\textwidth} \vspace{-3mm}
	\centerline{\includegraphics[width=0.30\textwidth]{evaluation-plan.png}}
	\vspace{-3mm} \caption{\small Evaluation using Agent simulations, Shared Online Virtual Worlds, and Real World Experiments.}
	\label{fig1}
	\vspace{-3mm}
\end{wrapfigure}

\subsubsection{Simulation-based Evaluations: Agents. }

The first part of our efforts will focus on developing simulator for the two demonstration settings. Throughout the different phases of the project we will be capitalizing upon ? as well as augmenting the current capabilities of ? the SIDnet SWANS simulator [195] and SteerSuite: The Human Movement and Crowd Simulator [196], being developed at Northwestern and Rutgers respectively. SIDnet is used to evaluate various routing and tracking protocols, shapes detection, and even to evaluate the benefits of high-level programming constructs for WSN-users [113, 160, 187, 191]. SteerSuite is an open-source framework for simulating human movement, deliberation, and is especially geared towards dense crowd situations. We will integrate the functionality of these two systems and further augment them with the capability of having heterogeneous nodes and capturing various mobility models and communications between subsets of such nodes. In addition, we will augment its visualization component as well as introduce a new class of nodes with the proper interfaces to encode the following important features: (1) camera/video/audio based sensing and (2) human and human-operated sensors.  For the multimodal mashup scenario, we will simulate a simplified version of the ?Gold Miner? using computer imagery, with virtual objects and sensors. 
The main use of the simulator in the initial stages of the proposal will be to test the various impacts of uncertainty (sensors, network structure/communication) on the quality of detecting the novel categories of spatio-temporal predicates (cf. Section 3.4) and the impact of different policies for local state-estimation and information extraction (cf. Section 3.1 and 3.3). We will employ a large variety of crowd datasets to train and validate our computational models, and in particular leverage our collaboration with Prof. Dirk Helbing who has significant expertise in crowd data collection, analysis, and modeling [197].  In particular, we will integrate our simulation infrastructure with the NervousNet platform [198] that is being actively developed by Prof. Helbing and his colleagues as part of the larger effort on the Planetary Nervous System project [199], which provides a large-scale distributed research platform for real-time mining of social activities using heterogeneous sensor networks.


\subsubsection{Task E-2. Game Playing Evaluation: Avatars. }

Agent-based simulations, such as those described above provide an efficient way to test and evaluate the different aspects of our system in an automated fashion. However, in an effort to come closer to actual human experimentation, we will conduct experiments using real human users in shared online virtual environments. The Rutgers team is uniquely positioned to conduct these experiments ? Co-PI Kapadia has developed HeapCraft [200, 2001]: a modular, extensible, and open framework for studying human behavior in shared virtual worlds such as Minecraft. HeapCraft has already been successfully demonstrated to study player behavior and incentivize cooperation in online societies [202, 203]. Additionally, Kapadia and his collaborators have also conducted lab experiments to study crowd evacuations using serious games.
We will augment the existing HeapCraft framework to design the two scenarios described in Task E-3 below. HeapCraft?s modular and extensible nature allows us to easily develop emulators for the technologies being developed here, which can be rigorously and robustly tested in shared virtual worlds using real human users. In addition, we will augment the knowledge-base for emulating behavioral aspects of the avatars using large datasets of preferences and semantic relations from real social networks obtained from 4C Insights Inc. (Letter of Commitment is available in the Supplementary Documents). 
Our experiments will be conducted in two phases: (I) laboratory studies where up to 40 participants will be recruited to participate in a controlled experiment; (II) online server with our tools released to setup a virtual world with hundreds of human-users  expected to participate. 


\subsubsection{Task E-3. Real-World Environment Evaluations.  }

The system we intend to develop is intrinsically oriented to be open source. It will need obviously the contribution of community members to feed, more or less automatically, the database of avatars in the social network of things. It also relies on a proactive contribution of users to improve the rules generator services. Rules generator, indeed, should not be a static repository of rules. Rather, it is a module which dynamically generates and improves rules at runtime. The way to accomplish these tasks, i.e., the software to do it, is based on well assessed algorithms. Nevertheless, it should be also open to improvements by the users. As common practice in open source communities, a third party serves as the booster of the software and the filter of contributions from users. We expect also that manufacturers and retailers will be highly motivated to foster this software so as to maintain the performance of their apparatuses at the high end of the market, with economic consequences. Users, meanwhile, are naturally inclined to communicate their best practices and analysis. They can do so at two levels: (1) just by issuing rules/instructions in a highly simplified script language involving both experience and their logical capabilities of formalizing it in a few well-formed expressions, and (2) by more sophisticated interventions on the open source codes, profiting their high level software experience and their knowledge of cognitive disciplines.\\
The primers of this community will be represented by two kinds of experiments respectively realizing small scale and large scale assets of our system.\\

\textbf{Small scale:}
A mock-up of the system will be physically realized in the lab of one of the partners (IIT?) with virtual extension to the other partners’ labs. Namely the former will consist of:

\begin{itemize}
	\item a set of 4 to 6 household appliances/IoTs connected to a middleware fog.
	\item a middleware fog hosted by a gateway that is connected to Internet.
	\item a social network instance hosted by a local server.
\end{itemize}

Equally relevant features validating the system will be:

(1) its reliability and recovery procedures w.r.t. any kind of inconvenience, ranging from power breakdown to loss of the Internet connection, or even hostile attack from an insider/outsider. The mock-up will also constitute a ground truth reference for the second class of experiments.\\
(2) its manageability on the part of the ordinary people, both in terms of realizing a middleware fog and connecting IoTs/appliances to it, and in terms of efficiently ruling the appliances/IoTs.\\

As for the last aspect, the cognitive capabilities of the networked intelligence will be dynamically improved as the entire project evolves.\\

This mock-up will constitute a continual ground truth reference for the second type of experiments.\\


\textbf{Large scale:}
Due to the great computational effort and massive statistics database required, this social community can survive only if attracts a huge number of members. Therefore, a main task of this project is to promote the functionalities of our infrastructure within the research community. In particular we plan to implement a cluster grouping from 1,000 to 10,000 domestic gateways (middleware fogs). Basing our dimensioning on the lower bound, the cluster should manage a total of 1,000,000 nodes. Many of the middleware functionalities are implemented locally on gateways (fogs). However, most of them needs frequent connections to the social networks of things, for instance to consult catalogues, reference instructions, tuning parameters, etc. We plan to attribute 2 GB of memory mass to each middleware – which is comparable with normal memory supply of smartphones and other smart personal appliances – for a total of 2 Terabytes. Moreover, given the scarce criticality of the ruled functions, we assume the clock stroke to be of the order of 1 second, and we prepare our software to manage a flow of 5,000 requests per second. These requests should be handled through 50 parallel processes entailing 25 GB RAM. As for the social network we plan to use 10 servers with 50GB RAM and 1TB memory for managing their operations in a modular and scalable way.



=======
Given the interdisciplinary nature of the tasks in the proposed project, it is necessary to have well-defined mechanisms for assessing the validity and benefits of our findings. We will base our demonstration on variations of the two methodologically interdependent scenarios used in the introductory section.  We now discuss both the broad categories of evaluation environments, and the details of the particular evaluation methodologies. 
We will adopt a three-fold analysis of the different aspects of our system: (1) agent-based computer simulations, (2) experiments with real humans controlling avatars in shared virtual worlds, and (3) experiments with real humans in actual real-life environments. Each of these experimental paradigms allow us stress our system along a combination of the following three axes: (1) human stress, (2) replicating the noise inherent in using humans as sensors, and (3) scalability ? i.e., the number of participants (humans or agents). Agent-based simulations harness computation to simulate high-stress events in arbitrarily large crowds of computer agents, but are unable to accurately model humans. Shared online virtual worlds provide a unique opportunity to conduct experiments with real humans in controlled settings. While experiments with human subjects in the real world offer the closest possible parallel to reality, these experiments are limited in number of subjects and the extent of stress that we can induce on the participants. However, by leveraging combinations of all 3 paradigms, we are uniquely positioned to evaluate our models and technologies in a rigorous manner. 


\begin{wrapfigure}{R}{0.30\textwidth} \vspace{-3mm}
	\centerline{\includegraphics[width=0.30\textwidth]{evaluation-plan.png}}
	\vspace{-3mm} \caption{\small Evaluation using Agent simulations, Shared Online Virtual Worlds, and Real World Experiments.}
	\label{fig1}
	\vspace{-3mm}
\end{wrapfigure}

\subsubsection{Simulation-based Evaluations: Agents. }

The first part of our efforts will focus on developing simulator for the two demonstration settings. Throughout the different phases of the project we will be capitalizing upon ? as well as augmenting the current capabilities of ? the SIDnet SWANS simulator [195] and SteerSuite: The Human Movement and Crowd Simulator [196], being developed at Northwestern and Rutgers respectively. SIDnet is used to evaluate various routing and tracking protocols, shapes detection, and even to evaluate the benefits of high-level programming constructs for WSN-users [113, 160, 187, 191]. SteerSuite is an open-source framework for simulating human movement, deliberation, and is especially geared towards dense crowd situations. We will integrate the functionality of these two systems and further augment them with the capability of having heterogeneous nodes and capturing various mobility models and communications between subsets of such nodes. In addition, we will augment its visualization component as well as introduce a new class of nodes with the proper interfaces to encode the following important features: (1) camera/video/audio based sensing and (2) human and human-operated sensors.  For the multimodal mashup scenario, we will simulate a simplified version of the ?Gold Miner? using computer imagery, with virtual objects and sensors. 
The main use of the simulator in the initial stages of the proposal will be to test the various impacts of uncertainty (sensors, network structure/communication) on the quality of detecting the novel categories of spatio-temporal predicates (cf. Section 3.4) and the impact of different policies for local state-estimation and information extraction (cf. Section 3.1 and 3.3). We will employ a large variety of crowd datasets to train and validate our computational models, and in particular leverage our collaboration with Prof. Dirk Helbing who has significant expertise in crowd data collection, analysis, and modeling [197].  In particular, we will integrate our simulation infrastructure with the NervousNet platform [198] that is being actively developed by Prof. Helbing and his colleagues as part of the larger effort on the Planetary Nervous System project [199], which provides a large-scale distributed research platform for real-time mining of social activities using heterogeneous sensor networks.


\subsubsection{Task E-2. Game Playing Evaluation: Avatars. }

Agent-based simulations, such as those described above provide an efficient way to test and evaluate the different aspects of our system in an automated fashion. However, in an effort to come closer to actual human experimentation, we will conduct experiments using real human users in shared online virtual environments. The Rutgers team is uniquely positioned to conduct these experiments ? Co-PI Kapadia has developed HeapCraft [200, 2001]: a modular, extensible, and open framework for studying human behavior in shared virtual worlds such as Minecraft. HeapCraft has already been successfully demonstrated to study player behavior and incentivize cooperation in online societies [202, 203]. Additionally, Kapadia and his collaborators have also conducted lab experiments to study crowd evacuations using serious games.
We will augment the existing HeapCraft framework to design the two scenarios described in Task E-3 below. HeapCraft?s modular and extensible nature allows us to easily develop emulators for the technologies being developed here, which can be rigorously and robustly tested in shared virtual worlds using real human users. In addition, we will augment the knowledge-base for emulating behavioral aspects of the avatars using large datasets of preferences and semantic relations from real social networks obtained from 4C Insights Inc. (Letter of Commitment is available in the Supplementary Documents). 
Our experiments will be conducted in two phases: (I) laboratory studies where up to 40 participants will be recruited to participate in a controlled experiment; (II) online server with our tools released to setup a virtual world with hundreds of human-users  expected to participate. 


\subsubsection{Task E-3. Real-World Environment Evaluations.  }

The system we intend to develop is intrinsically oriented to be open source. It will need obviously the contribution of community members to feed, more or less automatically, the database of avatars in the social network of things. It also relies on a proactive contribution of users to improve the rules generator services. Rules generator, indeed, should not be a static repository of rules. Rather, it is a module which dynamically generates and improves rules at runtime. The way to accomplish these tasks, i.e., the software to do it, is based on well assessed algorithms. Nevertheless, it should be also open to improvements by the users. As common practice in open source communities, a third party serves as the booster of the software and the filter of contributions from users. We expect also that manufacturers and retailers will be highly motivated to foster this software so as to maintain the performance of their apparatuses at the high end of the market, with economic consequences. Users, meanwhile, are naturally inclined to communicate their best practices and analysis. They can do so at two levels: (1) just by issuing rules/instructions in a highly simplified script language involving both experience and their logical capabilities of formalizing it in a few well-formed expressions, and (2) by more sophisticated interventions on the open source codes, profiting their high level software experience and their knowledge of cognitive disciplines.\\
The primers of this community will be represented by two kinds of experiments respectively realizing small scale and large scale assets of our system.\\

\textbf{Small scale:}
A mock-up of the system will be physically realized in the lab of one of the partners (IIT?) with virtual extension to the other partners’ labs. Namely the former will consist of:

\begin{itemize}
	\item a set of 4 to 6 household appliances/IoTs connected to a middleware fog.
	\item a middleware fog hosted by a gateway that is connected to Internet.
	\item a social network instance hosted by a local server.
\end{itemize}

Equally relevant features validating the system will be:

(1) its reliability and recovery procedures w.r.t. any kind of inconvenience, ranging from power breakdown to loss of the Internet connection, or even hostile attack from an insider/outsider. The mock-up will also constitute a ground truth reference for the second class of experiments.\\
(2) its manageability on the part of the ordinary people, both in terms of realizing a middleware fog and connecting IoTs/appliances to it, and in terms of efficiently ruling the appliances/IoTs.\\

As for the last aspect, the cognitive capabilities of the networked intelligence will be dynamically improved as the entire project evolves.\\

This mock-up will constitute a continual ground truth reference for the second type of experiments.\\


\textbf{Large scale:}
Due to the great computational effort and massive statistics database required, this social community can survive only if attracts a huge number of members. Therefore, a main task of this project is to promote the functionalities of our infrastructure within the research community. In particular we plan to implement a cluster grouping from 1,000 to 10,000 domestic gateways (middleware fogs). Basing our dimensioning on the lower bound, the cluster should manage a total of 1,000,000 nodes. Many of the middleware functionalities are implemented locally on gateways (fogs). However, most of them needs frequent connections to the social networks of things, for instance to consult catalogues, reference instructions, tuning parameters, etc. We plan to attribute 2 GB of memory mass to each middleware – which is comparable with normal memory supply of smartphones and other smart personal appliances – for a total of 2 Terabytes. Moreover, given the scarce criticality of the ruled functions, we assume the clock stroke to be of the order of 1 second, and we prepare our software to manage a flow of 5,000 requests per second. These requests should be handled through 50 parallel processes entailing 25 GB RAM. As for the social network we plan to use 10 servers with 50GB RAM and 1TB memory for managing their operations in a modular and scalable way.


 
=======
>>>>>>> Stashed changes
>>>>>>> origin/master
