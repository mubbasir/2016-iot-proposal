We now present an overview of the state of the art, both from the perspective of the current properties of devices and networking technology features, as well as from the perspective of related works in several fields that, in one way or another, may be used to achieve our objectives – however, the fall short of several important aspects.

\subsection{Whatever Proper title of this sub-section}
The vast majority of devices are endowed with electronic interfaces, made of buttons and the like, inviting users to operate selections from among a limited set of options pre-featured by the manufacturers. They may concern options, such as, ventilation on/off button in an oven, a selection of delicate option in a washing machine, air-condition on/off button in a car, and so on. While these older devices are considered somehow efficient and simple, the trend is towards more technological and ubiquitous devices. 
The operational innovation of our approach is to make a move of the interface a step ahead and a step farther so as to realize a deep, remote and smart control. In this case, the manufacturer makes all/most of the operational parameters, such as spin speed and duration, totally manageable and configurable by the user. In turn, the user drives these parameters via software, hence makes them obeying to automatic procedures/instructions that have been elaborated and optimized in advance, with the help of the manufacturer, other contributors, and/or the networked intelligence. 
Thus, we need to have this new hardware/software interface managing signals/data back and forth between actuators/sensors and a logic unit with the result of detaching the user from any physical contact with the device (thing). To this end, in the project we will use open source electronic boards, such as Arduino boards (https://www.arduino.cc/), Parallella boards (https://www.parallella.org/) and/or Raspberry-Pi boards (https://www.raspberrypi.org/). They are powerful prototyping boards based on flexible, open and easy-to-use hardware and software. These devices are wirily and/or wirelessly connected to the Internet from which they can be easily visible and accessible. We foresee that their exploitation will lead microelectronic companies to develop specific chips for a massive diffusion of this new technology in a near future.
In order to realize the vision of an Ambient Intelligence in a future network and service environment, heterogeneous wireless sensor and actuator networks have to be integrated into a common framework of global scale and made available to services and applications via universal service interfaces [1]. The goal is to reach a distributed open architecture with interoperability of heterogeneous systems, neutral and easy access, clear layering and resilience. It should provide the necessary network and information management services to enable reliable, secure and accurate interactions with the physical environment. 
The idea is to provide an integrated platform that offers unified data access, processing and services on top of existing ubiquitous services of the Internet of Things to integrate heterogeneous sensors/actuators in a uniform way. From an application perspective, a set of basic services encapsulates sensor/actuator network infrastructures hiding the underlying layers with the network communication details and heterogeneous sensor hardware and lower level protocols. 
A heterogeneous networking environment indeed calls for means to hide the complexity from the end-user, as well as applications, by providing intelligent and adaptable connectivity services, thus providing an efficient application development framework. To face the coexistence of many heterogeneous set of things, a common trend in Internet of Things applications is the adoption of an abstraction layer capable of harmonizing the access to different devices with a common language and procedures [2]. Standard interfaces and data models ensure a high degree of interoperability among multiple systems. However, typical drawbacks of misconfigurations and traffic congestions are normally exasperated by the node heterogeneity. These drawbacks will be overcome in the project through the adoption of islands architecture. On the one hand, smart gateways in their locations will hide all the complexities of the underlying standards. Hence, wireless connections of the devices/machines to the web/Internet, using communication technologies like Zigbee [3], Z-wave [4], Wi-Fi [5], plus the mapping of every device to a unique ID, will provide full, intelligent and secure control over it. With these smart boxes/gateways, technological limitations will completely disappear and the devices will become identifiable only through their functionalities with clear and consistent APIs. On the other hand, the middleware fog constitutes an abstraction level where all the devices, viewed as entities sharing their functionalities (avatars), are transparently managed and used. The goal is to manage the collaboration between heterogeneous devices through a simple API level in conjunction with the mentioned communication protocol able to reach the peer within the location singularly.
Most existing solutions for middleware adopt in general a service-oriented design tailored mainly to support a network topology of sensors that is both unknown and dynamic. But while some projects focus on abstracting the sensors in the network as services (such as in HYDRA [6–8], SENSEI [9], SOCRADES [10], and COBIS [11]), other projects devote more attention to data/information abstractions and their integrations with services (among which are SOFIA1 [12], SATware [13], and Global Sensor Networks GSN [14]). A common thread throughout all of these solutions, however, is that they handle the challenge of an unknown topology through the use of discovery methods that are largely based on the well-known traditional service/resource discovery approaches of the existing Internet, ubiquitous environments and wireless sensor and actuator networks [15–17]. For instance, SOCRADES provides discovery on two levels, the sensor level and the service level, which can employ either standard web-service discovery or a RESTful discovery mechanism (for RESTful services). COBIS, on the other hand, uses its own service description language COBIL 2 (Collaborative Business Item Language), where service functions and keywords are annotated with a verbal description.
Another point of agreement in the state-of-the-art of middleware solutions is in the widespread use of semantics and metadata to overcome heterogeneity challenges. Indeed, it is standard practice to use ontologies to model sensors, their domains, and sensor data repositories [7, 18, 19]. Some projects even go a step further and also include context information [9], or service descriptions [6, 7, 8]. And as a type of service composition, many projects support the concept of virtual/semantic sensors (for instance, in HYDRA, GSN and SATware), i.e. entities that abstract several aggregated physical devices under a single service. A different implementation of a similar idea, though, is provided in the SATware project, where virtual sensors actually correspond to transformations applied to a set of raw sensor streams to produce another semantically meaningful stream.
Regarding scalability, most projects address this challenge by pursuing modifications in the underlying sensor/actuator network topology. Sometimes, this is done by adopting fully-distributed infrastructures (such as in COBIS and SOFIA), and sometimes through an architecture of peer-to-peer clusters (e.g., GSN). In our view, however, while these approaches work well for the existing Internet, where traffic is made up of a relatively small amount of service interactions, they will not fit for the complex weave of interactions that will be commonplace in the Internet of Things. In the Internet of Things, a large number of requests will involve intricate coordination among millions of things and services, whereas on today��s Internet most requests are largely point-to-point. Therefore, the number of packets transmitted in the network will grow strongly and nonlinearly as the number of available services increases. In such an environment, performing even a simple service discovery may exceed acceptable time, processing, and memory constraints. //NOTE: the last paragraph needs to be tied (cf. (2)/Fig.2 in the intro with the problems related to “Possible Worlds Semantics�� and the efficient pruning the needs to be performed 

\subsection{Data and Process Integration}
Heterogeneous data integration is a topic which, in addition to the database community broadly \cite{Cohen98, HalevyRO06}, has also been investigated by more specialized research foci: -- bioinformatics \cite{MostafaviM10, Sujansky02}; information management and text retrieval \cite{Garcia-MolinaHI+95, LebartSB98, Stas14}; sensing and tracking \cite{ChawatheKRS04} (to name but a few…). While the role and the impact of the semantics have been recognized and addressed \cite{BergamaschiCVB01, CastanoFMR04}, the existing results still have limited cross-contextual support. For instance, the works investigating the benefits of semantics for resource discovery (cf. \cite{CastanoFMR04}) do not take into consideration the role of different devices that may enter and/or leave a particular “social” network dynamically, as well as their impact on real-time adjustments (as well as the impact of their cessation).
To-Do – Goce:
Overview of:
a.	Data integration + distributed query processing; workflows; Wireless Actuator Networks
b.	Analytics and platforms in-context (Cloud/Hadoop; NoSQL; Warehousing)
